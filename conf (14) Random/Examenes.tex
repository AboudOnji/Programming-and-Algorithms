\documentclass[aspectratio=169,xcolor=dvipsnames]{beamer}
\usetheme{Berlin} % Elegir un tema de beamer

\usepackage[spanish]{babel}
\usepackage{hyperref}
\usepackage{graphicx} % Allows including images
\usepackage{booktabs} % Allows the use of \toprule, \midrule and \bottomrule in tables
\usepackage{amsmath}
\usepackage{lettrine}
\setbeamertemplate{caption}[numbered]
\usepackage[dvipsnames,svgnames,x11names]{xcolor}% Para definir y usar colores (ej. en hipervínculos)
\usepackage{xurl}
\usepackage{algorithm}
\usepackage{algorithmicx}
\usepackage{algpseudocode}
\hypersetup{
    colorlinks=true,        % Colorear los enlaces en lugar de usar recuadros
    linkcolor=blue,     % Color para enlaces internos (índice, referencias cruzadas)
    filecolor=blue, % Color para enlaces a archivos locales
    urlcolor=blue,      % Color para URLs
    citecolor=blue,     % Color para citas bibliográficas
}
%----------------------------------------------------------------------------------------

\usepackage{listings}
\usepackage{xcolor} % Para colores en listings
 \definecolor{codegreen}{rgb}{0,0.6,0}
 \definecolor{codegray}{rgb}{0.5,0.5,0.5}
 \definecolor{codepurple}{rgb}{0.58,0,0.82}
 \definecolor{backcolour}{rgb}{0.97,0.97,0.99}

\lstdefinestyle{PythonStyle}{
  language=Python,
  basicstyle=\ttfamily\scriptsize,
  keywordstyle=\color{blue}\bfseries,
  commentstyle=\color{codegreen},
  stringstyle=\color{violet},
  numberstyle=\tiny\color{gray},
  breakatwhitespace=false,
  breaklines=true,
  captionpos=b,
  keepspaces=true,
  numbers=left,
  numbersep=5pt,
  showspaces=false,
  showstringspaces=false,
  showtabs=false,
  tabsize=2,
  frame=lines, % Añade un marco alrededor del código
  framerule=0.4pt, % Grosor del marco
  backgroundcolor=\color{backcolour} % Color de fondo suave
}
\lstset{style=PythonStyle}
%	TITLE PAGE
\title{Ejemplos para preparar exámenes en Python}


\author{Prof. D.Sc. BARSEKH-ONJI Aboud}
\institute
{
    Facultad de Ingeniería \\
    Universidad Anáhuac México
}
\date{\today}

%----------------------------------------------------------------------------------------
%   CONTENIDO DE LA PRESENTACIÓN
%----------------------------------------------------------------------------------------

% --- Agenda automática al inicio de cada sección ---
\AtBeginSection[]
{
  \begin{frame}{Agenda}
    \tableofcontents[currentsection]
  \end{frame}
}

\begin{document}

\begin{frame}
    \titlepage
\end{frame}

%------------------------------------------------
\section{Ejemplo 1: Uso de While con números Aleatorios}
%------------------------------------------------
\begin{frame}[fragile]
    \frametitle{Ejemplo 1: Uso de While con números Aleatorios}
    \scriptsize

    \begin{block}{Ejemplo 1}
Una estación meteorológica registra las temperaturas máximas durante 5 semanas y 7 días. 
\begin{itemize}
    \item LLene la tabla (matriz) con valores aleatorios de temperatura entre 15 y 35 (grados). Utilice únicamente ciclo(s) WHILE.
    \item Imprima la tabla completa en forma de matriz, con las temperaturas bien alineadas.
    \item Calcule e imprima la temperatura promedio de la Semana 1.  Utilice únicamente ciclo(s) WHILE.
    \item Calcule e imprima cuántos días de la Semana 5 (fila 4) superaron los 30 grados. Utilice únicamente ciclo(s) WHILE...TRUE.
Nota: Utilice únicamente los ciclos solicitados para cada punto. No cree arreglos o listas adicionales aparte de la matriz principal.
\end{itemize}
    \end{block}
\end{frame}
%------------------------------------------------

\section{Ejemplo 2: Uso While}
\begin{frame}[fragile]
    \frametitle{Ejemplo 2: Uso While}
    \scriptsize

    \begin{block}{Ejemplo 2}
Realice un programa para la siguiente situación: Una taquilla de cine vende boletos para la función de estreno. Ofrece tres tipos de boletos: Adulto: \$85.0 por boleto; Niño: \$60.0 por boleto; 3ra Edad: \$55.0 por boleto. El programa debe ejecutarse de la siguiente manera:
\begin{itemize}
    \item El sistema debe mostrar la lista de precios de los boletos.
    \item El usuario debe poder registrar la venta de boletos, indicando el tipo de boleto y la cantidad que desea comprar.
    \item El sistema debe calcular y mostrar el costo total para esa compra específica.
    \item Después de cada compra, el programa debe preguntar si se desea registrar otra venta. Si el usuario introduce la palabra "FIN", el programa debe detenerse.
    \item Al finalizar (cuando el usuario introduce "FIN"), el sistema debe mostrar un reporte final con:
\begin{itemize}
    \item El total de boletos vendidos de cada tipo (cuántos de Adulto, Niño y 3ra Edad).
    \item El total de ingresos del día (la suma de todas las ventas).
\end{itemize}
\end{itemize}
Restricción: Utilice únicamente ciclo(s) WHILE para resolver todo el problema.
\end{block}
\end{frame}

\section{Ejemplo 3}
\begin{frame}[fragile]
    \frametitle{Ejemplo 3:}
    \scriptsize
    \begin{block}{Ejemplo 3}
        Escriba un programa para una competencia de atletismo que realice las siguientes acciones: 
        \begin{itemize}
            \item Solicite al usuario el nombre de un competidor y los tiempos (en segundos) de sus 6 carreras (vueltas). Guarde los tiempos en un arreglo (lista). Utilice únicamente ciclo(s) WHILE para solicitar los 6 tiempos.
            \item Copia la lista de tiempos en otro arreglo nuevo. Ordene este nuevo arreglo de menor a mayor (del más rápido al más lento).
            \item Imprima el nombre del competidor y ambos arreglos (el original y el ordenado).
            \item Al imprimir el arreglo original, a la derecha de cada tiempo, imprima "RÉCORD" si el tiempo es menor a 9.5 segundos, e imprima "NORMAL" si el tiempo es 9.5 segundos o más. Utilice únicamente ciclo(s) WHILE para imprimir los arreglos.
        \end{itemize}
Nota: Utilice únicamente los ciclos solicitados. No agregue más arreglos que los dos solicitados (el original y la copia ordenada).
    \end{block}
\end{frame}
%------------------------------------------------
\end{document}