\documentclass[aspectratio=169,xcolor=dvipsnames]{beamer}
\usetheme{Berlin} % Elegir un tema de beamer

\usepackage[spanish]{babel}
\usepackage{hyperref}
\usepackage{graphicx} % Allows including images
\usepackage{booktabs} % Allows the use of \toprule, \midrule and \bottomrule in tables
\usepackage{amsmath}
\usepackage{lettrine}
\setbeamertemplate{caption}[numbered]
\usepackage[dvipsnames,svgnames,x11names]{xcolor}% Para definir y usar colores (ej. en hipervínculos)
\usepackage{xurl}
\usepackage{algorithm}
\usepackage{algorithmicx}
\usepackage{algpseudocode}
\hypersetup{
    colorlinks=true,        % Colorear los enlaces en lugar de usar recuadros
    linkcolor=blue,     % Color para enlaces internos (índice, referencias cruzadas)
    filecolor=blue, % Color para enlaces a archivos locales
    urlcolor=blue,      % Color para URLs
    citecolor=blue,     % Color para citas bibliográficas
}
%----------------------------------------------------------------------------------------

\usepackage{listings}
\usepackage{xcolor} % Para colores en listings
 \definecolor{codegreen}{rgb}{0,0.6,0}
 \definecolor{codegray}{rgb}{0.5,0.5,0.5}
 \definecolor{codepurple}{rgb}{0.58,0,0.82}
 \definecolor{backcolour}{rgb}{0.97,0.97,0.99}

\lstdefinestyle{PythonStyle}{
  language=Python,
  basicstyle=\ttfamily\scriptsize,
  keywordstyle=\color{blue}\bfseries,
  commentstyle=\color{codegreen},
  stringstyle=\color{violet},
  numberstyle=\tiny\color{gray},
  breakatwhitespace=false,
  breaklines=true,
  captionpos=b,
  keepspaces=true,
  numbers=left,
  numbersep=5pt,
  showspaces=false,
  showstringspaces=false,
  showtabs=false,
  tabsize=2,
  frame=lines, % Añade un marco alrededor del código
  framerule=0.4pt, % Grosor del marco
  backgroundcolor=\color{backcolour} % Color de fondo suave
}
\lstset{style=PythonStyle}
%	TITLE PAGE
\title{Números Aleatorios en Python}
\subtitle{Usando el Módulo \texttt{random}}

\author{Prof. D.Sc. BARSEKH-ONJI Aboud}
\institute
{
    Facultad de Ingeniería \\
    Universidad Anáhuac México
}
\date{\today}

%----------------------------------------------------------------------------------------
%   CONTENIDO DE LA PRESENTACIÓN
%----------------------------------------------------------------------------------------

% --- Agenda automática al inicio de cada sección ---
\AtBeginSection[]
{
  \begin{frame}{Agenda}
    \tableofcontents[currentsection]
  \end{frame}
}

\begin{document}

\begin{frame}
    \titlepage
\end{frame}

%------------------------------------------------
\section{Introducción al Módulo \texttt{random}}
%------------------------------------------------

\begin{frame}[fragile]
    \frametitle{¿Qué es el Módulo \texttt{random}?}
    
    \begin{block}{Definición}
    El módulo \texttt{random} es una herramienta incorporada en Python que proporciona funciones para generar números pseudoaleatorios (PRNGs).
    \end{block}
    
    \begin{itemize}
        \item \textbf{"Pseudoaleatorio"} significa que los números parecen aleatorios, pero se generan a partir de un algoritmo determinista (basado en una "semilla" o \textit{seed}).
        \item Para la mayoría de las tareas (simulaciones, juegos, etc.), esta aleatoriedad es perfectamente suficiente.
    \end{itemize}
\end{frame}
\begin{frame}[fragile]
    \frametitle{¿Qué es el Módulo \texttt{random}?}
    \begin{alertblock}{Primer Paso: Importar}
    Para usar cualquiera de sus funciones, siempre debemos empezar por importar el módulo:
    \begin{lstlisting}[language=Python]
import random
    \end{lstlisting}
    \end{alertblock}
\end{frame}

%------------------------------------------------
\section{Tarea 1: Enteros entre Límites}
%------------------------------------------------

\begin{frame}[fragile] % "fragile" es necesario por el verbatim
    \frametitle{Tarea 1: Generar Enteros con \texttt{random.randint()}}
    
    \begin{block}{El Problema}
    Necesitamos un número entero aleatorio dentro de un rango específico, como simular el lanzamiento de un dado de 6 caras (números del 1 al 6).
    \end{block}
    
    \begin{exampleblock}{La Solución: \texttt{random.randint(a, b)}}
    Esta función toma dos argumentos, \texttt{a} (límite inferior) y \texttt{b} (límite superior).
    \begin{itemize}
        \item \textbf{Importante:} El rango es \textbf{inclusivo}. Incluye tanto \texttt{a} como \texttt{b} en los posibles resultados.
    \end{itemize}
    \end{exampleblock}

\end{frame}

\begin{frame}[fragile] % "fragile" es necesario por el verbatim
    \frametitle{Tarea 1: Generar Enteros con \texttt{random.randint()}}
    \begin{block}{Código de Ejemplo}
\begin{lstlisting}[language=Python]
import random

# Simular un dado de 6 caras (1 al 6)
dado_d6 = random.randint(1, 6)
print(f"Resultado del dado D6: {dado_d6}")

# Simular un dado de 20 caras (1 al 20)
dado_d20 = random.randint(1, 20)
print(f"Resultado del dado D20: {dado_d20}")
\end{lstlisting}
    \end{block}

\end{frame}

%------------------------------------------------
\section{Tarea 2: Elegir de una Lista}
%------------------------------------------------

\begin{frame}[fragile]
    \frametitle{Tarea 2: Elegir un Elemento con \texttt{random.choice()}}
    
    \begin{block}{El Problema}
    Tenemos una colección de opciones (por ejemplo, una lista de nombres o de jugadas) y necesitamos seleccionar solo una de ellas al azar.
    \end{block}
    
    \begin{exampleblock}{La Solución: \texttt{random.choice(secuencia)}}
    Esta función toma una secuencia (como una \texttt{list} o \texttt{tuple}) y devuelve un elemento elegido uniformemente al azar de esa secuencia.
    \end{exampleblock}
\end{frame}

\begin{frame}[fragile]
    \frametitle{Tarea 2: Elegir un Elemento con \texttt{random.choice()}}
    \begin{block}{Código de Ejemplo}
\begin{lstlisting}[language=Python]
import random

opciones = ["piedra", "papel", "tijera"]
jugador = random.choice(opciones)

print(f"El jugador eligió: {jugador}")

# Ejemplo con un sorteo
participantes = ["Ana", "Bruno", "Carla", "David"]
ganador = random.choice(participantes)
print(f"¡El ganador del sorteo es: {ganador}!")
\end{lstlisting}
    \end{block}
    

\end{frame}

%------------------------------------------------
\section{Resumen}
%------------------------------------------------

\begin{frame}[fragile]
    \frametitle{Resumen de Funciones Clave}
    
    \begin{block}{Las Dos Funciones Esenciales}
    Con solo estas dos funciones, podemos cubrir una gran cantidad de casos de uso comunes:
    \begin{itemize}
        \item \textbf{\texttt{import random}}
            \begin{itemize}
                \item Es el primer paso obligatorio.
            \end{itemize}
        \item \textbf{\texttt{random.randint(a, b)}}
            \begin{itemize}
                \item Devuelve un entero \textbf{entre} \texttt{a} y \texttt{b}.
                \item \textbf{Inclusivo} en ambos extremos (incluye \texttt{a} e incluye \texttt{b}).
            \end{itemize}
        \item \textbf{\texttt{random.choice(lista)}}
            \begin{itemize}
                \item Devuelve un elemento \textbf{dentro} de la \texttt{lista}.
                \item La lista debe contener al menos un elemento.
            \end{itemize}
    \end{itemize}
    \end{block}
\end{frame}

\end{document}