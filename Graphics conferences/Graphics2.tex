\documentclass[aspectratio=169,xcolor=dvipsnames]{beamer}
\usetheme{Berlin}

\usepackage[spanish]{babel}
\usepackage{hyperref}
\usepackage{graphicx} % Allows including images
\usepackage{booktabs} % Allows the use of \toprule, \midrule and \bottomrule in tables
\usepackage{amsmath}
\usepackage{lettrine}
\setbeamertemplate{caption}[numbered]
\usepackage[dvipsnames,svgnames,x11names]{xcolor}% Para definir y usar colores (ej. en hipervínculos)
\usepackage{xurl}
\usepackage{hyperref}       % Para crear hipervínculos internos y externos
\usepackage{algorithm}
\usepackage{algorithmicx}
\usepackage{algpseudocode}
\hypersetup{
    colorlinks=true,        % Colorear los enlaces en lugar de usar recuadros
    linkcolor=blue,     % Color para enlaces internos (índice, referencias cruzadas)
    filecolor=blue, % Color para enlaces a archivos locales
    urlcolor=blue,      % Color para URLs
    citecolor=blue,     % Color para citas bibliográficas
}
%----------------------------------------------------------------------------------------

\usepackage{listings}
\usepackage{xcolor} % Para colores en listings
 \definecolor{codegreen}{rgb}{0,0.6,0}
 \definecolor{codegray}{rgb}{0.5,0.5,0.5}
 \definecolor{codepurple}{rgb}{0.58,0,0.82}
 \definecolor{backcolour}{rgb}{0.97,0.97,0.99}

\lstdefinestyle{PythonStyle}{
  language=Python,
  basicstyle=\ttfamily\scriptsize,
  keywordstyle=\color{blue}\bfseries,
  commentstyle=\color{codegreen},
  stringstyle=\color{violet},
  numberstyle=\tiny\color{gray},
  breakatwhitespace=false,
  breaklines=true,
  captionpos=b,
  keepspaces=true,
  numbers=left,
  numbersep=5pt,
  showspaces=false,
  showstringspaces=false,
  showtabs=false,
  tabsize=2,
  frame=lines, % Añade un marco alrededor del código
  framerule=0.4pt, % Grosor del marco
  backgroundcolor=\color{backcolour} % Color de fondo suave
}
\lstset{style=PythonStyle}
%	TITLE PAGE
\title{Interfaces Gráficas (GUI) con Tkinter}
\subtitle{Clase 2: Interactividad y Controles}

\author{Prof. D.Sc. BARSEKH-ONJI Aboud}
\institute
{
    Facultad de Ingeniería \\
    Universidad Anáhuac México
}
\date{\today}

%----------------------------------------------------------------------------------------
%   CONTENIDO DE LA PRESENTACIÓN
%----------------------------------------------------------------------------------------

% --- Agenda automática al inicio de cada sección ---
\AtBeginSection[]
{
  \begin{frame}{Agenda}
    \tableofcontents[currentsection]
  \end{frame}
}

\begin{document}

\begin{frame}
    \titlepage
\end{frame}

%------------------------------------------------
\section{Eventos y Funciones}
%------------------------------------------------

\begin{frame}[fragile]
    \frametitle{Dando Vida a los Widgets}
    
    \begin{block}{Repaso de la Clase 1}
    En la clase anterior, creamos una ventana con ''widgets'' (controles), pero eran \textbf{estáticos}. El botón estaba ahí, pero hacer clic en él no provocaba ninguna acción.
    \end{block}
    
    \begin{alertblock}{El Evento \texttt{command}}
    La forma más simple de hacer que un widget sea interactivo es a través de su propiedad \texttt{command}.
    \begin{itemize}
        \item La propiedad \texttt{command} se usa principalmente en los botones.
        \item Le decimos al botón: ''Cuando alguien te haga clic, ejecuta esta función de Python''.
        \item A esto se le llama \textbf{''callback''} (llamada de vuelta).
    \end{itemize}
    \end{alertblock}

\end{frame}

%------------------------------------------------

\begin{frame}[fragile]
    \frametitle{Conectando un Botón a una Función}
    
    \begin{block}{Paso 1: Definir la Función}
    Primero, creamos una función de Python (con \texttt{def}) que contenga el código que queremos ejecutar.
    \begin{lstlisting}[language=Python]
def decir_hola():
    print("¡Hola! Has hecho clic en el boton.")
    \end{lstlisting}
    \end{block}
\end{frame}

\begin{frame}[fragile]
    \frametitle{Conectando un Botón a una Función}  
    \begin{alertblock}{Paso 2: Conectar el Botón con \texttt{command}}
    Al crear el botón, le asignamos el \textbf{nombre de la función} a su propiedad \texttt{command}.
    \begin{lstlisting}[language=Python]
# ¡MUY IMPORTANTE!
# Se pasa el nombre de la funcion SIN parentesis:
#
# Correcto:   command=decir_hola
# Incorrecto: command=decir_hola()

boton = tk.Button(ventana, 
                  text="Haz Clic",
                  command=decir_hola)
    \end{lstlisting}
    \end{alertblock}

\end{frame}

%------------------------------------------------
\section{Recolectando Datos del Usuario}
%------------------------------------------------

\begin{frame}[fragile]
    \frametitle{Control de Entrada: \texttt{tk.Entry}}
    
    \begin{block}{¿Qué es un Widget \texttt{Entry}?}
    Es el widget estándar para que el usuario pueda \textbf{ingresar una sola línea de texto} (como un nombre, una contraseña o un número). Es el equivalente gráfico de la función \texttt{input()} de la consola.
    \end{block}
\end{frame}
\begin{frame}[fragile]
    \frametitle{Control de Entrada: \texttt{tk.Entry}}    
    \begin{alertblock}{Creación del Widget}
    Se crea de la misma forma que un \texttt{Label} o \texttt{Button}.
    \begin{lstlisting}[language=Python]
import tkinter as tk
ventana = tk.Tk()

# Crear un campo de entrada
etiqueta_nombre = tk.Label(ventana, text=''Nombre:'')
entrada_nombre = tk.Entry(ventana)

etiqueta_nombre.pack()
entrada_nombre.pack()

ventana.mainloop()
    \end{lstlisting}
    \end{alertblock}

\end{frame}

%------------------------------------------------

\begin{frame}[fragile]
    \frametitle{Obteniendo el Texto: El Método \texttt{.get()}}
    
    \begin{block}{¿Cómo leemos lo que el usuario escribió?}
    El widget \texttt{Entry} tiene un método especial llamado \texttt{.get()} que nos permite \textbf{extraer} el texto que contiene en ese momento.
    \end{block}
\end{frame}
\begin{frame}[fragile]
    \frametitle{Obteniendo el Texto: El Método \texttt{.get()}}
       
    \begin{alertblock}{Uso de \texttt{.get()}}
    La llamada a \texttt{.get()} se hace típicamente \textbf{dentro de la función} que es llamada por el botón.
    \begin{lstlisting}[language=Python]
def on_button_click():
    nombre_usuario = entrada_nombre.get()
    
    print(f''El usuario escribio: {nombre_usuario}'')


boton = tk.Button(ventana, 
                  text=''Enviar'', 
                  command=on_button_click)
    \end{lstlisting}
    \end{alertblock}

\end{frame}

%------------------------------------------------
\section{Actualizando la Interfaz en Vivo}
%------------------------------------------------

\begin{frame}[fragile]
    \frametitle{Modificando Widgets: El Método \texttt{.config()}}
    
    \begin{block}{El Problema}
    Hacer \texttt{print()} en la consola no es una verdadera aplicación gráfica. Queremos que la propia ventana \textbf{reaccione y cambie}.
    \end{block}
\end{frame}

\begin{frame}[fragile]
    \frametitle{Modificando Widgets: El Método \texttt{.config()}}
        
    \begin{alertblock}{La Solución: \texttt{.config()}}
    Todos los widgets tienen un método \texttt{.config()} que nos permite \textbf{cambiar sus propiedades} (como el texto) \textbf{después} de que han sido creados.
    \begin{lstlisting}[language=Python]

def on_button_click():
    # Cambiamos el texto de 'etiqueta_saludo'
    etiqueta_saludo.config(text="¡Has hecho clic!")

# Creamos una etiqueta...
etiqueta_saludo = tk.Label(ventana, text="Esperando clic...")
etiqueta_saludo.pack()

# ...y un boton que la controla
boton = tk.Button(ventana, command=on_button_click)
boton.pack()
    \end{lstlisting}
    \end{alertblock}

\end{frame}

%------------------------------------------------
\section{Ejemplo Completo: App ``Saludador''}
%------------------------------------------------

\begin{frame}[fragile]
    \frametitle{Proyecto: Aplicación ''Saludador''}
    
    \begin{block}{Objetivo}
    Combinar todo lo aprendido:
    \begin{enumerate}
        \item Un \texttt{Label} y un \texttt{Entry} para pedir un nombre.
        \item Un \texttt{Button} para ejecutar la acción.
        \item Un \texttt{Label} vacío que se actualizará con el saludo.
    \end{enumerate}
    \end{block}
\end{frame}
\begin{frame}[fragile]
    \frametitle{Proyecto: Aplicación ''Saludador''}
        
    \begin{alertblock}{Código Completo}
    \begin{lstlisting}[language=Python]
import tkinter as tk
def saludar():
    nombre = entrada_nombre.get()
    saludo_final = f"¡Hola, {nombre}!"
    etiqueta_resultado.config(text=saludo_final)
ventana = tk.Tk()
ventana.title("Saludador v1.0")
# --- Widgets ---
etiqueta_instruccion = tk.Label(ventana, text="Escribe tu nombre:")
entrada_nombre = tk.Entry(ventana)
boton_saludar = tk.Button(ventana, text="Saludar", command=saludar)
etiqueta_resultado = tk.Label(ventana, text="")
# --- Posicionamiento ---
etiqueta_instruccion.pack()
entrada_nombre.pack()
boton_saludar.pack()
etiqueta_resultado.pack()
ventana.mainloop()# --- Bucle principal ---
    \end{lstlisting}
    \end{alertblock}

\end{frame}

%------------------------------------------------
\section{Tarea 2}
%------------------------------------------------

\begin{frame}[fragile]
    \frametitle{Tarea 2: Mini-Calculadora (Sumadora)}

    \begin{block}{Objetivo}
    Crear una aplicación gráfica que funcione como una sumadora simple, aplicando todo lo visto en clase.
    \end{block}
\end{frame}
\begin{frame}[fragile]
    \frametitle{Tarea 2: Mini-Calculadora (Sumadora)}
    \begin{alertblock}{Instrucciones}
    Tu programa debe tener la siguiente interfaz:
    \begin{itemize}
        \item Una etiqueta \texttt{tk.Label} con el texto ''Número 1:''.
        \item Un campo de entrada \texttt{tk.Entry} para el primer número.
        \item Una etiqueta \texttt{tk.Label} con el texto ''Número 2:''.
        \item Un segundo campo de entrada \texttt{tk.Entry} para el segundo número.
        \item Un botón \texttt{tk.Button} con el texto ''Sumar''.
        \item Una etiqueta \texttt{tk.Label} (inicialmente vacía) donde se mostrará el resultado (ej: ''Resultado: 15'').
    \end{itemize}
    \end{alertblock}
 \end{frame}
 \begin{frame}[fragile]
    \frametitle{Tarea 2: Mini-Calculadora (Sumadora)}   
    \begin{exampleblock}{Pistas Clave}
    \begin{itemize}
        \item El método \texttt{.get()} \textbf{siempre devuelve un string}.
        \item Deberás \textbf{convertir} esos strings a números (con \texttt{int()} o \texttt{float()}) antes de poder sumarlos.
        \item Deberás crear una función (ej: \texttt{def sumar():}) y conectarla al \texttt{command} del botón.
    \end{itemize}
    \end{exampleblock}

\end{frame}

\end{document}