\documentclass[aspectratio=169,xcolor=dvipsnames]{beamer}
\usetheme{Berlin}

\usepackage[spanish]{babel}
\usepackage{hyperref}
\usepackage{graphicx} % Allows including images
\usepackage{booktabs} % Allows the use of \toprule, \midrule and \bottomrule in tables
\usepackage{amsmath}
\usepackage{lettrine}
\setbeamertemplate{caption}[numbered]
\usepackage[dvipsnames,svgnames,x11names]{xcolor}% Para definir y usar colores (ej. en hipervínculos)
\usepackage{xurl}
\usepackage{hyperref}       % Para crear hipervínculos internos y externos
\usepackage{algorithm}
\usepackage{algorithmicx}
\usepackage{algpseudocode}
\hypersetup{
    colorlinks=true,        % Colorear los enlaces en lugar de usar recuadros
    linkcolor=blue,     % Color para enlaces internos (índice, referencias cruzadas)
    filecolor=blue, % Color para enlaces a archivos locales
    urlcolor=blue,      % Color para URLs
    citecolor=blue,     % Color para citas bibliográficas
}
%----------------------------------------------------------------------------------------

\usepackage{listings}
\usepackage{xcolor} % Para colores en listings
 \definecolor{codegreen}{rgb}{0,0.6,0}
 \definecolor{codegray}{rgb}{0.5,0.5,0.5}
 \definecolor{codepurple}{rgb}{0.58,0,0.82}
 \definecolor{backcolour}{rgb}{0.97,0.97,0.99}

\lstdefinestyle{PythonStyle}{
  language=Python,
  basicstyle=\ttfamily\scriptsize,
  keywordstyle=\color{blue}\bfseries,
  commentstyle=\color{codegreen},
  stringstyle=\color{violet},
  numberstyle=\tiny\color{gray},
  breakatwhitespace=false,
  breaklines=true,
  captionpos=b,
  keepspaces=true,
  numbers=left,
  numbersep=5pt,
  showspaces=false,
  showstringspaces=false,
  showtabs=false,
  tabsize=2,
  frame=lines, % Añade un marco alrededor del código
  framerule=0.4pt, % Grosor del marco
  backgroundcolor=\color{backcolour} % Color de fondo suave
}
\lstset{style=PythonStyle}
%	TITLE PAGE
\title{Interfaces Gráficas (GUI) con Tkinter}
\subtitle{Clase 1: "Hola, Mundo" Gráfico}

\author{Prof. D.Sc. BARSEKH-ONJI Aboud}
\institute
{
    Facultad de Ingeniería \\
    Universidad Anáhuac México
}
\date{\today}

%----------------------------------------------------------------------------------------
%   CONTENIDO DE LA PRESENTACIÓN
%----------------------------------------------------------------------------------------

% --- Agenda automática al inicio de cada sección ---
\AtBeginSection[]
{
  \begin{frame}{Agenda}
    \tableofcontents[currentsection]
  \end{frame}
}

\begin{document}

\begin{frame}
    \titlepage
\end{frame}

%------------------------------------------------
\section{Fundamentos de GUI y Tkinter}
%------------------------------------------------

\begin{frame}[fragile]
    \frametitle{¿Qué es una Aplicación Gráfica (GUI)?}
    
    \begin{columns}[t]
        \column{.48\textwidth}
            \begin{block}{Programa de Consola (Lo que conocemos)}
                \begin{itemize}
                    \item Se ejecuta en una terminal.
                    \item La interacción es solo texto.
                    \item Flujo lineal: \texttt{input()} (espera), procesa, \texttt{print()} (muestra).
                \end{itemize}
            \end{block}

        \column{.48\textwidth}
            \begin{alertblock}{Aplicación Gráfica (GUI)}
                \begin{itemize}
                    \item \textbf{G}raphical \textbf{U}ser \textbf{I}nterface.
                    \item Se ejecuta en una \textbf{ventana}.
                    \item La interacción es visual: botones, menús, clics del mouse.
                    \item \textbf{Dirigida por eventos}: El programa no tiene un flujo lineal, sino que \textbf{reacciona} a las acciones del usuario.
                \end{itemize}
            \end{alertblock}
    \end{columns}
\end{frame}
\begin{frame}[fragile]
    \frametitle{¿Qué es una Aplicación Gráfica (GUI)?}
    
    \begin{exampleblock}{Nuestra Herramienta: Tkinter}
    \texttt{Tkinter} es la biblioteca \textbf{estándar} de Python para crear interfaces gráficas. ¡Viene incluida, no hay que instalar nada!
    \end{exampleblock}

\end{frame}

%------------------------------------------------

\begin{frame}[fragile]
    \frametitle{El ''Formulario'' o Ventana Principal}
    
    \begin{block}{El Lienzo de Nuestra Aplicación}
    Toda aplicación gráfica necesita una \textbf{ventana raíz} o principal (también llamada "formulario"). Es el contenedor donde vivirá todo lo demás (botones, texto, etc.).
    \end{block}

    \begin{alertblock}{El Bucle de Eventos}
    A diferencia de un script normal, una GUI debe \textbf{mantenerse abierta} esperando a que el usuario haga algo (un clic, escribir, etc.).
    \begin{itemize}
        \item Usamos \texttt{ventana.mainloop()} al final.
        \item Este comando inicia el "bucle de eventos" que mantiene la ventana visible y receptiva.
    \end{itemize}
    \end{alertblock}
\end{frame}
\begin{frame}[fragile]
    \frametitle{El ''Formulario'' o Ventana Principal}
    \begin{exampleblock}{El Código Mínimo para una Ventana}
    \begin{lstlisting}[language=Python]
import tkinter as tk

# 1. Crear la ventana principal (formulario)
ventana = tk.Tk()

# (Aqui iria el contenido de la ventana)

# 2. Iniciar el bucle de eventos
ventana.mainloop()

# El programa se pausara aqui hasta que
# el usuario cierre la ventana.
    \end{lstlisting}
    \end{exampleblock}

\end{frame}

%------------------------------------------------
\section{Nuestro "Hola, Mundo" Gráfico}
%------------------------------------------------

\begin{frame}[fragile]
    \frametitle{Objetos o "Widgets"}
    
    \begin{block}{¿Qué es un Widget?}
    Un "widget" (o control) es cualquier \textbf{objeto} o elemento visual que ponemos dentro de nuestra ventana para interactuar con el usuario.
    \end{block}
\end{frame}
\begin{frame}[fragile]
    \frametitle{Objetos o "Widgets"} 
    \begin{columns}[t]
        \column{.48\textwidth}
            \begin{alertblock}{Widget: \texttt{tk.Label}}
                Una \textbf{Etiqueta}. Se usa simplemente para mostrar texto (o imágenes) al usuario. No es interactiva.
                \begin{lstlisting}[language=Python]

etiqueta = tk.Label(ventana, text="Hola Mundo")
                \end{lstlisting}
            \end{alertblock}

        \column{.48\textwidth}
            \begin{examples}{Widget: \texttt{tk.Button}}
                Un \textbf{Botón}. El usuario puede hacer clic en él para disparar una acción o \textbf{evento}.
                 \begin{lstlisting}[language=Python]
# Sintaxis:
# Button(donde_vive, propiedades...)
boton = tk.Button(ventana, text="Haz Clic")
                \end{lstlisting}
            \end{examples}
    \end{columns}

\end{frame}

%------------------------------------------------

\begin{frame}[fragile]
    \frametitle{Crear y Posicionar: El Método \texttt{.pack()}}
    
    \begin{block}{Un Proceso de Dos Pasos}
    Para que un widget aparezca, siempre debemos seguir dos pasos:
    \begin{enumerate}
        \item \textbf{Crear} el widget (como vimos: \texttt{tk.Label(...)}).
        \item \textbf{Posicionarlo} en la ventana.
    \end{enumerate}
    \end{block}
\end{frame}
\begin{frame}[fragile]
    \frametitle{Crear y Posicionar: El Método \texttt{.pack()}}
    \begin{alertblock}{El Posicionador \texttt{.pack()}}
    \texttt{.pack()} es el administrador de geometría más simple. Simplemente ''empaqueta'' los widgets uno debajo del otro, en el orden en que los llamas.
    \begin{lstlisting}[language=Python]
import tkinter as tk
ventana = tk.Tk()

# 1. Crear la etiqueta
etiqueta = tk.Label(ventana, text="¡Hola, Tkinter!")
# 2. Posicionar la etiqueta
etiqueta.pack()
# 1. Crear el boton
boton = tk.Button(ventana, text="Soy un boton")
# 2. Posicionar el boton
boton.pack()
ventana.mainloop()
    \end{lstlisting}
    \end{alertblock}

\end{frame}

%------------------------------------------------
\section{Tarea 1}
%------------------------------------------------

\begin{frame}[fragile]
    \frametitle{Tarea 1: Tu Primera Ventana}

    \begin{block}{Objetivo}
    Crear tu primera aplicación gráfica funcional, aplicando los conceptos de \textbf{ventana}, \textbf{widgets} y \textbf{posicionamiento}.
    \end{block}
\end{frame}
\begin{frame}[fragile]
    \frametitle{Tarea 1: Tu Primera Ventana}
    \begin{alertblock}{Instrucciones}
    Escribe un programa de Python que haga lo siguiente:
    \begin{enumerate}
        \item Importa la biblioteca \texttt{tkinter} (usa el alias \texttt{tk}).
        \item Crea la ventana principal (formulario) llamada \texttt{ventana}.
        \item \textbf{(Nuevo)} Dale un título a tu ventana usando: \texttt{ventana.title("Mi Aplicación")}
        \item Crea un \texttt{tk.Label} que viva en \texttt{ventana} y muestre el texto: "¡Bienvenido a mi primera GUI!".
        \item Crea un \texttt{tk.Button} que viva en \texttt{ventana} y muestre el texto: "No hago nada (aún)".
        \item Usa el método \texttt{.pack()} en la etiqueta y luego en el botón para que aparezcan.
        \item Inicia el bucle de eventos con \texttt{ventana.mainloop()}.
    \end{enumerate}
    \end{alertblock}

\end{frame}


\end{document}