\documentclass[aspectratio=169,xcolor=dvipsnames]{beamer}
\usetheme{Berlin}

\usepackage[spanish]{babel}
\usepackage{hyperref}
\usepackage{graphicx} % Allows including images
\usepackage{booktabs} % Allows the use of \toprule, \midrule and \bottomrule in tables
\usepackage{amsmath}
\usepackage{lettrine}
\setbeamertemplate{caption}[numbered]
\usepackage[dvipsnames,svgnames,x11names]{xcolor}% Para definir y usar colores (ej. en hipervínculos)
\usepackage{xurl}
\usepackage{hyperref}       % Para crear hipervínculos internos y externos
\usepackage{algorithm}
\usepackage{algorithmicx}
\usepackage{algpseudocode}
\hypersetup{
    colorlinks=true,        % Colorear los enlaces en lugar de usar recuadros
    linkcolor=blue,     % Color para enlaces internos (índice, referencias cruzadas)
    filecolor=blue, % Color para enlaces a archivos locales
    urlcolor=blue,      % Color para URLs
    citecolor=blue,     % Color para citas bibliográficas
}
%----------------------------------------------------------------------------------------

\usepackage{listings}
\usepackage{xcolor} % Para colores en listings
 \definecolor{codegreen}{rgb}{0,0.6,0}
 \definecolor{codegray}{rgb}{0.5,0.5,0.5}
 \definecolor{codepurple}{rgb}{0.58,0,0.82}
 \definecolor{backcolour}{rgb}{0.97,0.97,0.99}

\lstdefinestyle{PythonStyle}{
  language=Python,
  basicstyle=\ttfamily\scriptsize,
  keywordstyle=\color{blue}\bfseries,
  commentstyle=\color{codegreen},
  stringstyle=\color{violet},
  numberstyle=\tiny\color{gray},
  breakatwhitespace=false,
  breaklines=true,
  captionpos=b,
  keepspaces=true,
  numbers=left,
  numbersep=5pt,
  showspaces=false,
  showstringspaces=false,
  showtabs=false,
  tabsize=2,
  frame=lines, % Añade un marco alrededor del código
  framerule=0.4pt, % Grosor del marco
  backgroundcolor=\color{backcolour} % Color de fondo suave
}
\lstset{style=PythonStyle}
%	TITLE PAGE
\title{Interfaces Gráficas (GUI) con Tkinter}
\subtitle{Clase 4: Múltiples Ventanas y Eventos Avanzados}

\author{Prof. D.Sc. BARSEKH-ONJI Aboud}
\institute
{
    Facultad de Ingeniería \\
    Universidad Anáhuac México
}
\date{\today}

%----------------------------------------------------------------------------------------
%   CONTENIDO DE LA PRESENTACIÓN
%----------------------------------------------------------------------------------------

% --- Agenda automática al inicio de cada sección ---
\AtBeginSection[]
{
  \begin{frame}{Agenda}
    \tableofcontents[currentsection]
  \end{frame}
}

\begin{document}

\begin{frame}
    \titlepage
\end{frame}

%------------------------------------------------
\section{Manejo de Múltiples Formularios (Ventanas)}
%------------------------------------------------

\begin{frame}[fragile]
    \frametitle{Aplicaciones Reales}
    
    \begin{block}{El Problema}
    Casi ninguna aplicación real es una sola ventana. Pensemos en cualquier software profesional:
    \begin{itemize}
        \item Una ventana principal (el editor de texto, el navegador, etc.).
        \item Ventanas secundarias para ''Configuración'' o ''Preferencias''.
        \item Diálogos emergentes para ''Guardar Archivo'' o ''Acerca de...''.
        \item Una ventana de ''Login'' que da paso a la aplicación principal.
    \end{itemize}
    \end{block}
\end{frame}
\begin{frame}[fragile]
    \frametitle{Aplicaciones Reales}
    \begin{alertblock}{La Solución: \texttt{tk.Toplevel()}}
    Para crear una nueva ventana que ''flota'' encima de nuestra ventana principal, usamos el widget \texttt{tk.Toplevel()}.
    \end{alertblock}

\end{frame}

%------------------------------------------------

\begin{frame}[fragile]
    \frametitle{Crear y Manejar Múltiples Ventanas}
    
    \begin{block}{Crear una Ventana \texttt{Toplevel}}
    Se crea de forma muy similar a otros widgets. Es una ventana ''hija'' de la ventana raíz. Si cierras la raíz, esta también se cierra.
    \begin{lstlisting}[language=Python]
# Dentro de un metodo de nuestra clase App
def abrir_ventana_secundaria(self):
    # 'self' (el Frame) es el padre de la nueva ventana
    ventana_nueva = tk.Toplevel(self)
    
    ventana_nueva.title('Ventana Secundaria')
    tk.Label(ventana_nueva, text='¡Soy una nueva ventana!').pack()
    \end{lstlisting}
    \end{block}
\end{frame}
\begin{frame}[fragile]
    \frametitle{Crear y Manejar Múltiples Ventanas}
        
    \begin{alertblock}{Manejo de Formularios}
    Podemos controlar la visibilidad de las ventanas:
    \begin{itemize}
        \item \texttt{ventana.destroy()}: \textbf{Cierra y destruye} la ventana. Se usa para cerrar un pop-up o la ventana de login.
        \item \texttt{ventana.withdraw()}: \textbf{Oculta} la ventana sin destruirla.
        \item \texttt{ventana.deiconify()}: \textbf{Muestra} una ventana que estaba oculta.
    \end{itemize}
    \end{alertblock}

\end{frame}

%------------------------------------------------
\section{Eventos Avanzados}
%------------------------------------------------

\begin{frame}[fragile]
    \frametitle{Más Allá del \texttt{command}}
    
    \begin{block}{El Problema}
    La propiedad \texttt{command} es genial para botones, pero ¿qué pasa si queremos reaccionar a otras acciones?
    \begin{itemize}
        \item ¿Hacer clic derecho en una etiqueta?
        \item ¿Presionar la tecla ''Enter'' en un campo de texto?
        \item ¿Detectar cuándo el mouse pasa por encima de una imagen?
    \end{itemize}
    \end{block}
    
    \begin{alertblock}{La Solución: El Método \texttt{.bind()}}
    Todos los widgets en Tkinter tienen un método universal llamado \texttt{.bind()} que nos permite ''atar'' (bind) una función de Python a un evento específico en ese widget.
    \end{alertblock}

\end{frame}

%------------------------------------------------

\begin{frame}[fragile]
    \frametitle{Sintaxis de \texttt{.bind()}}
    
    \begin{block}{Paso 1: La Función (Callback)}
    La función que es llamada por \texttt{.bind()} \textbf{debe} aceptar un argumento, que por convención se llama \texttt{event}. Python pasa automáticamente información sobre el evento (como la posición X/Y del mouse).
    \begin{lstlisting}[language=Python]
# La funcion DEBE aceptar el argumento 'event'
def mi_callback(event):
    print(''¡Evento detectado!'')
    # event tiene info util: event.x, event.y
    print(f''Clic en la posicion: {event.x}, {event.y}'')
    \end{lstlisting}
    \end{block}
\end{frame}


\begin{frame}[fragile]
    \frametitle{Sintaxis de \texttt{.bind()}}
    \begin{alertblock}{Paso 2: El ''Atado''}
    Usamos \texttt{.bind()} con dos argumentos: el string del evento y el nombre de la función (sin paréntesis).
    \begin{lstlisting}[language=Python]
# Sintaxis:
# mi_widget.bind(''<DescriptorDelEvento>'', funcion_callback)

# Ejemplo:
etiqueta = tk.Label(self, text=''Haz clic derecho en mi'')
etiqueta.pack()

# Atamos el evento <Button-3> (clic derecho) a la funcion
etiqueta.bind(''<Button-3>'', mi_callback)
    \end{lstlisting}
    \end{alertblock}

\end{frame}

%------------------------------------------------

\begin{frame}[fragile]
    \frametitle{Principales Eventos de Mouse}
    
    \begin{block}{Tipos de Eventos de Clic}
    Podemos capturar diferentes tipos de clics:
    \begin{itemize}
        \item \texttt{<Button-1>}: Clic izquierdo del mouse. (También se le llama \textbf{mouse down}).
        \item \texttt{<Button-3>}: Clic derecho del mouse.
        \item \texttt{<Double-Button-1>}: Doble clic izquierdo (\textbf{dbclick}).
    \end{itemize}
    \end{block}
\end{frame}

\begin{frame}[fragile]
    \frametitle{Principales Eventos de Mouse}
    \begin{alertblock}{Drag y Drop (Arrastrar y Soltar)}
    El ''arrastrar y soltar'' es una combinación de dos eventos:
    \begin{itemize}
        \item \texttt{<B1-Motion>}: \textbf{(Drag)} Se dispara continuamente \textit{mientras} el botón 1 está presionado y el mouse se mueve.
        \item \texttt{<ButtonRelease-1>}: \textbf{(Drop)} Se dispara cuando el usuario \textit{suelta} el botón 1.
    \end{itemize}
    \end{alertblock}
\end{frame}

\begin{frame}[fragile]
    \frametitle{Principales Eventos de Mouse}   
    \begin{exampleblock}{Ejemplo: Rastrear el Mouse}
    \begin{lstlisting}[language=Python]
def al_arrastrar(event):
    print(f''Mouse arrastrado a: {event.x}, {event.y}'')
    
# Creamos un area de texto
area_texto = tk.Text(self, height=5, width=30)
area_texto.pack()

# Atamos el evento de arrastrar
area_texto.bind(''<B1-Motion>'', al_arrastrar)
    \end{lstlisting}
    \end{exampleblock}

\end{frame}

%------------------------------------------------
\section{Tarea 4}
%------------------------------------------------

\begin{frame}[fragile]
    \frametitle{Tarea 4: Aplicación de Login}

    \begin{block}{Objetivo}
    Combinar el conocimiento de \textbf{Clases} (Clase 3) con el manejo de \textbf{múltiples ventanas} (Clase 4) para crear una aplicación de login funcional.
    \end{block}
\end{frame}
\begin{frame}[fragile]
    \frametitle{Tarea 4: Aplicación de Login}
    \begin{alertblock}{Instrucciones}
    \begin{itemize}
        \item \textbf{Estructura:} Crea tu aplicación de Login usando la estructura de \textbf{Clase} que aprendimos.
        \item \textbf{Ventana de Login:} La ventana principal (`tk.Tk`) será la ventana de Login. Debe contener:
            \begin{itemize}
                \item Un \texttt{Label} y un \texttt{Entry} para la contraseña.
                \item Un \texttt{Button} para ''Ingresar''.
                \item Un \texttt{Label} para mensajes de error (inicialmente vacío).
            \end{itemize}
    \end{itemize}
    \end{alertblock}
\end{frame}

\begin{frame}[fragile]
    \frametitle{Tarea 4: Aplicación de Login}
    \begin{alertblock}{Instrucciones}
    \begin{itemize}
        \item \textbf{Lógica de Login:} Escribe un \textbf{método} (función) que se ejecute con el \texttt{command} del botón:
            \begin{itemize}
                \item Si la contraseña es \texttt{''1234''}, el programa debe \textbf{cerrar} la ventana de Login (`self.master.destroy()`) y \textbf{abrir} una nueva ventana principal (mira la pista).
                \item Si la contraseña es incorrecta, debe actualizar el \texttt{Label} de error con el texto ''Contraseña incorrecta''.
            \end{itemize}
    \end{itemize}
    \end{alertblock}
\end{frame}
\begin{frame}[fragile]
    \frametitle{Tarea 4: Aplicación de Login}   
    \begin{exampleblock}{Pista Clave: Dos Clases}
    La mejor forma de hacerlo es con \textbf{dos clases}: una para la App de Login y otra para la App Principal.
    \begin{lstlisting}[language=Python, literate=*{_}{{{\_}}}1]
class AppPrincipal(tk.Frame):
    def __init__(self, master):
        super().__init__(master)
        # ... widgets de la app principal ...  
class AppLogin(tk.Frame):
    def __init__(self, master):
        super().__init__(master)
        # ... widgets de login (entry, boton) ...
        self.boton_login.config(command=self.verificar_login)
    def verificar_login(self):
        if self.entrada_pass.get() == ''1234'':
            self.master.destroy() # Cierra la ventana de login

    \end{lstlisting}
    \end{exampleblock}

\end{frame}

\begin{frame}[fragile]
    \frametitle{Tarea 4: Aplicación de Login}   
    \begin{exampleblock}{Pista Clave: Dos Clases}
    La mejor forma de hacerlo es con \textbf{dos clases}: una para la App de Login y otra para la App Principal.
    \begin{lstlisting}[language=Python, literate=*{_}{{{\_}}}1]

            # Crea la nueva ventana principal
            nueva_ventana = tk.Tk()
            app_main = AppPrincipal(master=nueva_ventana)
            app_main.mainloop()
        else:
            # ... mostrar error ... 
# --- Codigo Principal ---
if __name__ == ''__main__'':
    ventana_login = tk.Tk()
    app = AppLogin(master=ventana_login)
    app.mainloop()
    \end{lstlisting}
    \end{exampleblock}

\end{frame}
\end{document}