%----------------------------------------------------------------------------------------
%	PAQUETES Y TEMAS
%----------------------------------------------------------------------------------------
\documentclass[aspectratio=169,xcolor=dvipsnames]{beamer}
\usetheme{SimpleDarkBlue}

\usepackage[spanish]{babel}
\usepackage{hyperref}
\usepackage{graphicx} % Allows including images
\usepackage{booktabs} % Allows the use of \toprule, \midrule and \bottomrule in tables
\usepackage{amsmath}
\usepackage{lettrine}
\setbeamertemplate{caption}[numbered]
\usepackage[dvipsnames,svgnames,x11names]{xcolor}% Para definir y usar colores (ej. en hipervínculos)
\usepackage{xurl}
\usepackage{hyperref}       % Para crear hipervínculos internos y externos
\usepackage{algorithm}
\usepackage{algorithmicx}
\usepackage{algpseudocode}
\hypersetup{
    colorlinks=true,        % Colorear los enlaces en lugar de usar recuadros
    linkcolor=blue,     % Color para enlaces internos (índice, referencias cruzadas)
    filecolor=blue, % Color para enlaces a archivos locales
    urlcolor=blue,      % Color para URLs
    citecolor=blue,     % Color para citas bibliográficas
}
%----------------------------------------------------------------------------------------

\usepackage{listings}
\usepackage{xcolor} % Para colores en listings
 \definecolor{codegreen}{rgb}{0,0.6,0}
 \definecolor{codegray}{rgb}{0.5,0.5,0.5}
 \definecolor{codepurple}{rgb}{0.58,0,0.82}
 \definecolor{backcolour}{rgb}{0.97,0.97,0.99}

\lstdefinestyle{PythonStyle}{
  language=Python,
  basicstyle=\ttfamily\footnotesize,
  keywordstyle=\color{blue}\bfseries,
  commentstyle=\color{codegreen},
  stringstyle=\color{violet},
  numberstyle=\tiny\color{gray},
  breakatwhitespace=false,
  breaklines=true,
  captionpos=b,
  keepspaces=true,
  numbers=left,
  numbersep=5pt,
  showspaces=false,
  showstringspaces=false,
  showtabs=false,
  tabsize=2,
  frame=lines, % Añade un marco alrededor del código
  framerule=0.4pt, % Grosor del marco
  backgroundcolor=\color{backcolour} % Color de fondo suave
}
\lstset{style=PythonStyle}
%	TITLE PAGE
%----------------------------------------------------------------------------------------
%	PÁGINA DE TÍTULO
%----------------------------------------------------------------------------------------

\title{Manejo de Cadenas (Strings) y Listas en Python}
\subtitle{Materia: Algoritmos y Programación}

\author{Prof. D.Sc. BARSEKH-ONJI Aboud}
\institute
{
    Facultad de Ingeniería \\
    Universidad Anáhuac México
}
\date{\today}

%----------------------------------------------------------------------------------------
%	CONTENIDO DE LA PRESENTACIÓN
%----------------------------------------------------------------------------------------

% --- Agenda automática al inicio de cada sección ---
\AtBeginSection[]
{
  \begin{frame}{Agenda}
    \tableofcontents[currentsection]
  \end{frame}
}

\begin{document}

\begin{frame}
    \titlepage
\end{frame}

%------------------------------------------------
\section{Accediendo y Representando Caracteres}
%------------------------------------------------

\begin{frame}[fragile]{Extrayendo Caracteres: Los Índices}
    \begin{block}{Los Strings son Secuencias Ordenadas}
    Cada carácter en un string tiene una posición o \textbf{índice}. En Python, la numeración de los índices siempre comienza en \textbf{cero (0)}.
    \begin{center}
    \ttfamily
    \begin{tabular}{c c c c c c}
    P & Y & T & H & O & N \\
    \hline
    0 & 1 & 2 & 3 & 4 & 5
    \end{tabular}
    \end{center}
    \end{block}
\end{frame}
\begin{frame}[fragile]{Extrayendo Caracteres: Los Índices}

    \begin{columns}[t]
        \column{.48\textwidth}
            \begin{alertblock}{Índices Positivos (de Izq. a Der.)}
                 Se usan para acceder a los caracteres desde el principio del string.
                 \begin{lstlisting}
palabra = 'PYTHON'

# Acceder al primer caracter
primer_letra = palabra[0] # 'P'

# Acceder al tercer caracter
tercer_letra = palabra[2] # 'T'
                 \end{lstlisting}
            \end{alertblock}

        \column{.48\textwidth}
            \begin{block}{Índices Negativos (de Der. a Izq.)}
                Permiten acceder a los caracteres desde el final del string. \texttt{-1} es el último carácter.
                 \begin{lstlisting}
palabra = 'PYTHON'

# Acceder al ultimo caracter
ultima_letra = palabra[-1] # 'N'

# Acceder al penultimo caracter
penultima_letra = palabra[-2] # 'O'
                 \end{lstlisting}
            \end{block}
    \end{columns}
\end{frame}

%------------------------------------------------
\section{Introducción a las Listas}
%------------------------------------------------

\begin{frame}[fragile]{Introducción a las Listas: Contenedores Ordenados}
    \begin{block}{¿Qué es una Lista?}
    Una \textbf{lista} es una colección de elementos que está \textbf{ordenada} y es \textbf{mutable} (es decir, se puede cambiar después de crearla). Se definen con corchetes \texttt{[]} y los elementos se separan por comas.
    \begin{itemize}
        \item Pueden contener diferentes tipos de datos a la vez.
    \end{itemize}
    \end{block}
\end{frame}   
\begin{frame}[fragile]{Introducción a las Listas: Contenedores Ordenados}

    \begin{alertblock}{Accediendo a Elementos con Índices}
    Las listas usan el \textbf{mismo sistema de índices que los strings} para acceder a sus elementos:
    \begin{itemize}
        \item El primer elemento está en el índice \texttt{0}.
        \item El último elemento está en el índice \texttt{-1}.
    \end{itemize}
    \begin{lstlisting}
# Una lista con 4 elementos de diferentes tipos
calificaciones = [10, 9.5, 'Aprobado', True]
# Acceder al primer elemento
primer_elemento = calificaciones[0] # 10
# Acceder al tercer elemento
tercer_elemento = calificaciones[2] # 'Aprobado'
# Acceder al ultimo elemento
ultimo_elemento = calificaciones[-1] # True
    \end{lstlisting}
    \end{alertblock}
\end{frame}

%------------------------------------------------


\section{Secuencias de Escape}
\begin{frame}[fragile]{Caracteres Especiales: Secuencias de Escape}
    \begin{block}{¿Cómo Escribir Caracteres ''Invisibles''?}
    Para representar caracteres especiales como saltos de línea, tabulaciones o incluso comillas dentro de un string, usamos \textbf{secuencias de escape}, que comienzan con una barra invertida (\textbackslash).
    \end{block}
    \centering
            \begin{alertblock}{Secuencias Comunes}
                \begin{tabular}{ll}
                \toprule
                \textbf{Secuencia} & \textbf{Significado} \\
                \midrule
                \texttt{\textbackslash n} & Salto de línea \\
                \texttt{\textbackslash t} & Tabulación \\
                \texttt{\textbackslash\textbackslash} & Barra invertida \\
                \texttt{\textbackslash'} & Comilla simple \\
                \texttt{\textbackslash''} & Comilla doble \\
                \bottomrule
                \end{tabular}
            \end{alertblock}
\end{frame}


\begin{frame}[fragile]{Caracteres Especiales: Secuencias de Escape}

            \begin{examples}
                 \begin{lstlisting}
# Salto de linea con \n
mensaje = 'Linea 1 \n Linea 2'
print(mensaje)
# Linea 1
# Linea 2

# Tabulacion con \t
lista = 'Productos: \n \t -Manzanas \n \t -Leche'
print(lista)

# Comillas dentro de un string
cita = 'El me dijo: \'Python es genial\'.'
print(cita)
# El me dijo: 'Python es genial'.
                 \end{lstlisting}
            \end{examples}
\end{frame}

%------------------------------------------------
\section{Formateo de Strings e Introspección}
%------------------------------------------------

\begin{frame}[fragile]{Formateando Strings con f-strings}
    \begin{columns}[t]
        \column{.48\textwidth}
            \begin{block}{El Método 'Antiguo'}
                Antes, para combinar texto y variables, se usaba la concatenación y la función 'str()', lo que podía volverse verboso y difícil de leer.
                \begin{lstlisting}
nombre = 'Maria'
edad = 30
promedio = 9.5

# Dificil de leer y escribir
print('Estudiante: ' + nombre + ', Edad: ' + str(edad) + ', Promedio: ' + str(promedio))
                \end{lstlisting}
            \end{block}

        \column{.48\textwidth}
            \begin{alertblock}{La Solución Moderna: f-strings}
                Un \textbf{f-string} (string formateado) simplifica enormemente esto. Solo debes poner una \texttt{f} antes de las comillas e insertar las variables directamente dentro de llaves \texttt{\{\}}.
                 \begin{lstlisting}
nombre = 'Maria'
edad = 30
promedio = 9.5

# Mucho mas limpio y legible
print(f'Estudiante: {nombre}, Edad: {edad}, Promedio: {promedio}')
                 \end{lstlisting}
            \end{alertblock}
    \end{columns}
\end{frame}

%------------------------------------------------

\begin{frame}[fragile]{Formateo Avanzado de Números con f-strings}
    \begin{block}{Controlando la Apariencia de los Números}
    Los f-strings no solo insertan valores, sino que también nos permiten controlar con precisión cómo se muestran los números, lo cual es fundamental para crear salidas de datos limpias y profesionales.
    \end{block}
\end{frame}  
\begin{frame}[fragile]{Formateo Avanzado de Números con f-strings}

    \begin{columns}[t]
        \column{.48\textwidth}
            \begin{alertblock}{Precisión Fija para Flotantes}
                Podemos especificar el número de decimales a mostrar usando \texttt{:.Nf}, donde \texttt{N} es el número de decimales.
                 \begin{lstlisting}
pi = 3.14159265
total = 1234.5
# Mostrar pi con 2 decimales
print(f'El valor de pi es: {pi:.2f}')
# Salida: El valor de pi es: 3.14
# Formato de moneda
print(f'Total a pagar: ${total:.2f}')
# Salida: Total a pagar: $1234.50
                 \end{lstlisting}
            \end{alertblock}

        \column{.48\textwidth}
            \begin{block}{Ceros a la Izquierda para Enteros}
                Podemos rellenar un número con ceros a la izquierda para que tenga un ancho fijo usando \texttt{:0Nd}, donde \texttt{N} es el ancho total.
                 \begin{lstlisting}
numero_factura = 45
dia = 7
mes = 9
# Rellenar a 5 digitos
print(f'Factura No: {numero_factura:05d}')
# Salida: Factura No: 00045
# Formato de fecha
print(f'Fecha: {dia:02d}/{mes:02d}/2023')
# Salida: Fecha: 07/09/2023
                 \end{lstlisting}
            \end{block}
    \end{columns}
\end{frame}

%------------------------------------------------

\begin{frame}[fragile]{Conociendo Nuestras Variables: 'type()' y 'id()'}
    \begin{block}{Funciones de Introspección}
    Python nos ofrece funciones para 'preguntar' a nuestras variables qué son y dónde están en la memoria.
    \end{block}
\end{frame}
\begin{frame}[fragile]{Conociendo Nuestras Variables: 'type()' y 'id()'}

            \begin{alertblock}{'type()' - ¿Qué es esto?}
                Devuelve el \textbf{tipo de dato} de una variable. Es muy útil para depurar y entender cómo se están manejando los datos.
                 \begin{lstlisting}
numero = 100
texto = 'Hola'
es_valido = True
print(type(numero))
# Salida: <class 'int'>
print(type(texto))
# Salida: <class 'str'>
print(type(es_valido))
# Salida: <class 'bool'>
                 \end{lstlisting}
            \end{alertblock}
\end{frame}

    \begin{frame}[fragile]{Conociendo Nuestras Variables: 'type()' y 'id()'}
        \begin{block}{'id()' - ¿Dónde está esto?}
                Devuelve el \textbf{identificador de memoria} único de un objeto. Es un número que representa la dirección donde está guardado el dato en la memoria RAM.
                 \begin{lstlisting}
x = 10
y = x # 'y' apunta al mismo objeto que 'x'
z = 10 # Python es eficiente, reutiliza el objeto
# Muestran el mismo ID, porque apuntan
# al mismo objeto '10' en memoria
print(id(x)) 
print(id(y))
print(id(z))
                 \end{lstlisting}
            \end{block}
\end{frame}


\begin{frame}[fragile]
    \frametitle{Tarea 1: Decodificador de Mensaje Secreto}
    
    \begin{block}{Objetivo}
    Usar el conocimiento de \textbf{índices} para extraer caracteres de un string y revelar un mensaje oculto. Deberás usar un bucle \texttt{for} para resolverlo.
    \end{block}
\end{frame}


\begin{frame}[fragile]
    \frametitle{Tarea 1: Decodificador de Mensaje Secreto}    
    \begin{alertblock}{El Código a Completar}
    \begin{lstlisting}
# El mensaje cifrado contiene la informacion oculta
mensaje_cifrado = 'aPzYleTtnHhOoNnlax'

# Estos son los indices de los caracteres correctos
indices_secretos = [1, 5, 7, 8, 10, 12]

# Variable para guardar el resultado
mensaje_decodificado = ''
# -----------------------
# Debes crear un bucle 'for' que recorra los 'indices_secretos'.
# En cada paso, extrae el caracter de 'mensaje_cifrado'
# usando el indice actual y anadelo a 'mensaje_decodificado'.
# Pista: usa el operador '+=' para anadir caracteres.
# Al final, imprime el mensaje decodificado
print('El mensaje secreto es:', mensaje_decodificado)
# La salida deberia ser: Python
    \end{lstlisting}
    \end{alertblock}
\end{frame}

\begin{frame}[fragile]
    \frametitle{Tarea 1: Decodificador de Mensaje Secreto}
    
    \begin{block}{Revisa el código completo}
    \url{https://github.com/AboudOnji/ExamplesAyP/blob/main/Example15.py}
    \end{block}
\end{frame}
%------------------------------------------------
\begin{frame}[fragile]
    \frametitle{Tarea 2: Generador de Recibos Formateado}
    
    \begin{block}{Objetivo}
    Utilizar \textbf{listas} para almacenar datos y \textbf{f-strings} para generar un recibo de compra con un formato profesional y alineado.
    \end{block}
\end{frame}  
\begin{frame}[fragile]
    \frametitle{Tarea 2: Generador de Recibos Formateado}
    \begin{alertblock}{El Código a Completar}
    \begin{lstlisting}
productos = ['Leche Entera', 'Pan de Caja', 'Huevo (12pza)']
precios = [25.50, 42.00, 38.95]
total = 0.0
print('--- RECIBO DE COMPRA ---')
print('No. | Producto      | Precio')
print('--------------------------')
# Crea un bucle 'for' que recorra las listas usando un rango
# de indices (pista: for i in range(len(productos)):).
# Dentro del bucle, imprime una linea formateada para cada producto:
# 1. El numero de item (i+1), con un cero a la izquierda (ej: 01).
# 2. El nombre del producto (productos[i]).
# 3. El precio (precios[i]), con exactamente 2 decimales.
# 4. Acumula el precio en la variable 'total'.
# Al final, imprime una linea de separacion y el total
print('--------------------------')

    \end{lstlisting}
    \end{alertblock}
\end{frame}
\begin{frame}[fragile]
    \frametitle{Tarea 2: Generador de Recibos Formateado}
    \begin{alertblock}{El Código a Completar}
    \begin{lstlisting}
# print(f'TOTAL:          ${total:.2f}')
# La salida deberia verse asi:
# --- RECIBO DE COMPRA ---
# No. | Producto      | Precio
# --------------------------
# 01  | Leche Entera  | 25.50
# 02  | Pan de Caja   | 42.00
# 03  | Huevo (12pza) | 38.95
# --------------------------
# TOTAL:          $106.45
    \end{lstlisting}
    \end{alertblock}
\end{frame}

\begin{frame}[fragile]
    \frametitle{Tarea 2: Generador de Recibos Formateado}
    
    \begin{block}{Revisa el código completo}
    \url{https://github.com/AboudOnji/ExamplesAyP/blob/main/Example16.py}
    \end{block}
\end{frame}
\begin{frame}[fragile]{Traduciendo Caracteres: 'ord()' y 'chr()'}
    \begin{block}{El Código Detrás de Cada Carácter}
    Cada carácter que ves en la pantalla (como 'A', 'b', o '\$') está almacenado en la memoria como un número. El estándar que define esta correspondencia se llama \textbf{Unicode} (y su subconjunto más conocido es ASCII).
    \end{block}
\end{frame}
\begin{frame}[fragile]{Traduciendo Caracteres: 'ord()' y 'chr()'}
    \begin{columns}[t]
        \column{.48\textwidth}
            \begin{alertblock}{De Carácter a Número: 'ord()'}
                La función \texttt{ord()} (de 'ordinal') toma un carácter y devuelve su código numérico Unicode.
                 \begin{lstlisting}
# Obtener el codigo de 'A'
codigo_A = ord('A')
print(codigo_A) # Imprime: 65

# Obtener el codigo de 'b'
codigo_b = ord('b')
print(codigo_b) # Imprime: 98
                 \end{lstlisting}
            \end{alertblock}

        \column{.48\textwidth}
            \begin{block}{De Número a Carácter: 'chr()'}
                La función \texttt{chr()} (de 'character') hace lo opuesto: toma un código numérico y devuelve el carácter correspondiente.
                 \begin{lstlisting}
# Obtener el caracter para 65
caracter_1 = chr(65)
print(caracter_1) # Imprime: 'A'

# Obtener el caracter para 98
caracter_2 = chr(98)
print(caracter_2) # Imprime: 'b'
                 \end{lstlisting}
            \end{block}
    \end{columns}
\end{frame}

%------------------------------------------------

\begin{frame}[fragile]
    \frametitle{Tarea 3: Cifrador César Simple}
    
    \begin{block}{Objetivo}
    Escribir un programa que cifre un mensaje moviendo cada letra un número determinado de posiciones en el alfabeto. Para esto, deberás combinar el uso de bucles \texttt{for} con las funciones \texttt{ord()} y \texttt{chr()}.
    \end{block}
\end{frame}
\begin{frame}[fragile]
    \frametitle{Tarea 3: Cifrador César Simple}
    \begin{alertblock}{El Código a Completar}
    \begin{lstlisting}
mensaje_original = input('Introduce el mensaje a cifrar: ')
desplazamiento = int(input('Introduce el desplazamiento (ej: 3): '))

mensaje_cifrado = ''

# --- TU CODIGO AQUI ---
# Crea un bucle 'for' que recorra cada 'letra' en 'mensaje_original'.
# Dentro del bucle:
# 1. Obtiene el codigo numerico de la 'letra' con ord().
# 2. Sumale el 'desplazamiento' para obtener el nuevo codigo.
# 3. Convierte el nuevo codigo de vuelta a un caracter con chr().
# 4. Anade el nuevo caracter a 'mensaje_cifrado'.
print('Mensaje cifrado:', mensaje_cifrado)
# Ejemplo de salida:
# Si mensaje='HOLA' y desplazamiento=3, la salida deberia ser 'KROD'
    \end{lstlisting}
    \end{alertblock}
\end{frame}
\begin{frame}[fragile]
    \frametitle{Tarea 3: Cifrador César Simple}
    
    \begin{block}{Revisa el código completo}
    \url{https://github.com/AboudOnji/ExamplesAyP/blob/main/Example17.py}
    \end{block}
\end{frame}

%------------------------------------------------
%------------------------------------------------
\end{document}