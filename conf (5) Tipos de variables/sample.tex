%----------------------------------------------------------------------------------------
%	PAQUETES Y TEMAS
%----------------------------------------------------------------------------------------
\documentclass[aspectratio=169,xcolor=dvipsnames]{beamer}
\usetheme{SimpleDarkBlue}

\usepackage[spanish]{babel}
\usepackage{hyperref}
\usepackage{graphicx} % Allows including images
\usepackage{booktabs} % Allows the use of \toprule, \midrule and \bottomrule in tables
\usepackage{amsmath}
\usepackage{lettrine}
\setbeamertemplate{caption}[numbered]
\usepackage[dvipsnames,svgnames,x11names]{xcolor}% Para definir y usar colores (ej. en hipervínculos)
\usepackage{xurl}
\usepackage{hyperref}       % Para crear hipervínculos internos y externos
\usepackage{algorithm}
\usepackage{algorithmicx}
\usepackage{algpseudocode}
\hypersetup{
    colorlinks=true,        % Colorear los enlaces en lugar de usar recuadros
    linkcolor=blue,     % Color para enlaces internos (índice, referencias cruzadas)
    filecolor=blue, % Color para enlaces a archivos locales
    urlcolor=blue,      % Color para URLs
    citecolor=blue,     % Color para citas bibliográficas
}
%----------------------------------------------------------------------------------------

\usepackage{listings}
\usepackage{xcolor} % Para colores en listings
 \definecolor{codegreen}{rgb}{0,0.6,0}
 \definecolor{codegray}{rgb}{0.5,0.5,0.5}
 \definecolor{codepurple}{rgb}{0.58,0,0.82}
 \definecolor{backcolour}{rgb}{0.97,0.97,0.99}

\lstdefinestyle{PythonStyle}{
  language=Python,
  basicstyle=\ttfamily\footnotesize,
  keywordstyle=\color{blue}\bfseries,
  commentstyle=\color{codegreen},
  stringstyle=\color{violet},
  numberstyle=\tiny\color{gray},
  breakatwhitespace=false,
  breaklines=true,
  captionpos=b,
  keepspaces=true,
  numbers=left,
  numbersep=5pt,
  showspaces=false,
  showstringspaces=false,
  showtabs=false,
  tabsize=2,
  frame=lines, % Añade un marco alrededor del código
  framerule=0.4pt, % Grosor del marco
  backgroundcolor=\color{backcolour} % Color de fondo suave
}
\lstset{style=PythonStyle}
%	TITLE PAGE
%----------------------------------------------------------------------------------------
%	PÁGINA DE TÍTULO
%----------------------------------------------------------------------------------------

\title{Tipos de variables y operaciones en Python}
\subtitle{Materia: Algoritmos y Programación}

\author{Prof. D.Sc. BARSEKH-ONJI Aboud}
\institute
{
    Facultad de Ingeniería \\
    Universidad Anáhuac México
}
\date{\today}

%----------------------------------------------------------------------------------------
%	CONTENIDO DE LA PRESENTACIÓN
%----------------------------------------------------------------------------------------

% --- Agenda automática al inicio de cada sección ---
\AtBeginSection[]
{
  \begin{frame}{Agenda}
    \tableofcontents[currentsection]
  \end{frame}
}

\begin{document}

\begin{frame}
    % Print the title page as the first slide
    \titlepage
\end{frame}

%------------------------------------------------
\section{Tipos de Variables en Python}
%------------------------------------------------

\begin{frame}[fragile]{¿Qué es una Variable?}
    \begin{block}{Analogía: Una Caja con Etiqueta}
    Una variable es como una \textbf{caja en la memoria} de la computadora donde podemos guardar información. Le ponemos una \textbf{etiqueta (el nombre de la variable)} para poder encontrar y usar esa información más tarde.
    \end{block}
    
    \begin{alertblock}{Tipado Dinámico en Python}
    A diferencia de otros lenguajes, en Python no necesitas declarar el tipo de una variable. Python lo infiere automáticamente cuando le asignas un valor.
        \begin{lstlisting}
# Python sabe que 'edad' es un numero
edad = 25 

# Python sabe que 'nombre' es texto
nombre = 'Juan Perez' 
        \end{lstlisting}
    \end{alertblock}
\end{frame}

%------------------------------------------------

\begin{frame}[fragile]{Tipos de Datos Numéricos: 'int' y 'float'}
    \begin{columns}[t]
        \column{.48\textwidth}
            \begin{block}{Enteros ('int')}
                Representan números enteros, positivos o negativos, sin parte decimal.
                \begin{lstlisting}
# Ejemplos de enteros
edad = 30
temperatura = -5
numero_de_alumnos = 150
                \end{lstlisting}
            \end{block}

        \column{.48\textwidth}
            \begin{alertblock}{Flotantes ('float')}
                 Representan números de punto flotante, es decir, números con parte decimal.
                 \begin{lstlisting}
# Ejemplos de flotantes
precio = 199.99
pi = 3.14159
altura = 1.75
                 \end{lstlisting}
            \end{alertblock}
    \end{columns}
    
\end{frame}
\begin{frame}[fragile]{Tipos de Datos Numéricos: 'int' y 'float'}
\begin{block}{Operaciones Matemáticas}
    Python realiza las operaciones aritméticas de forma natural. ¡Ojo! La división '/' siempre produce un 'float'.
    \begin{lstlisting}
suma = 10 + 5       # Resultado: 15 (int)
resta = 20.5 - 10   # Resultado: 10.5 (float)
division = 10 / 2   # Resultado: 5.0 (float!)
    \end{lstlisting}
    \end{block}
\end{frame}

    
%------------------------------------------------

\begin{frame}[fragile]{Tipo de Dato de Texto: 'str'}
    \begin{block}{Cadenas de Caracteres ('str')}
    Una cadena de caracteres, o \textit{string}, se usa para almacenar texto. El texto debe ir entre comillas simples (' ') o dobles ('' '').
    \begin{lstlisting}
nombre = 'Ana'
mensaje = 'Hola Mundo'
frase = 'El lenguaje "Python" es muy versatil.'
    \end{lstlisting}
    \end{block}
    
    
\end{frame}
\begin{frame}[fragile]{Tipo de Dato de Texto: 'str'}

\begin{alertblock}{Operación Principal: Concatenación (+)}
    El operador '+' con cadenas de texto no suma, sino que las une (concatena).
    \begin{lstlisting}
nombre = 'Carlos'
apellido = 'Santana'
nombre_completo = nombre + ' ' + apellido

# Resultado: 'Carlos Santana'
print(nombre_completo) 
    \end{lstlisting}
    \end{alertblock}
\end{frame}    
%------------------------------------------------

\begin{frame}[fragile]{Tipo de Dato Lógico: 'bool'}
    \begin{block}{Booleanos ('bool')}
    El tipo booleano solo puede tener dos valores: \textbf{'True'} (Verdadero) o \textbf{'False'} (Falso).
    \begin{lstlisting}
es_estudiante = True
es_mayor_de_edad = False
    \end{lstlisting}
    \end{block}
    
    
\end{frame}

\begin{frame}[fragile]{Tipo de Dato Lógico: 'bool'}

\begin{alertblock}{Uso Principal}
    Los booleanos son el resultado de las comparaciones y son fundamentales para la toma de decisiones en el código (condicionales 'if', bucles 'while', etc.).
    \begin{lstlisting}
edad = 20
es_mayor = edad >= 18 # es_mayor sera True

temperatura = 15
hace_frio = temperatura < 10 # hace_frio sera False
    \end{lstlisting}
    \end{alertblock}
\end{frame}
%------------------------------------------------
\section{Conversión de Tipos (Casting) y Operaciones}
%------------------------------------------------

\begin{frame}[fragile]{El Problema: Operaciones entre Tipos Incompatibles}
    \begin{block}{Python es de Tipado Fuerte}
    Aunque Python es flexible, no permite realizar operaciones entre tipos que no son compatibles. Por ejemplo, no se puede 'sumar' un número a un texto directamente.
    \end{block}
    
    \begin{alertblock}{Ejemplo de Error Típico ('TypeError')}
    Este código producirá un error:
    \begin{lstlisting}
edad = 25
# TypeError: can only concatenate str (not 'int') to str
mensaje = 'Tu edad es: ' + edad 
    \end{lstlisting}
    La solución es convertir explícitamente uno de los tipos para que sean compatibles. A esto se le llama \textbf{Casting}.
    \end{alertblock}
\end{frame}

%------------------------------------------------

\begin{frame}[fragile]{Conversión de Tipos (Casting)}

            \begin{block}{'str()'}
            Convierte un valor a una cadena de texto.
            \begin{lstlisting}
edad = 25
# Correcto!
mensaje = 'Tu edad es: ' + str(edad)
# mensaje = 'Tu edad es: 25'

numero = 123
texto = str(numero)
# texto = '123'
            \end{lstlisting}
            \end{block}

           
\end{frame}

\begin{frame}[fragile]{Conversión de Tipos (Casting)}

 \begin{alertblock}{'int()'}
            Convierte un valor a un entero. \textbf{Trunca (corta) los decimales}, no redondea.
            \begin{lstlisting}
# De str a int
puntos_str = '100'
puntos_int = int(puntos_str)

# De float a int
precio = 199.99
precio_int = int(precio)
# precio_int = 199
            \end{lstlisting}
            \end{alertblock}
\end{frame}
\begin{frame}[fragile]{Conversión de Tipos (Casting)}

            \begin{block}{'float()'}
            Convierte un valor a un número de punto flotante.
            \begin{lstlisting}
# De str a float
pi_str = '3.14159'
pi_float = float(pi_str)

# De int a float
edad = 30
edad_float = float(edad)
# edad_float = 30.0
            \end{lstlisting}
            \end{block}
\end{frame}

%------------------------------------------------
\section{Operaciones Matemáticas en Python}
%------------------------------------------------

\begin{frame}[fragile]{Operadores Aritméticos Básicos}

            \begin{block}{Suma, Resta y Multiplicación}
                \begin{itemize}
                    \item '+' (Suma)
                    \item '-' (Resta)
                    \item '*' (Multiplicación)
                \end{itemize}
                \begin{lstlisting}
a = 15
b = 4

suma = a + b         # 19
resta = a - b        # 11
multiplicacion = a * b # 60
                \end{lstlisting}
            \end{block}

           
\end{frame}
\begin{frame}[fragile]{Operadores Aritméticos Básicos}

 \begin{alertblock}{División}
                \begin{itemize}
                    \item '/' (División)
                \end{itemize}
                \textbf{¡Importante!}: El operador de división '/' siempre devuelve un resultado de tipo 'float', incluso si la división es exacta.
                 \begin{lstlisting}
# La division siempre resulta en float
division_1 = 15 / 4  # 3.75
division_2 = 20 / 4  # 5.0
                 \end{lstlisting}
            \end{alertblock}
\end{frame}            
%------------------------------------------------

\begin{frame}[fragile]{Operadores Aritméticos Avanzados}
            \begin{block}{División Entera '//'}
                Realiza la división y descarta (trunca) la parte decimal, devolviendo solo la parte entera.
                \begin{lstlisting}
resultado = 15 // 4 # 3
                \end{lstlisting}
            \end{block}
\end{frame}
\begin{frame}[fragile]{Operadores Aritméticos Avanzados}

            \begin{block}{Módulo \%}
                Devuelve el \textbf{residuo} de una división. Es muy útil para saber si un número es par o impar.
                \begin{lstlisting}
# 15 / 4 es 3 con residuo 3
residuo = 15 % 4  # 3

# Si n % 2 == 0, n es par
es_par = 10 % 2   # 0
                \end{lstlisting}
            \end{block}
\end{frame}
\begin{frame}[fragile]{Operadores Aritméticos Avanzados}

            \begin{block}{Exponenciación '**'}
            Eleva un número a una potencia.
            \begin{lstlisting}
# 2 elevado a la 3
potencia = 2 ** 3 # 8

# Raiz cuadrada
raiz = 9 ** 0.5 # 3.0
            \end{lstlisting}
            \end{block}

\end{frame}

%------------------------------------------------

\begin{frame}[fragile]{Orden de Operaciones (Precedencia)}
    \begin{block}{PEMDAS}
    Python sigue el orden estándar de las operaciones matemáticas, a menudo recordado por el acrónimo PEMDAS.
    \end{block}
    
    \begin{alertblock}{Jerarquía de Operadores}
        \begin{enumerate}
            \item \textbf{P}aréntesis '()'
            \item \textbf{E}xponentes '**'
            \item \textbf{M}ultiplicación '*', \textbf{D}ivisión '/', Módulo '\%', División Entera '//' (de izq. a der.)
            \item \textbf{S}uma (Adición) '+', \textbf{R}esta '-' (de izq. a der.)
        \end{enumerate}
    \end{alertblock}
    
   
\end{frame}
\begin{frame}[fragile]{Orden de Operaciones (Precedencia)}

 \begin{examples}
    \begin{lstlisting}
# Como lo evalua Python:
# 1. Exponente: 3 ** 2 = 9
# 2. Multiplicacion: 2 * 9 = 18
# 3. Suma: 5 + 18 = 23
resultado = 5 + 2 * 3 ** 2 
print(resultado) # Imprime 23
    \end{lstlisting}
    \end{examples}
\end{frame}

\begin{frame}[fragile]
    \frametitle{Homework: Simulador de Cajero Automático (ATM)}
    
    \begin{block}{Objetivo}
    Escribir un programa en Python que simule las operaciones básicas de un cajero automático. Deberás aplicar todo lo aprendido: variables, input, casting, operaciones matemáticas y estructuras de control ('if'/'elif'/'else' y 'while').
    \end{block}
\end{frame}
\begin{frame}[fragile]
    \frametitle{Homework: Simulador de Cajero Automático (ATM)}
    
    \begin{alertblock}{Requisitos del Programa}
    \begin{enumerate}
        \item \textbf{Saldo Inicial:} El programa debe comenzar con un saldo inicial (por ejemplo, \texttt{saldo = 1000.0}).
        
        \item \textbf{Menú Interactivo:} El programa debe mostrar un menú con 4 opciones y repetirse hasta que el usuario elija "Salir".
        \begin{itemize}
            \item 1. Consultar Saldo
            \item 2. Depositar Dinero
            \item 3. Retirar Dinero
            \item 4. Salir
        \end{itemize}
        
        \item \textbf{Funcionalidad de Opciones:}
        \begin{itemize}
            \item \textbf{Opción 1:} Muestra el saldo actual formateado.
            \item \textbf{Opción 2:} Pide al usuario una cantidad, la suma al saldo y muestra el nuevo saldo.
            \item \textbf{Opción 3:} Pide al usuario una cantidad. \textbf{Si hay fondos suficientes}, la resta del saldo. Si no, muestra un mensaje de "Fondos insuficientes".
            \item \textbf{Opción 4:} Muestra un mensaje de despedida y termina el programa.
        \end{itemize}
    \end{enumerate}
    \end{alertblock}
\end{frame}
\begin{frame}[fragile]
    \frametitle{Homework: Simulador de Cajero Automático (ATM)}
    
    \begin{block}{Pistas y Consejos}
    \begin{itemize}
        \item Usa un bucle \texttt{while:} para que el menú se muestre continuamente.
        \item Recuerda convertir la entrada del usuario a número (\texttt{int()} o \texttt{float()}) antes de hacer cálculos.
        \item Usa una estructura \texttt{if/elif/else} para manejar las 4 opciones del menú.
        \item Para salir del bucle \texttt{while}, puedes usar la palabra clave \texttt{break}.
    \end{itemize}
    \end{block}
\end{frame}
\end{document}