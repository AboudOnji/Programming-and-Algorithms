\documentclass[aspectratio=169,xcolor=dvipsnames]{beamer}
\usetheme{Berlin} % Elegir un tema de beamer

\usepackage[spanish]{babel}
\usepackage{hyperref}
\usepackage{graphicx} % Allows including images
\usepackage{booktabs} % Allows the use of \toprule, \midrule and \bottomrule in tables
\usepackage{amsmath}
\usepackage{lettrine}
\setbeamertemplate{caption}[numbered]
\usepackage[dvipsnames,svgnames,x11names]{xcolor}% Para definir y usar colores (ej. en hipervínculos)
\usepackage{xurl}
\usepackage{algorithm}
\usepackage{algorithmicx}
\usepackage{algpseudocode}
\hypersetup{
    colorlinks=true,        % Colorear los enlaces en lugar de usar recuadros
    linkcolor=blue,     % Color para enlaces internos (índice, referencias cruzadas)
    filecolor=blue, % Color para enlaces a archivos locales
    urlcolor=blue,      % Color para URLs
    citecolor=blue,     % Color para citas bibliográficas
}
%----------------------------------------------------------------------------------------

\usepackage{listings}
\usepackage{xcolor} % Para colores en listings
 \definecolor{codegreen}{rgb}{0,0.6,0}
 \definecolor{codegray}{rgb}{0.5,0.5,0.5}
 \definecolor{codepurple}{rgb}{0.58,0,0.82}
 \definecolor{backcolour}{rgb}{0.97,0.97,0.99}

\lstdefinestyle{PythonStyle}{
  language=Python,
  basicstyle=\ttfamily\footnotesize,
  keywordstyle=\color{blue}\bfseries,
  commentstyle=\color{codegreen},
  stringstyle=\color{violet},
  numberstyle=\tiny\color{gray},
  breakatwhitespace=false,
  breaklines=true,
  captionpos=b,
  keepspaces=true,
  numbers=left,
  numbersep=5pt,
  showspaces=false,
  showstringspaces=false,
  showtabs=false,
  tabsize=2,
  frame=lines, % Añade un marco alrededor del código
  framerule=0.4pt, % Grosor del marco
  backgroundcolor=\color{backcolour} % Color de fondo suave
}
\lstset{style=PythonStyle}
%	TITLE PAGE


\title{Ordenando Arreglos: El M\'etodo \texttt{.sort()}}
\subtitle{Materia: Algoritmos y Programación}

\author{Prof. D.Sc. BARSEKH-ONJI Aboud}
\institute
{
    Facultad de Ingeniería \\
    Universidad Anáhuac México
}
\date{\today}

%----------------------------------------------------------------------------------------

% --- Agenda automática al inicio de cada sección ---
\AtBeginSection[]
{
  \begin{frame}{Agenda}
    \tableofcontents[currentsection]
  \end{frame}
}

\begin{document}
\begin{frame}
    \titlepage
\end{frame}

%------------------------------------------------
\section{Fundamentos de \texttt{.sort()}}
%------------------------------------------------

\begin{frame}[fragile]
    \frametitle{¿Qu\'e es Ordenar?}
    
    \begin{block}{Ordenar una Lista}
    'Ordenar' es el proceso algorítmico de reorganizar los elementos de una lista (o arreglo) para que sigan un criterio específico, comúnmente numérico (de menor a mayor) o alfabético.
    \end{block}
\end{frame}
\begin{frame}[fragile]
    \frametitle{¿Qu\'e es Ordenar?}
    \begin{alertblock}{Diferencia Clave: \texttt{.sort()} vs. \texttt{sorted()}}
    Python tiene dos formas de ordenar, y es crucial entender la diferencia:
    \begin{itemize}
        \item \textbf{\texttt{mi\_lista.sort()}} (M\'etodo):
        \begin{itemize}
            \item \textbf{Modifica la lista original} (la ordena 'in-place').
            \item No devuelve nada (retorna \texttt{None}).
        \end{itemize}
        \item \textbf{\texttt{nueva\_lista = sorted(mi\_lista)}} (Funci\'on):
        \begin{itemize}
            \item \textbf{Devuelve una nueva lista} ordenada.
            \item La lista original no sufre ningún cambio.
        \end{itemize}
    \end{itemize}
    \textbf{Hoy nos enfocaremos en el m\'etodo \texttt{.sort()} de las listas.}
    \end{alertblock}

\end{frame}

%------------------------------------------------

\begin{frame}[fragile]
    \frametitle{Sintaxis B\'asica: \texttt{mi\_lista.sort()}}
    
    \begin{block}{Comportamiento por Defecto}
    Por defecto, \texttt{.sort()} ordena la lista de forma \textbf{ascendente} (de menor a mayor, o alfabéticamente).
    \end{block}
\end{frame}

\begin{frame}[fragile]
    \frametitle{Sintaxis B\'asica: \texttt{mi\_lista.sort()}}
    \begin{columns}[t]
        \column{.48\textwidth}
            \begin{alertblock}{Ejemplo 1: Números}
                \begin{lstlisting}[language=Python]
numeros = [5, 1, 10, -2, 8]

# Ordenamos la lista
numeros.sort()

print(numeros)
# Imprime: [-2, 1, 5, 8, 10]
                \end{lstlisting}
            \end{alertblock}

        \column{.48\textwidth}
            \begin{exampleblock}{Ejemplo 2: Strings}
                 \begin{lstlisting}[language=Python]
palabras = ['Zebra', 'Manzana', 'Auto']

# Ordenamos alfabeticamente
palabras.sort()

print(palabras)
# Imprime: ['Auto', 'Manzana', 'Zebra']
                \end{lstlisting}
            \end{exampleblock}
    \end{columns}
\end{frame}

%------------------------------------------------

\begin{frame}[fragile]
    \frametitle{Par\'ametro 1: \texttt{reverse}}
    
    \begin{block}{Sintaxis}
    \texttt{mi\_lista.sort(reverse=False)}
    \end{block}
    
    \begin{alertblock}{Orden Descendente}
    El método \texttt{.sort()} acepta un parámetro opcional llamado \texttt{reverse}.
    \begin{itemize}
        \item Por defecto, \texttt{reverse} es \texttt{False} (orden ascendente).
        \item Si lo establecemos a \texttt{True}, la lista se ordenará de forma \textbf{descendente} (de mayor a menor).
    \end{itemize}
    \end{alertblock}
\end{frame}
\begin{frame}[fragile]
    \frametitle{Par\'ametro 1: \texttt{reverse}} 
    \begin{exampleblock}{Ejemplo de Uso}
    \begin{lstlisting}[language=Python]
numeros = [5, 1, 10, -2, 8]

# Ordenamos en reversa (descendente)
numeros.sort(reverse=True)

print(numeros)
# Imprime: [10, 8, 5, 1, -2]
    \end{lstlisting}
    \end{exampleblock}

\end{frame}


%------------------------------------------------
\section{Ordenamiento Avanzado con \texttt{key}}
%------------------------------------------------

\begin{frame}[fragile]
    \frametitle{Par\'ametro 2: \texttt{key}}
    
    \begin{block}{Sintaxis}
    \texttt{mi\_lista.sort(key=None, reverse=False)}
    \end{block}
    
    \begin{alertblock}{El Par\'ametro M\'as Poderoso: \texttt{key}}
    El parámetro \texttt{key} nos permite cambiar \textbf{c\'omo} Python compara los elementos.
    \begin{itemize}
        \item La \texttt{key} (llave) debe ser una \textbf{funci\'on}.
        \item Python aplicará esta función a \textbf{cada elemento} de la lista \textit{antes} de compararlos.
        \item El ordenamiento se basará en el \textbf{resultado} de esa función, no en el elemento original.
    \end{itemize}
    \end{alertblock}
\end{frame}
\begin{frame}[fragile]
    \frametitle{Par\'ametro 2: \texttt{key}}   
    \begin{exampleblock}{Analogía: Ordenar Libros}
    Imagina que tienes una pila de libros. ¿Cómo los ordenas?
    \begin{itemize}
        \item \textbf{Sin \texttt{key} (por defecto):} Los ordenas por título (alfabéticamente).
        \item \textbf{Con \texttt{key=obtener\_num\_paginas}:} Los ordenas por su número de páginas (de menor a mayor).
        \item \textbf{Con \texttt{key=obtener\_apellido\_autor}:} Los ordenas alfabéticamente por el apellido del autor.
    \end{itemize}
    La \texttt{key} es la 'llave' o 'criterio' que usamos para decidir el orden.
    \end{exampleblock}

\end{frame}

%------------------------------------------------

\begin{frame}[fragile]
    \frametitle{Ejemplo de \texttt{key}: Ordenar por Longitud}
    
    \begin{block}{Usando la Función \texttt{len()}}
    Queremos ordenar una lista de palabras, no alfabéticamente, sino por su \textbf{longitud} (de la más corta a la más larga).
    
    Podemos usar la función incorporada \texttt{len()} (que devuelve la longitud de un string) como nuestra \texttt{key}.
    \end{block}
\end{frame}

\begin{frame}[fragile]
    \frametitle{Ejemplo de \texttt{key}: Ordenar por Longitud}
    
    \begin{alertblock}{Código de Ejemplo}
    \begin{lstlisting}[language=Python]
palabras = ['Gato', 'Mariposa', 'Pez', 'Elefante']

# Le decimos a .sort() que use la funcion 'len'
# como el criterio de ordenamiento.
palabras.sort(key=len)

print(palabras)
# Imprime: ['Pez', 'Gato', 'Mariposa', 'Elefante']
# (3, 4, 8, 8) <-- OJO: 'Mariposa' y 'Elefante'
# tienen la misma longitud (8), por lo que
# mantienen su orden relativo original (orden estable).
    \end{lstlisting}
    \end{alertblock}

\end{frame}

%------------------------------------------------

\begin{frame}[fragile]
    \frametitle{Ejemplo de \texttt{key}: Ignorar Mayúsculas}
    
    \begin{block}{El Problema del Orden ASCII}
    Por defecto, Python ordena las letras mayúsculas \textbf{antes} que las minúsculas (orden ASCII). Esto lleva a resultados confusos:
    \begin{lstlisting}[language=Python]
palabras = ['Zebra', 'manzana', 'Auto', 'elefante']

palabras.sort() # Orden por defecto (ASCII)
print(palabras)
# Imprime: ['Auto', 'Zebra', 'elefante', 'manzana']
# (A, Z, e, m) <-- ¡Incorrecto!
    \end{lstlisting}
    \end{block}
\end{frame}
\begin{frame}[fragile]
    \frametitle{Ejemplo de \texttt{key}: Ignorar Mayúsculas}
    \begin{alertblock}{La Solución: \texttt{key=str.lower}}
    Podemos pasar el método \texttt{str.lower} como la \texttt{key}. Esto le dice a Python que, \textit{solo para propósitos de comparación}, trate a todas las palabras como si estuvieran en minúsculas.
    \begin{lstlisting}[language=Python]
palabras = ['Zebra', 'manzana', 'Auto', 'elefante']

# Orden correcto, ignorando mayusculas
palabras.sort(key=str.lower)

print(palabras)
# Imprime: ['Auto', 'elefante', 'manzana', 'Zebra']
# (Compara: 'auto', 'elefante', 'manzana', 'zebra')
    \end{lstlisting}
    \end{alertblock}

\end{frame}

\section{Sort en Arreglos 2D (Matrices)}

%------------------------------------------------
\section{Práctica}
%------------------------------------------------

\begin{frame}[fragile, allowframebreaks]
    \frametitle{Tarea: Ordenando Datos Complejos}
    
    \begin{block}{El Problema}
    Tenemos una lista de estudiantes. Cada estudiante es, a su vez, una lista que contiene su nombre, su calificación y su número de faltas:
    
    \texttt{[Nombre, Calificaci\'on, Faltas]}
    \end{block}
\end{frame}

\begin{frame}[fragile, allowframebreaks]
    \frametitle{Tarea: Ordenando Datos Complejos}
    \begin{alertblock}{Estructura de Datos}
    \begin{lstlisting}[language=Python]
estudiantes = [
    ['Ana', 85, 2],
    ['Luis', 92, 0],
    ['Carla', 92, 3], # Carla y Luis tienen la misma nota
    ['Bruno', 76, 1]
]
    \end{lstlisting}
    \end{alertblock}
\end{frame}
\begin{frame}[fragile, allowframebreaks]
    \frametitle{Tarea: Ordenando Datos Complejos}   
    \begin{examples}{Tu Tarea}
    \begin{itemize}
        \item \textbf{Parte 1:} Ordena la lista de estudiantes por \textbf{calificaci\'on} (de menor a mayor).
        \item \textbf{Parte 2:} Ordena la lista de estudiantes por \textbf{n\'umero de faltas} (de menor a mayor).
    \end{itemize}
    \end{examples}
\end{frame}

%------------------------------------------------

\begin{frame}[fragile]
    \frametitle{Tarea: Pista (Definir una Función \texttt{key})}
    
    \begin{block}{No podemos usar \texttt{len} o \texttt{str.lower}}
    Necesitamos crear nuestra \textbf{propia función} que le diga a \texttt{.sort()} qu\'e \'indice de la sub-lista debe mirar.
    \end{block}
    \begin{alertblock}{Usamos \texttt{key = lambda}}
        key lambda nos ayuda a ordenar la matriz de acuerdo a una columna específica.
        \texttt{lambda columna: columna[indice]} donde \texttt{indice} es la columna que queremos usar para ordenar.
    \end{alertblock}
\end{frame}

\begin{frame}[fragile]
    \frametitle{Tarea: Pista (Definir una Función \texttt{key})}
    
    \begin{alertblock}{Código de Ejemplo (Parte 1: Por Calificaci\'on)}
    \begin{lstlisting}[language=Python]
estudiantes = [
    ['Ana', 85, 2], ['Luis', 92, 0],
    ['Carla', 92, 3], ['Bruno', 76, 1]
]

estudiantes.sort(key=lambda columna: columna[1])

print('Ordenados por calificacion (con lambda):')
print(estudiantes)
    \end{lstlisting}
    \end{alertblock}

\end{frame}

%----------------------------------------------------------------------------------------
%----------------------------------------------------------------------------------------

\begin{frame}
    \frametitle{Desafío de Programación: El Censo del Distrito}

    
    \begin{block}{Eres un trabajador del censo y necesitas registrar a los habitantes de un distrito.}
    \begin{itemize}
        \item Crear una lista vacía llamada \texttt{censo}.
        \item Iniciar un bucle infinito (\texttt{while True}) que pida al usuario: Nombre (string), Edad (int), Código Postal (int)
        \item \textbf{Condición de salida:} Si el usuario escribe \textbf{\texttt{"salir"}} en el nombre, el bucle debe romperse (\texttt{break}).
        \item \textbf{Almacenamiento:} Guardar cada ciudadano como una lista (ej. \texttt{['Ana', 30, 10300]}) dentro de la lista principal \texttt{censo}.
        \item \textbf{Reporte Final:} Al terminar el bucle, ordenar el \texttt{censo} por \textbf{edad}, de la persona \textbf{más vieja a la más joven}.
        \item Imprimir la lista \texttt{censo} ya ordenada.
    \end{itemize}
    \end{block}
\end{frame}
\end{document}

