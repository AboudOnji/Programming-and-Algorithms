%----------------------------------------------------------------------------------------
%	PAQUETES Y TEMAS
%----------------------------------------------------------------------------------------
\documentclass[aspectratio=169,xcolor=dvipsnames]{beamer}
\usetheme{SimpleDarkBlue}

\usepackage[spanish]{babel}
\usepackage{hyperref}
\usepackage{graphicx} % Allows including images
\usepackage{booktabs} % Allows the use of \toprule, \midrule and \bottomrule in tables
\usepackage{amsmath}
\usepackage{lettrine}
\setbeamertemplate{caption}[numbered]
\usepackage[dvipsnames,svgnames,x11names]{xcolor}% Para definir y usar colores (ej. en hipervínculos)
\usepackage{xurl}
\usepackage{hyperref}       % Para crear hipervínculos internos y externos
\usepackage{algorithm}
\usepackage{algorithmicx}
\usepackage{algpseudocode}
\hypersetup{
    colorlinks=true,        % Colorear los enlaces en lugar de usar recuadros
    linkcolor=blue,     % Color para enlaces internos (índice, referencias cruzadas)
    filecolor=blue, % Color para enlaces a archivos locales
    urlcolor=blue,      % Color para URLs
    citecolor=blue,     % Color para citas bibliográficas
}
%----------------------------------------------------------------------------------------

\usepackage{listings}
\usepackage{xcolor} % Para colores en listings
 \definecolor{codegreen}{rgb}{0,0.6,0}
 \definecolor{codegray}{rgb}{0.5,0.5,0.5}
 \definecolor{codepurple}{rgb}{0.58,0,0.82}
 \definecolor{backcolour}{rgb}{0.97,0.97,0.99}

\lstdefinestyle{PythonStyle}{
  language=Python,
  basicstyle=\ttfamily\footnotesize,
  keywordstyle=\color{blue}\bfseries,
  commentstyle=\color{codegreen},
  stringstyle=\color{violet},
  numberstyle=\tiny\color{gray},
  breakatwhitespace=false,
  breaklines=true,
  captionpos=b,
  keepspaces=true,
  numbers=left,
  numbersep=5pt,
  showspaces=false,
  showstringspaces=false,
  showtabs=false,
  tabsize=2,
  frame=lines, % Añade un marco alrededor del código
  framerule=0.4pt, % Grosor del marco
  backgroundcolor=\color{backcolour} % Color de fondo suave
}
\lstset{style=PythonStyle}
%	TITLE PAGE
%----------------------------------------------------------------------------------------
%	PÁGINA DE TÍTULO
%----------------------------------------------------------------------------------------

\title{Sesión de prácticas}
\subtitle{Materia: Algoritmos y Programación}

\author{Prof. D.Sc. BARSEKH-ONJI Aboud}
\institute
{
    Facultad de Ingeniería \\
    Universidad Anáhuac México
}
\date{\today}

%----------------------------------------------------------------------------------------
%	CONTENIDO DE LA PRESENTACIÓN
%----------------------------------------------------------------------------------------

% --- Agenda automática al inicio de cada sección ---
\AtBeginSection[]
{
  \begin{frame}{Agenda}
    \tableofcontents[currentsection]
  \end{frame}
}

\begin{document}

\begin{frame}
    \titlepage
\end{frame}

\section{uso de If - elif - else}

\begin{frame}{Ejercicio 1}
\begin{block}{Ejercicio 1}
Un instituto para aprendizaje en programnación ofrece los siguientes cursos para verano:
\begin{itemize}
    \item Python para principiantes, costo: 10,000.00\$
    \item Python intermedio, costo: 15,000.00\$
    \item Python avanzado, costo: 20,000.00\$
\end{itemize}

El usuario puede pagar con tarjetas de crédito o de débito, si paga con tarjeta de crédito se le cobra 7\% más al costo total.

Escribe un programa, \textbf{utilizando únicamente IF, IF-ELSE o ELSE-IF}, para preguntar al usuario que curso quiere elegir, y para mostrar el costo total que debe pagar de acuerdo con su forma de pago.

\end{block}
    
\end{frame}

\begin{frame}{Ejercicio 1}
\begin{block}{Ejercicio 1 - solución}
\url{https://github.com/AboudOnji/ExamplesAyP/blob/main/Example21.py}

\end{block}
    
\end{frame}

\section{uso de If - elif - else con For}

\begin{frame}{Ejercicio 2}
\begin{block}{Ejercicio 2}
Un verificentro evalúa los autos de acuerdo con el modelo del año de cada auto, usando este criterio para decidir cuantos días a la semana deben dejar de circular:
\begin{itemize}
    \item (Grupo D) Los autos entre 2000 a 2005 dejan de circular 2 días a la semana
    \item (Grupo C) Los autos entre 2006 a 2010 dejan de circular 1 día a la semana.
    \item (Grupo B) Los autos entre 2011 a 2020 dejan de circular 1 día al mes.
    \item (Grupo A) Los autos de 2021 a 2025 circulan todos los días.
\end{itemize}
Escribe un programa, utilizando \textbf{únicamente IF - Else IF - Else y FOR} que: muestra primero la tabla de cirterios, pide al usuario saber cuantos autos quiere evaluar, solicita al usuario el año de cada auto (uno por uno) y le muestra cuantos días dejará de circular ese auto. Al finalizar el programa debe mostrar cuantos autos se calificaron en cada grupo.
\end{block}
    
\end{frame}

\begin{frame}{Ejercicio 2}
\begin{block}{Ejercicio 2 - solución}
\url{https://github.com/AboudOnji/ExamplesAyP/blob/main/Example22.py}

\end{block}
\end{frame}
\section{uso de match ... case}

\begin{frame}{Ejercicio 3}
\begin{block}{Ejercicio 3}
Escribe un programa que permita a 10 jugadores elegir entre tres juegos diferentes (A, B, C). Utiliza una estructura \textbf{match-case} para solicitar que el usaurio elija un juego. El programa debe saludar al jugador por su nombre y confirmar su elección. Al finalizar muestra un conteo de cuántos jugadores eligieron cada juego.
\end{block}
    
\end{frame}

\begin{frame}{Ejercicio 3}
\begin{block}{Ejercicio 3 - solución}
\url{https://github.com/AboudOnji/ExamplesAyP/blob/main/Example20.py}

\end{block}
\end{frame}

%------------------------------------------------
%------------------------------------------------
\end{document}