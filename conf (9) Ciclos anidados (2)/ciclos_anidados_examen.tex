\documentclass[aspectratio=169,xcolor=dvipsnames]{beamer}
\usetheme{SimpleDarkBlue}

\usepackage[spanish]{babel}
\usepackage{hyperref}
\usepackage{graphicx} % Allows including images
\usepackage{booktabs} % Allows the use of \toprule, \midrule and \bottomrule in tables
\usepackage{amsmath}
\usepackage{lettrine}
\setbeamertemplate{caption}[numbered]
\usepackage[dvipsnames,svgnames,x11names]{xcolor}% Para definir y usar colores (ej. en hipervínculos)
\usepackage{xurl}
\usepackage{hyperref}       % Para crear hipervínculos internos y externos
\usepackage{algorithm}
\usepackage{algorithmicx}
\usepackage{algpseudocode}
\hypersetup{
    colorlinks=true,        % Colorear los enlaces en lugar de usar recuadros
    linkcolor=blue,     % Color para enlaces internos (índice, referencias cruzadas)
    filecolor=blue, % Color para enlaces a archivos locales
    urlcolor=blue,      % Color para URLs
    citecolor=blue,     % Color para citas bibliográficas
}
%----------------------------------------------------------------------------------------

\usepackage{listings}
\usepackage{xcolor} % Para colores en listings
 \definecolor{codegreen}{rgb}{0,0.6,0}
 \definecolor{codegray}{rgb}{0.5,0.5,0.5}
 \definecolor{codepurple}{rgb}{0.58,0,0.82}
 \definecolor{backcolour}{rgb}{0.97,0.97,0.99}

\lstdefinestyle{PythonStyle}{
  language=Python,
  basicstyle=\ttfamily\footnotesize,
  keywordstyle=\color{blue}\bfseries,
  commentstyle=\color{codegreen},
  stringstyle=\color{violet},
  numberstyle=\tiny\color{gray},
  breakatwhitespace=false,
  breaklines=true,
  captionpos=b,
  keepspaces=true,
  numbers=left,
  numbersep=5pt,
  showspaces=false,
  showstringspaces=false,
  showtabs=false,
  tabsize=2,
  frame=lines, % Añade un marco alrededor del código
  framerule=0.4pt, % Grosor del marco
  backgroundcolor=\color{backcolour} % Color de fondo suave
}
\lstset{style=PythonStyle}
%	TITLE PAGE


\title{Ciclos Anidados - Examen}
\subtitle{Materia: Algoritmos y Programación}

\author{Prof. D.Sc. BARSEKH-ONJI Aboud}
\institute
{
    Facultad de Ingeniería \\
    Universidad Anáhuac México
}
\date{\today}

%----------------------------------------------------------------------------------------

% --- Agenda automática al inicio de cada sección ---
\AtBeginSection[]
{
  \begin{frame}{Agenda}
    \tableofcontents[currentsection]
  \end{frame}
}

\begin{document}
\begin{frame}
    \titlepage
\end{frame}
\begin{frame}[fragile]
    \frametitle{Ejercicio de Práctica}

    \begin{block}{Objetivo y Contexto}
    Estás programando el sistema de reservas para una sala de cine. La sala se representa como una matriz (lista de listas) donde cada fila es una fila de asientos y cada columna es un número de asiento.
    
    Un \textbf{0} significa que el asiento está \textbf{vacío} y un \textbf{1} significa que está \textbf{ocupado}.
    \end{block}
    
    \begin{alertblock}{Datos: Mapa de la Sala}
    Usa la siguiente lista de listas como los datos de la sala:
    \begin{lstlisting}[language=Python]
sala_cine = [
  # Asiento 0, Asiento 1, Asiento 2, Asiento 3, Asiento 4
    [1, 1, 0, 0, 1],  # Fila 0
    [0, 0, 0, 1, 1],  # Fila 1
    [1, 1, 1, 1, 1],  # Fila 2
    [0, 0, 1, 1, 0],  # Fila 3
    [1, 0, 1, 0, 1]   # Fila 4
]
    \end{lstlisting}
    \end{alertblock}

\end{frame}

%------------------------------------------------

\begin{frame}[fragile]
    \frametitle{Ejercicio de Práctica (Instrucciones)}
    
    \begin{block}{Instrucciones del Examen}
    La gerencia del cine quiere saber dos cosas para su reporte de limpieza y seguridad:
    
    \begin{enumerate}
        \item \textbf{Ocupación en Pasillos:} Contar cuántas personas (valor \texttt{1}) están sentadas \textit{junto a los pasillos}. Los asientos de pasillo son la \textbf{columna 0}.
        
        \item \textbf{Ocupación Central:} Contar cuántas personas (valor \texttt{1}) están sentadas en los asientos \textit{centrales} (es decir, que \textbf{NO} están en la columna 0 ni en la última columna).
    \end{enumerate}
    \end{block}
\end{frame}
\begin{frame}[fragile]
    \frametitle{Ejercicio de Práctica (Instrucciones)}   
    \begin{alertblock}{Requisitos Obligatorios}
    \begin{itemize}
        \item Debes usar \textbf{ciclos anidados} (\texttt{for}).
        \item \textbf{Pista Clave:} Necesitarás usar los \textbf{índices} (especialmente el de la columna, \texttt{j}) para resolver esto.
       
    \end{itemize}
    \end{alertblock}

\end{frame}

%------------------------------------------------

\begin{frame}[fragile]
    \frametitle{Ejercicio de Práctica (Salida Esperada)}
    
    \begin{alertblock}{Salida Esperada}
    Tu programa debe imprimir en la consola un resultado idéntico a este:
    \begin{verbatim}
Analizando ocupacion de la sala...
Asientos ocupados en pasillos: 6
Asientos ocupados en zona central: 8
    \end{verbatim}
    \end{alertblock}

\end{frame}


\end{document}