\documentclass[aspectratio=169,xcolor=dvipsnames]{beamer}
\usetheme{SimpleDarkBlue}

\usepackage[spanish]{babel}
\usepackage{hyperref}
\usepackage{graphicx} % Allows including images
\usepackage{booktabs} % Allows the use of \toprule, \midrule and \bottomrule in tables
\usepackage{amsmath}
\usepackage{lettrine}
\setbeamertemplate{caption}[numbered]
\usepackage[dvipsnames,svgnames,x11names]{xcolor}% Para definir y usar colores (ej. en hipervínculos)
\usepackage{xurl}
\usepackage{hyperref}       % Para crear hipervínculos internos y externos
\usepackage{algorithm}
\usepackage{algorithmicx}
\usepackage{algpseudocode}
\hypersetup{
    colorlinks=true,        % Colorear los enlaces en lugar de usar recuadros
    linkcolor=blue,     % Color para enlaces internos (índice, referencias cruzadas)
    filecolor=blue, % Color para enlaces a archivos locales
    urlcolor=blue,      % Color para URLs
    citecolor=blue,     % Color para citas bibliográficas
}
%----------------------------------------------------------------------------------------

\usepackage{listings}
\usepackage{xcolor} % Para colores en listings
 \definecolor{codegreen}{rgb}{0,0.6,0}
 \definecolor{codegray}{rgb}{0.5,0.5,0.5}
 \definecolor{codepurple}{rgb}{0.58,0,0.82}
 \definecolor{backcolour}{rgb}{0.97,0.97,0.99}

\lstdefinestyle{PythonStyle}{
  language=Python,
  basicstyle=\ttfamily\footnotesize,
  keywordstyle=\color{blue}\bfseries,
  commentstyle=\color{codegreen},
  stringstyle=\color{violet},
  numberstyle=\tiny\color{gray},
  breakatwhitespace=false,
  breaklines=true,
  captionpos=b,
  keepspaces=true,
  numbers=left,
  numbersep=5pt,
  showspaces=false,
  showstringspaces=false,
  showtabs=false,
  tabsize=2,
  frame=lines, % Añade un marco alrededor del código
  framerule=0.4pt, % Grosor del marco
  backgroundcolor=\color{backcolour} % Color de fondo suave
}
\lstset{style=PythonStyle}
%	TITLE PAGE


\title{Ciclos Anidados - Datos bidimensionales (2)}
\subtitle{Materia: Algoritmos y Programación}

\author{Prof. D.Sc. BARSEKH-ONJI Aboud}
\institute
{
    Facultad de Ingeniería \\
    Universidad Anáhuac México
}
\date{\today}

%----------------------------------------------------------------------------------------
%	CONTENIDO DE LA PRESENTACIÓN
%----------------------------------------------------------------------------------------

% --- Agenda automática al inicio de cada sección ---
\AtBeginSection[]
{
  \begin{frame}{Agenda}
    \tableofcontents[currentsection]
  \end{frame}
}

\begin{document}

\begin{frame}
    \titlepage
\end{frame}

%------------------------------------------------
\section{Ordenamiento de Arreglos 2D}
%------------------------------------------------

\begin{frame}[fragile]{¿Qué vamos a ordenar?}
    
    \begin{block}{Objetivo del Ejercicio}
    En lugar de ordenar las filas entre sí, nuestro objetivo de hoy es ordenar los elementos \textbf{DENTRO} de cada fila.
    
    Queremos transformar una matriz desordenada en una donde cada fila individual esté ordenada de menor a mayor.
    \end{block}
\end{frame}
\begin{frame}[fragile]{¿Qué vamos a ordenar?}

    \begin{alertblock}{Transformación Visual}
    \textbf{Matriz Original:}
    \begin{verbatim}
[
    [8, 5, 10],
    [9, 9, 7],
    [6, 10, 7]
]
    \end{verbatim}
    \textbf{Matriz Resultante (Orden Interno):}
    \begin{verbatim}
[
    [5, 8, 10],
    [7, 9, 9],
    [6, 7, 10]
]
    \end{verbatim}
    \end{alertblock}
\end{frame}

%------------------------------------------------

\begin{frame}[fragile]
    \frametitle{Ejemplo: Ordenando Calificaciones}

    \begin{block}{Objetivo}
    Dada una matriz que representa las calificaciones de varias tareas para diferentes estudiantes, ordenar las calificaciones de \textbf{cada} estudiante (cada fila) de menor a mayor.
    \end{block}
    
    \begin{alertblock}{Datos de Entrada}
    \begin{lstlisting}[language=Python]
calificaciones = [
    [8, 5, 10, 7],  # Notas del Estudiante 1
    [9, 9, 6, 8],   # Notas del Estudiante 2
    [10, 7, 7, 9]   # Notas del Estudiante 3
]
    \end{lstlisting}
    \end{alertblock}
    
\end{frame}

%------------------------------------------------

\begin{frame}[fragile]
    \frametitle{Lógica de la Solución (Burbuja)}

    \begin{block}{¿Cómo resolverlo?}
    Necesitamos una combinación de ciclos:
    \begin{enumerate}
        \item \textbf{Ciclo Externo:} Un ciclo \texttt{for} que recorra la matriz principal (\texttt{calificaciones}) para obtener \textbf{cada fila} (cada estudiante).
        
        \item \textbf{Ciclos Anidados (Ordenamiento):} Por \textit{cada} fila obtenida, debemos aplicar un algoritmo de ordenamiento. Usaremos el \textbf{Ordenamiento de Burbuja}.
    \end{enumerate}
    \end{block}
\end{frame}
\begin{frame}[fragile]
    \frametitle{Lógica de la Solución (Burbuja)}
    \begin{alertblock}{Recordatorio: Ordenamiento de Burbuja}
    Para ordenar una lista 1D (llamada \texttt{lista}):
    \begin{lstlisting}[language=Python, numbers=none, frame=none, backgroundcolor=\color{backcolour!0}]
n = len(lista)
for i in range(n):
    for j in range(0, n - i - 1):
        if lista[j] > lista[j+1]:
            # Hacemos el "swap" (intercambio)
            lista[j], lista[j+1] = lista[j+1], lista[j]
    \end{lstlisting}
    \end{alertblock}
\end{frame}

%------------------------------------------------

\begin{frame}[fragile]
    \frametitle{Ejemplo: Código Completo y Salida}

            \begin{block}{Código en Python}
                \begin{lstlisting}[language=Python]
calificaciones = [
    [8, 5, 10, 7],
    [9, 9, 6, 8],
    [10, 7, 7, 9]
]
print("--- Matriz Original ---")
for fila in calificaciones:
    print(fila)

for fila in calificaciones:
    n = len(fila)
    for i in range(n):
        for j in range(0, n - i - 1):
            if fila[j] > fila[j+1]:
                # Intercambiamos
                fila[j], fila[j+1] = fila[j+1], fila[j]
print("\n--- Matriz Ordenada ---")

                \end{lstlisting}
            \end{block}
\end{frame}

\begin{frame}[fragile]
    \frametitle{Ejemplo: Código Completo y Salida}
            \begin{block}{Punto Clave}
            Estamos modificando la \texttt{fila} ''en el lugar''. Como las listas son mutables, la matriz \texttt{calificaciones} original se actualiza automáticamente.
            \end{block}
\begin{block}{código en Python}
    \url{https://github.com/AboudOnji/ExamplesAyP/blob/main/Example30.py}
\end{block}

\end{frame}


%------------------------------------------------
\end{document}