\documentclass[aspectratio=169,xcolor=dvipsnames]{beamer}
\usetheme{SimpleDarkBlue}

\usepackage[spanish]{babel}
\usepackage{hyperref}
\usepackage{graphicx} % Allows including images
\usepackage{booktabs} % Allows the use of \toprule, \midrule and \bottomrule in tables
\usepackage{amsmath}
\usepackage{lettrine}
\setbeamertemplate{caption}[numbered]
\usepackage[dvipsnames,svgnames,x11names]{xcolor}% Para definir y usar colores (ej. en hipervínculos)
\usepackage{xurl}
\usepackage{hyperref}       % Para crear hipervínculos internos y externos
\usepackage{algorithm}
\usepackage{algorithmicx}
\usepackage{algpseudocode}
\hypersetup{
    colorlinks=true,        % Colorear los enlaces en lugar de usar recuadros
    linkcolor=blue,     % Color para enlaces internos (índice, referencias cruzadas)
    filecolor=blue, % Color para enlaces a archivos locales
    urlcolor=blue,      % Color para URLs
    citecolor=blue,     % Color para citas bibliográficas
}
%----------------------------------------------------------------------------------------

\usepackage{listings}
\usepackage{xcolor} % Para colores en listings
 \definecolor{codegreen}{rgb}{0,0.6,0}
 \definecolor{codegray}{rgb}{0.5,0.5,0.5}
 \definecolor{codepurple}{rgb}{0.58,0,0.82}
 \definecolor{backcolour}{rgb}{0.97,0.97,0.99}

\lstdefinestyle{PythonStyle}{
  language=Python,
  basicstyle=\ttfamily\footnotesize,
  keywordstyle=\color{blue}\bfseries,
  commentstyle=\color{codegreen},
  stringstyle=\color{violet},
  numberstyle=\tiny\color{gray},
  breakatwhitespace=false,
  breaklines=true,
  captionpos=b,
  keepspaces=true,
  numbers=left,
  numbersep=5pt,
  showspaces=false,
  showstringspaces=false,
  showtabs=false,
  tabsize=2,
  frame=lines, % Añade un marco alrededor del código
  framerule=0.4pt, % Grosor del marco
  backgroundcolor=\color{backcolour} % Color de fondo suave
}
\lstset{style=PythonStyle}
%	TITLE PAGE


\title{Ciclos Anidados - Datos bidimensionales}
\subtitle{Materia: Algoritmos y Programación}

\author{Prof. D.Sc. BARSEKH-ONJI Aboud}
\institute
{
    Facultad de Ingeniería \\
    Universidad Anáhuac México
}
\date{\today}

%----------------------------------------------------------------------------------------
%	CONTENIDO DE LA PRESENTACIÓN
%----------------------------------------------------------------------------------------

% --- Agenda automática al inicio de cada sección ---
\AtBeginSection[]
{
  \begin{frame}{Agenda}
    \tableofcontents[currentsection]
  \end{frame}
}

\begin{document}

\begin{frame}
    \titlepage
\end{frame}

%------------------------------------------------
\section{Ejemplo 4: Dibujando un Triángulo}


\begin{frame}[fragile]
    \frametitle{Ejemplo 4: Dibujando un Triángulo}

    \begin{block}{Objetivo}
    Escribir un programa que le pida al usuario un número entero (que representará la altura) y que, usando ciclos anidados, dibuje un triángulo rectángulo de asteriscos de esa misma altura.
    \end{block}
    
    \begin{alertblock}{Salida Esperada}
    Si el usuario introduce el número \textbf{5}, la salida en consola debe ser:
    \begin{verbatim}
*
**
***
****
*****
    \end{verbatim}
    \end{alertblock}
\end{frame}

%------------------------------------------------

\begin{frame}[fragile]
    \frametitle{Ejemplo 4: Lógica y Pistas}
    
    \begin{block}{Nuestra Lógica}
    Este problema nos obliga a pensar en cómo el número de iteraciones del ciclo interno cambia en cada paso del ciclo externo.
    \begin{itemize}
        \item \textbf{Ciclo Externo (Filas):} Controlará el número de filas del triángulo. Si la altura es 5, este ciclo debe ejecutarse 5 veces (una por cada línea).
        \pause
        \item \textbf{Ciclo Interno (Columnas):} Controlará cuántos asteriscos imprimir en la fila \textbf{actual}. 
        \pause
        \item \textbf{La Clave:} Observa el patrón. En la fila 1, hay 1 asterisco. En la fila 2, hay 2 asteriscos, y así sucesivamente. Por lo tanto, el número de veces que el ciclo interno debe ejecutarse es igual al número de la fila actual del ciclo externo.
        \pause
        \item \textbf{Pista de Impresión:} Para imprimir los asteriscos en la misma línea, puedes usar \texttt{print(''*'', end='''')}. Después de que el ciclo interno termine, necesitarás un \texttt{print()} vacío para pasar a la siguiente línea.
    \end{itemize}
    \end{block}

\end{frame}

%------------------------------------------------

\begin{frame}[fragile]
    \frametitle{Ejemplo 4: Código y Salida}
    
    \begin{columns}[t]
        \column{.5\textwidth}
            \begin{block}{Código en Python}
                \begin{lstlisting}[language=Python]
altura = int(input(''Introduce la altura del triangulo: ''))
# Ciclo externo para cada fila
for fila in range(1, altura + 1):
    # Se ejecuta 'fila' veces
    for columna in range(fila):
        print(''*'', end='''')
    print() # Salto de línea después de cada fila
                \end{lstlisting}
            \end{block}

        \column{.5\textwidth}
            \begin{alertblock}{Salida en Consola (para altura = 7)}
                \begin{verbatim}
Introduce la altura del triangulo: 7
*
**
***
****
*****
******
*******
                \end{verbatim}
            \end{alertblock}
    \end{columns}
\end{frame}


%------------------------------------------------
\section{Estructuras de Datos 2D: Listas de Listas}
%------------------------------------------------

\begin{frame}[fragile]{De una Dimensión a Dos Dimensiones}
    
    \begin{block}{Recordatorio: Listas Unidimensionales}
    Hasta ahora, hemos trabajado con listas simples, que son como una fila de casillas. Cada elemento tiene una única posición o índice.
    \begin{lstlisting}[language=Python]
miLista = [''A'', ''B'', ''C'', ''D''] # Accedemos con miLista[i]
    \end{lstlisting}
    \end{block}
    
    \begin{alertblock}{El Siguiente Nivel: Listas de Listas}
    Una lista de listas es exactamente lo que su nombre indica: una lista que en lugar de contener números o strings, contiene \textbf{otras listas}.
    \begin{itemize}
        \item Nos permite organizar la información en una estructura de \textbf{filas y columnas}, como una tabla de Excel, un tablero de ajedrez o una matriz matemática.
        \item Se conocen como estructuras de datos bidimensionales (2D).
    \end{itemize}
    \end{alertblock}

\end{frame}

%------------------------------------------------

\begin{frame}[fragile]{Anatomía de una Lista de Listas}
    
    \begin{block}{Sintaxis y Estructura Visual}
    Imagina una matriz simple de 2 filas y 3 columnas.
    \begin{lstlisting}[language=Python]
matriz = [
    [1,    2,    3],  # Fila 0
    [4,    5,    6]   # Fila 1
]
    \end{lstlisting}
    
    \begin{itemize}
        \item La \textbf{lista exterior} contiene todas las filas.
        \item Cada \textbf{lista interior} representa una fila completa.
    \end{itemize}
    \end{block}
\end{frame}
\begin{frame}[fragile]{Anatomía de una Lista de Listas}

    \begin{alertblock}{Accediendo a los Elementos: Doble Índice}
    Para llegar a un elemento específico, necesitamos dos coordenadas: primero la fila, y luego la columna. Usamos la notación: \texttt{nombre\_lista[indice\_fila][indice\_columna]}
    \begin{itemize}
        \item \texttt{matriz[0]} nos daría la primera fila completa: \texttt{[1, 2, 3]}
        \item \texttt{matriz[1][2]} nos daría el elemento en la fila 1, columna 2: \textbf{6}
    \end{itemize}
    \end{alertblock}

\end{frame}

%------------------------------------------------

            \begin{frame}[fragile]{Ejemplo Práctico: Un Tablero de Gato (Tic-Tac-Toe)}

                \begin{block}{Objetivo}
    Vamos a crear un tablero de Gato de 3x3 usando una lista de listas. Luego, colocaremos una ''X'' en el centro del tablero y lo mostraremos en consola.
    \end{block}
            \begin{alertblock}{Salida en Consola}
                \begin{verbatim}
Tablero de Gato:
['-', '-', '-']
['-', 'X', '-']
['-', '-', '-']
                \end{verbatim}
            \end{alertblock}
            
            \begin{block}{Punto Clave}
            La línea \texttt{tablero[1][1] = ''X''} es la más importante. Demuestra cómo podemos \textbf{leer y escribir} en una posición específica de nuestra estructura 2D sin necesidad de ciclos anidados.
            \end{block}
\end{frame}

\begin{frame}[fragile]{Ejemplo Práctico: Un Tablero de Gato (Tic-Tac-Toe)}

    
    

            \begin{block}{Código en Python}
                \begin{lstlisting}[language=Python]
tablero = [
    [''-'', ''-'', ''-''],
    [''-'', ''-'', ''-''],
    [''-'', ''-'', ''-'']
]

tablero[1][1] = ''X''

print(''Tablero de Gato:'')
for fila in tablero:
    print(fila)

                \end{lstlisting}
            \end{block}
\end{frame}


\begin{frame}
    \frametitle{Conectando con el Siguiente Tema}

    \begin{block}{¿Y ahora qué?}
    Ya sabemos cómo acceder a \textbf{UN} elemento específico. Pero, ¿qué pasaría si quisiéramos hacer algo con \textbf{TODOS} los elementos?
    \begin{itemize}
        \item ¿Cómo sumaríamos todas las calificaciones de todos los estudiantes?
        \item ¿Cómo buscaríamos un nombre en una lista de invitados sentados en una mesa rectangular?
    \end{itemize}
    \end{block}
    
    \begin{alertblock}{La Solución: Ciclos Anidados}
    Para recorrer sistemáticamente cada fila y, dentro de cada fila, cada columna, necesitamos una herramienta que haga precisamente eso. Aquí es donde los \textbf{ciclos anidados} se vuelven indispensables.
    
    Ahora sí, veamos el ejemplo de las calificaciones...
    \end{alertblock}
\end{frame}
\section{Ejemplo 5: Procesando Datos de Estudiantes}
\begin{frame}[fragile]
    \frametitle{Ejemplo 5: Procesando Datos de Estudiantes}

    \begin{block}{Objetivo}
    Tenemos una ''lista de listas'' donde cada lista interna representa las calificaciones de un estudiante. El objetivo es escribir un programa que itere a través de esta estructura para calcular e imprimir el promedio de calificaciones de \textbf{cada} estudiante.
    \end{block}
    
\end{frame}

\begin{frame}[fragile]
    \frametitle{Ejemplo 5: Procesando Datos de Estudiantes}

    \begin{alertblock}{Estructura de Datos y Salida Esperada}
    Dada la siguiente lista:
    \begin{lstlisting}[language=Python]
calificaciones_grupo = [
    [10, 9, 10],  # Calificaciones del Estudiante 1
    [8, 7, 8],   # Calificaciones del Estudiante 2
    [9, 9, 10]   # Calificaciones del Estudiante 3
]
    \end{lstlisting}
    La salida esperada en consola es:
    \begin{verbatim}
Promedio del Estudiante 1: 9.67
Promedio del Estudiante 2: 7.67
Promedio del Estudiante 3: 9.33
    \end{verbatim}
    \end{alertblock}
\end{frame}

%------------------------------------------------

\begin{frame}[fragile]
    \frametitle{Ejemplo 5: Lógica y Pistas}
    
    \begin{block}{Nuestra Lógica}
    Para resolver esto, el ciclo externo se encargará de un estudiante a la vez, y el ciclo interno se encargará de procesar todas las calificaciones de ESE estudiante.
    \begin{itemize}
        \item \textbf{Ciclo Externo (por Estudiante):} Recorrerá la lista principal \texttt{calificaciones\_grupo}. En cada iteración, obtendremos una de las listas internas (ej: \texttt{[10, 9, 10]}).
        \pause
        \item \textbf{Inicializar un Acumulador:} \textbf{Antes} de que empiece el ciclo interno, debemos crear una variable (por ejemplo, \texttt{suma = 0}). Es crucial que se reinicie a cero por cada nuevo estudiante.
        \pause
        \item \textbf{Ciclo Interno (por Calificación):} Para la lista de calificaciones del estudiante actual, este ciclo recorrerá cada número (cada calificación) y lo sumará a nuestra variable \texttt{suma}.
        \pause
        \item \textbf{Calcular el Promedio:} \textbf{Después} de que el ciclo interno haya terminado de sumar todas las notas, calculamos el promedio. La fórmula es \texttt{suma / cantidad\_de\_calificaciones}.
    \end{itemize}
    \end{block}

\end{frame}

%------------------------------------------------

\begin{frame}[fragile]
    \frametitle{Ejemplo 6: Código y Salida}
    
            \begin{block}{Código en Python}
                \begin{lstlisting}[language=Python]
calificaciones_grupo = [
    [10, 9, 10],
    [8, 7, 8],
    [9, 9, 10]
]

for i in range (len(calificaciones_grupo)):
    suma_de_notas = 0 
    for j in range (len(calificaciones_grupo[i])):
        suma_de_notas += calificaciones_grupo[i][j]
    promedio = suma_de_notas / len(calificaciones_grupo[i])
    print(f"Promedio del Estudiante {i + 1}: {promedio:.2f}")

                \end{lstlisting}
            \end{block}
\end{frame}

\section{Ejemplo 6: Encontrando el Valor Máximo}
\begin{frame}[fragile]
    \frametitle{Ejemplo 6: Encontrando el Valor Máximo}

    \begin{block}{Objetivo}
    Dada una matriz (lista de listas) que representa las ventas diarias de diferentes sucursales durante una semana, escribir un programa que encuentre e imprima cuál fue la venta más alta registrada en general.
    \end{block}
\end{frame}
\begin{frame}[fragile]
    \frametitle{Ejemplo 6: Encontrando el Valor Máximo}

    \begin{alertblock}{Estructura de Datos y Salida Esperada}
    Dada la siguiente matriz de ventas:
    \begin{lstlisting}[language=Python]
ventas_semanales = [
  # Lun, Mar, Mie, Jue, Vie
    [150, 200, 180, 220, 250], # Sucursal 1
    [180, 190, 210, 200, 230], # Sucursal 2
    [120, 130, 110, 140, 100], # Sucursal 3
    [260, 240, 280, 270, 290]  # Sucursal 4
]
    \end{lstlisting}
    La salida esperada en consola es:
    \begin{verbatim}
La venta mas alta registrada fue: 290
    \end{verbatim}
    \end{alertblock}
\end{frame}

%------------------------------------------------

\begin{frame}[fragile]
    \frametitle{Ejemplo 6: Lógica y Pistas}
    
    \begin{block}{Nuestra Lógica}
    Necesitamos una variable que "recuerde" el número más grande que hemos visto hasta ahora. Luego, compararemos cada venta con ese recuerdo.
    \begin{itemize}
        \item \textbf{Inicializar una variable de seguimiento:} Antes de empezar los ciclos, crea una variable llamada \texttt{venta\_maxima}. Una forma segura de inicializarla es con el \textbf{primer elemento} de la matriz (\texttt{ventas\_semanales[0][0]}).
        \pause
        \item \textbf{Ciclo Externo (por Fila):} Recorrerá cada lista de sucursal.
        \pause
        \item \textbf{Ciclo Interno (por Venta):} Recorrerá cada venta individual dentro de la lista de la sucursal actual.
        \pause
        \item \textbf{La Comparación (el `if`):} Dentro del ciclo interno, compara la venta actual con tu variable \texttt{venta\_maxima}. Si la venta actual es \textbf{mayor}, ¡actualiza \texttt{venta\_maxima} con este nuevo valor!
        \pause
        \item \textbf{Imprimir el Resultado Final:} \textbf{Después} de que ambos ciclos hayan terminado.
    \end{itemize}
    \end{block}

\end{frame}

%------------------------------------------------

\begin{frame}[fragile]
    \frametitle{Ejemplo 6: Código y Salida}
    
            \begin{block}{Código en Python}
                \begin{lstlisting}[language=Python]
ventas_semanales = [
    [150, 200, 180, 220, 250],
    [180, 190, 210, 200, 230],
    [120, 130, 110, 140, 100],
    [260, 240, 280, 270, 290]
]

venta_maxima = ventas_semanales[0][0]

for sucursal in ventas_semanales:
    for venta in sucursal:
        if venta > venta_maxima:
            venta_maxima = venta

print(f"La venta mas alta registrada fue: {venta_maxima}")
                \end{lstlisting}
            \end{block}


\end{frame}

\begin{frame}{Enlaces de los ejemplos}
    \begin{block}{Ejemplo 4: Dibujando un Triángulo}
        \url{https://github.com/AboudOnji/ExamplesAyP/blob/main/Example26.py}
    \end{block} 
    \begin{block}{Ejemplo 5: Procesando Datos de Estudiantes}
        \url{https://github.com/AboudOnji/ExamplesAyP/blob/main/Example28.py}
    \end{block}
    \begin{block}{Ejemplo 6: Encontrando el Valor Máximo}
        \url{https://github.com/AboudOnji/ExamplesAyP/blob/main/Example29.py}
    \end{block}
\end{frame}
%------------------------------------------------
\end{document}