%----------------------------------------------------------------------------------------
%	PAQUETES Y TEMAS
%----------------------------------------------------------------------------------------
\documentclass[aspectratio=169,xcolor=dvipsnames]{beamer}
\usetheme{SimpleDarkBlue}

\usepackage[spanish]{babel}
\usepackage{graphicx} % Permite incluir imágenes
\usepackage{booktabs} % Permite el uso de \toprule, \midrule y \bottomrule en tablas
\usepackage{amsmath}
\usepackage{lettrine}
\usepackage[dvipsnames,svgnames,x11names]{xcolor}
\usepackage{xurl}
\usepackage{hyperref} % Para crear hipervínculos
\hypersetup{
    colorlinks=true,
    linkcolor=blue, % Un azul más claro para temas oscuros
    filecolor=blue,
    urlcolor=blue,
    citecolor=blue,
}

% --- Numeración de figuras y tablas ---
\setbeamertemplate{caption}[numbered]

%----------------------------------------------------------------------------------------
%	PÁGINA DE TÍTULO
%----------------------------------------------------------------------------------------

\title{Herramientas para Programar en Python}
\subtitle{Entornos de Desarrollo Integrado (IDEs) y Editores de Código}

\author{Prof. D.Sc. BARSEKH-ONJI Aboud}
\institute
{
    Facultad de Ingeniería \\
    Universidad Anáhuac México % Puedes ajustar esto si es necesario
}
\date{\today}

%----------------------------------------------------------------------------------------
%	CONTENIDO DE LA PRESENTACIÓN
%----------------------------------------------------------------------------------------

% --- Agenda automática al inicio de cada sección ---
\AtBeginSection[]
{
  \begin{frame}{Agenda}
    \tableofcontents[currentsection]
  \end{frame}
}

\begin{document}

\begin{frame}
    \titlepage
\end{frame}

% --- Introducción a la Agenda General ---
\begin{frame}{Agenda de la Sesión}
    \tableofcontents
\end{frame}

%------------------------------------------------
\section{¿Por qué Aprender a Programar?}
%------------------------------------------------

\begin{frame}{El Diálogo entre Humanos y Computadoras}
	\begin{block}{El Problema Fundamental}
		Las computadoras son increíblemente poderosas, pero solo entienden un lenguaje: el \textbf{código binario}.
	\end{block}
	
	\begin{columns}[c]
		\column{0.4\textwidth}
			\begin{alertblock}{Lenguaje de Máquina (Binario)}
				\texttt{01101011 01101011}\\
				\texttt{11000000 00110101}\\
				\texttt{10111101 00100100}
				
				\vspace{1em}
				Esto es incomprensible y casi imposible de escribir para un humano.
			\end{alertblock}
		\column{0.6\textwidth}
			\begin{block}{La Solución: Lenguajes de Programación}
				 Son lenguajes formales que actúan como un \textbf{puente}. Nos permiten escribir instrucciones de forma legible para nosotros, que luego son traducidas a un formato que la computadora puede ejecutar.
			\end{block}
	\end{columns}
\end{frame}

\begin{frame}{Niveles de Abstracción}

	\begin{block}{No todos los lenguajes son iguales}
	Los lenguajes de programación se clasifican por su \textbf{nivel de abstracción}, es decir, qué tan lejos están del hardware de la computadora.
	
	\end{block}
	
	\begin{figure}
		\centering
		% NOTA: Asegúrate de tener esta imagen en la carpeta Figuras/cap01/
		\includegraphics[width=0.7\textwidth]{Figuras/cap01/fig01_1.png}
		\caption{Aumento de la abstracción desde el código máquina hasta Python.}
		\label{fig:abstraccion}
	\end{figure}

	
\end{frame}
\begin{frame}{Niveles de Abstracción}

\begin{alertblock}{Lenguajes de Alto Nivel (como Python)}
	Nos permiten expresar ideas complejas de forma simple, sin preocuparnos por los detalles de la máquina (registros, memoria, etc.).
	\end{alertblock}
\end{frame}
%------------------------------------------------
\section{¿Cómo 'Entiende' la Máquina a Python?}
%------------------------------------------------

\begin{frame}{Compilación vs. Interpretación}
	\frametitle{Traduciendo nuestras ideas a código máquina}
	\begin{columns}[c]
		\column{0.5\textwidth}
			\begin{block}{Compilación (Ej: C++)}
				\begin{itemize}
					\item<1-> Un \textbf{compilador} traduce \textbf{todo} el código fuente a un archivo ejecutable en código máquina \textbf{antes} de que el programa se ejecute.
					\pause
					\item<2-> El resultado es un programa muy rápido, pero específico para una arquitectura (ej. Windows x86).
				\end{itemize}
				% \includegraphics[width=\linewidth]{...} % Puedes poner la figura de compilación aquí
			\end{block}
		
		\column{0.5\textwidth}
			\begin{alertblock}{Interpretación (Ej: Python)}
				\begin{itemize}
					\item<1-> Un \textbf{intérprete} lee el código fuente línea por línea y lo ejecuta \textbf{al momento}.
					\pause
					\item<2-> No se crea un archivo ejecutable final, lo que lo hace más flexible y portable.
				\end{itemize}
			\end{alertblock}
	\end{columns}
	\begin{block}{Nota}
	La distinción real es más compleja. Python, de hecho, usa un paso intermedio.
	\end{block}
\end{frame}

\begin{frame}{El Proceso de Ejecución en Python}
	\frametitle{¿Qué pasa al hacer clic en 'Run'?}
	
\begin{figure}
		\centering
		% NOTA: Asegúrate de tener esta imagen en la carpeta Figuras/cap01/
		\includegraphics[width=0.6\textwidth]{Figuras/cap01/fig01-2.png}
		\caption{Ejecución de un programa en Python.}
		\label{fig:ejecucion_python}
	\end{figure}
\end{frame}    
\begin{frame}{El Proceso de Ejecución en Python}
	\frametitle{¿Qué pasa al hacer clic en 'Run'?}
	
	\begin{block}{De Código Fuente a Ejecución}
	Python combina lo mejor de ambos mundos con un proceso de dos pasos.
	\end{block}
	
	
	
	\begin{enumerate}
		\item<1-> El intérprete lee tu archivo \texttt{.py} y lo traduce a un código intermedio llamado \textbf{bytecode}.
		\pause
		\item<2-> Este bytecode (que es portable) es ejecutado por la \textbf{Máquina Virtual de Python (PVM)}.
	\end{enumerate}
\end{frame}

%------------------------------------------------
\section{Introducción a Python}
%------------------------------------------------

\begin{frame}{¿Por qué elegimos Python?}
	\begin{columns}[c]
		\column{0.4\textwidth}
			\begin{figure}
				\centering
				% NOTA: Asegúrate de tener esta imagen en la carpeta Figuras/cap01/
				\includegraphics[width=\linewidth]{Figuras/cap01/van.png}
				\caption{Guido van Rossum, creador de Python.}
			\end{figure}
		\column{0.6\textwidth}
			\begin{alertblock}{Filosofía de Diseño}
				La filosofía de Python enfatiza la \textbf{legibilidad} y la \textbf{simplicidad}. El objetivo es permitir a los programadores escribir código claro y lógico.
			\end{alertblock}
			
			\begin{block}{Características Clave}
				\begin{itemize}
					\item \textbf{Sintaxis simple:} Se parece al inglés, facilitando el aprendizaje.
					\item \textbf{Versatilidad:} Útil para desarrollo web, ciencia de datos, IA, automatización y más.
					\item \textbf{Gran comunidad:} Enorme cantidad de librerías y frameworks (NumPy, Pandas, Django).
				\end{itemize}
			\end{block}
	\end{columns}
\end{frame}

\begin{frame}{Popularidad y Aplicaciones}
	\begin{block}{Python en el Mundo Real}
	Python se ha convertido en el lenguaje de facto en muchas de las áreas tecnológicas más importantes.
	\end{block}
	
			\begin{alertblock}{Principales Campos de Aplicación}
			\begin{itemize}
				\item Ciencia de Datos y Machine Learning
				\item Desarrollo Web (Backend con Django/Flask)
				\item Automatización de Tareas (Scripting)
				\item Cómputo Científico y de Ingeniería
			\end{itemize}
			\end{alertblock}

            
\end{frame}
\begin{frame}{Popularidad y Aplicaciones}

\begin{figure}
				\centering
				% NOTA: Asegúrate de tener esta imagen en la carpeta Figuras/cap01/
				\includegraphics[width=0.75\linewidth]{Figuras/cap01/pythonrating.png}
				\caption{Popularidad de lenguajes (Índice TIOBE).}
			\end{figure}
\end{frame}
%------------------------------------------------
\section{Herramientas para Escribir Python (IDEs)}
%------------------------------------------------

\begin{frame}{¿Qué es un Entorno de Desarrollo Integrado (IDE)?}
    \begin{block}{El desafío}
    Escribir código involucra muchas tareas: editar, ejecutar, encontrar errores, instalar librerías... Hacerlo con herramientas separadas es ineficiente.
    \end{block}
    \pause
    \begin{alertblock}{La solución: Un Entorno de Desarrollo Integrado (IDE) ��}
    Un IDE es una aplicación que \textbf{consolida todas las herramientas esenciales} para el desarrollo de software en una sola interfaz gráfica.
    \end{alertblock}
\end{frame}

\begin{frame}{Componentes Clave de un IDE}
    \begin{columns}[c]
        \column{0.5\textwidth}
            \begin{block}{1. Editor de Código Fuente}
            Un editor de texto 'inteligente'.
                \begin{itemize}
                    \item Resaltado de sintaxis
                    \item Autocompletado
                    \item Formateo automático
                \end{itemize}
            \end{block}
             \begin{block}{3. Automatización y Ejecución}
             \begin{itemize}
                    \item Botón de 'Play' para correr el script.
                    \item Consola integrada para ver la salida.
                \end{itemize}
            \end{block}

        \column{0.5\textwidth}
            \begin{alertblock}{2. Depurador (Debugger)}
            La herramienta más importante para encontrar y corregir errores.
                \begin{itemize}
                    \item Puntos de ruptura (\textit{breakpoints})
                    \item Ejecución paso a paso
                    \item Inspección de variables
                \end{itemize}
            \end{alertblock}
    \end{columns}
\end{frame}

\begin{frame}[fragile]{Tabla Comparativa de Herramientas}
    \frametitle{Elegir la herramienta adecuada para el trabajo}
    \centering
    \resizebox{0.95\textwidth}{!}{%
    \begin{tabular}{lllll}
    \toprule
    \textbf{Plataforma/Editor} & \textbf{Nivel} & \textbf{Ideal para...} & \textbf{Ventajas Principales} & \textbf{Desventajas Principales} \\
    \midrule
    \texttt{Google Colab} & Fácil & Machine Learning, colaboración & Cero instalación, GPU gratis & Dependencia de internet, temporal \\
    \texttt{Thonny} & Muy Fácil & Principiantes absolutos & Simple, depurador visual & Limitado para proyectos grandes \\
    \texttt{Jupyter Notebook} & Fácil & Ciencia de datos, experimentación & Interactivo, visualización & No ideal para software robusto \\
    \texttt{VS Code} & Intermedio & Desarrollo general, personalización & Versátil, ligero, extensiones & Requiere configuración inicial \\
    \texttt{Spyder} & Fácil-Inter. & Ingeniería, computación científica & Similar a MATLAB, expl. de variables & Interfaz densa para principiantes \\
    \texttt{PyCharm (Comm.)} & Inter.-Avan. & Desarrollo profesional & Muy potente, autocompletado & Curva de aprendizaje, pesado \\
    \bottomrule
    \end{tabular}
    }
\end{frame}

%------------------------------------------------
\section{Recomendación para la Clase}
%------------------------------------------------

\begin{frame}{¿Qué usaremos en este curso?}
    \begin{block}{Nuestra Elección: La Distribución Anaconda}
    Para asegurar que todos tengamos el mismo entorno y minimizar problemas de instalación, usaremos la \textbf{distribución de Anaconda}.
    \end{block}
    
    \begin{columns}[T]
        \column{0.4\textwidth}
            % Nota: Aquí puedes poner una imagen del logo de Anaconda
            % \includegraphics[width=\linewidth]{Figuras/anaconda_logo.png}
            \vspace{1cm} % Espacio para la imagen
            
        \column{0.6\textwidth}
            \begin{alertblock}{¿Por qué Anaconda?}
                \begin{itemize}
                    \item Es una instalación única que incluye:
                        \begin{itemize}
                            \item El lenguaje \textbf{Python}.
                            \item \textbf{Jupyter Notebook} y \textbf{Spyder}.
                            \item Cientos de librerías científicas preinstaladas (NumPy, Pandas, Matplotlib).
                        \end{itemize}
                    \item El \textbf{Anaconda Navigator} nos da una interfaz gráfica para lanzar las aplicaciones.
                \end{itemize}
            \end{alertblock}
    \end{columns}
\end{frame}

\begin{frame}{Pasos para la Instalación}
    \begin{enumerate}
        \item<1-> Ir a la página oficial de descargas de Anaconda: \url{https://www.anaconda.com/download}
        \pause
        \item<2-> Elegir la opción \textbf{'Distribution Installers'} (la versión completa), no Miniconda.
        \pause
        \item<3-> Descargar el instalador gráfico para tu sistema operativo (Windows, macOS, Linux).
        \pause
        \item<4-> Ejecutar el instalador y seguir las instrucciones. Se recomienda dejar las opciones por defecto.
    \end{enumerate}
    \vfill
    \begin{examples}{¡Y eso es todo! Estarás listo para nuestra primera práctica de programación.}
    \end{examples}
\end{frame}

%------------------------------------------------
% --- Diapositiva de cierre ---
%------------------------------------------------
\begin{frame}
    \Huge{\centerline{¿Preguntas?}}
\end{frame}

\end{document}