\documentclass[aspectratio=169,xcolor=dvipsnames]{beamer}
\usetheme{SimpleDarkBlue}

\usepackage[spanish]{babel}
\usepackage{hyperref}
\usepackage{graphicx} % Allows including images
\usepackage{booktabs} % Allows the use of \toprule, \midrule and \bottomrule in tables
\usepackage{amsmath}
\usepackage{lettrine}
\setbeamertemplate{caption}[numbered]
\usepackage[dvipsnames,svgnames,x11names]{xcolor}% Para definir y usar colores (ej. en hipervínculos)
\usepackage{xurl}
\usepackage{hyperref}       % Para crear hipervínculos internos y externos
\usepackage{algorithm}
\usepackage{algorithmicx}
\usepackage{algpseudocode}
\hypersetup{
    colorlinks=true,        % Colorear los enlaces en lugar de usar recuadros
    linkcolor=blue,     % Color para enlaces internos (índice, referencias cruzadas)
    filecolor=blue, % Color para enlaces a archivos locales
    urlcolor=blue,      % Color para URLs
    citecolor=blue,     % Color para citas bibliográficas
}
%----------------------------------------------------------------------------------------

\usepackage{listings}
\usepackage{xcolor} % Para colores en listings
 \definecolor{codegreen}{rgb}{0,0.6,0}
 \definecolor{codegray}{rgb}{0.5,0.5,0.5}
 \definecolor{codepurple}{rgb}{0.58,0,0.82}
 \definecolor{backcolour}{rgb}{0.97,0.97,0.99}

\lstdefinestyle{PythonStyle}{
  language=Python,
  basicstyle=\ttfamily\footnotesize,
  keywordstyle=\color{blue}\bfseries,
  commentstyle=\color{codegreen},
  stringstyle=\color{violet},
  numberstyle=\tiny\color{gray},
  breakatwhitespace=false,
  breaklines=true,
  captionpos=b,
  keepspaces=true,
  numbers=left,
  numbersep=5pt,
  showspaces=false,
  showstringspaces=false,
  showtabs=false,
  tabsize=2,
  frame=lines, % Añade un marco alrededor del código
  framerule=0.4pt, % Grosor del marco
  backgroundcolor=\color{backcolour} % Color de fondo suave
}
\lstset{style=PythonStyle}
%	TITLE PAGE


\title{Subrutinas en Python (Ejercicios)}
\subtitle{Materia: Algoritmos y Programación}

\author{Prof. D.Sc. BARSEKH-ONJI Aboud}
\institute
{
    Facultad de Ingeniería \\
    Universidad Anáhuac México
}
\date{\today}

%----------------------------------------------------------------------------------------
%	CONTENIDO DE LA PRESENTACIÓN
%----------------------------------------------------------------------------------------

% --- Agenda automática al inicio de cada sección ---
\AtBeginSection[]
{
  \begin{frame}{Agenda}
    \tableofcontents[currentsection]
  \end{frame}
}

\begin{document}

\begin{frame}
    \titlepage
\end{frame}

%------------------------------------------------
%------------------------------------------------
\section{Ejercicio 1}
\begin{frame}{Ejercicio 1: El Validador de Números Primos}
    
    \begin{block}{Objetivo}
    Escribir un programa que le pida al usuario un número entero y determine si es un número primo o no.
    \end{block}
    
    \begin{alertblock}{}
    \begin{enumerate}
        \item Deben crear una \textbf{función} llamada \texttt{es\_primo(numero)}.
        \item Esta función debe recibir un número como \textbf{parámetro}.
        \item La función debe \textbf{devolver} un valor Booleano: \texttt{True} si el número es primo, o \texttt{False} si no lo es.
        \item \textbf{Lógica (Pista):} Un número es primo si solo es divisible por 1 y por sí mismo. Pueden usar un ciclo \texttt{for} que vaya desde 2 hasta \texttt{numero - 1} y revisar si el residuo (\texttt{\%}) de la división es 0.
        \item El \textbf{programa principal} debe pedir el número, llamar a la función \texttt{es\_primo()} y, basado en el resultado que \textit{retornó}, imprimir el mensaje final (ej. ''El 17 ES primo'' o ''El 18 NO es primo'').
    \end{enumerate}
    \end{alertblock}

\end{frame}

%------------------------------------------------
\section{Ejercicio 2}

\begin{frame}{Ejercicio 2: El Buscador de Coordenadas}
    
    \begin{block}{Objetivo}
    Dada una matriz (lista de listas) y un valor, encontrar la posición (fila y columna) de ese valor.
    \end{block}
    
    \begin{alertblock}{}
    \begin{enumerate}
        \item Deben crear una \textbf{función} llamada \texttt{buscar\_valor(matriz, valor\_buscado)}.
        \item Esta función debe recibir la matriz (lista 2D) y el valor a buscar como \textbf{parámetros}.
        \item \textbf{Lógica (Pista):} Deben usar \textbf{ciclos anidados} (\texttt{for}) para recorrer cada fila y cada columna.
        \item Si encuentra el valor, la función debe \textbf{devolver} las coordenadas. (Pista: pueden devolver una tupla \texttt{(fila, columna)}).
        \item Si el valor \textbf{no se encuentra} después de recorrer toda la matriz, la función debe \textbf{devolver} \texttt{None}.
        \item El \textbf{programa principal} debe definir una matriz, llamar a la función y luego usar un \texttt{if} para revisar el valor devuelto e imprimir un mensaje (ej. "Valor encontrado en (2, 1)" o "Valor no encontrado").
    \end{enumerate}
    \end{alertblock}

\end{frame}

%------------------------------------------------
\section{Ejercicio 3}

\begin{frame}{Ejercicio 3: Mini Analizador de Texto (Menú Interactivo)}
    
    \begin{block}{Objetivo}
    Crear un programa que ofrezca un menú para realizar diferentes operaciones sobre una frase que ingrese el usuario.
    \end{block}
    
    \begin{alertblock}{}
    \begin{enumerate}
        \item El \textbf{programa principal} debe pedir al usuario una frase \textbf{una sola vez} al inicio.
        \item Luego, debe entrar en un ciclo \texttt{while} que muestre un menú de opciones (ej. 1. Contar Vocales, 2. Contar Palabras, 3. Invertir Frase, 4. Salir).
        \item Deben usar una estructura \texttt{match} (o \texttt{if/elif}) para manejar la opción del usuario.
        \item Deben crear \textbf{funciones separadas} para cada operación:
            \begin{itemize}
                \item \texttt{contar\_vocales(frase)} $\rightarrow$ debe \textbf{devolver} un número.
                \item \texttt{contar\_palabras(frase)} $\rightarrow$ debe \textbf{devolver} un número (Pista: \texttt{frase.split()}).
                \item \texttt{invertir\_frase(frase)} $\rightarrow$ debe \textbf{devolver} un nuevo string (Pista: \texttt{frase[::-1]}).
            \end{itemize}
        \item El ciclo \texttt{while} principal llamará a la función correspondiente e imprimirá el resultado \textbf{devuelto} por ella.
        \item El ciclo debe terminar si el usuario elige la opción "4. Salir".
    \end{enumerate}
    \end{alertblock}

\end{frame}
%------------------------------------------------
\end{document}