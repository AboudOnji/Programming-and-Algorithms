\documentclass[aspectratio=169,xcolor=dvipsnames]{beamer}
\usetheme{SimpleDarkBlue}

\usepackage[spanish]{babel}
\usepackage{hyperref}
\usepackage{graphicx} % Allows including images
\usepackage{booktabs} % Allows the use of \toprule, \midrule and \bottomrule in tables
\usepackage{amsmath}
\usepackage{lettrine}
\setbeamertemplate{caption}[numbered]
\usepackage[dvipsnames,svgnames,x11names]{xcolor}% Para definir y usar colores (ej. en hipervínculos)
\usepackage{xurl}
\usepackage{hyperref}       % Para crear hipervínculos internos y externos
\usepackage{algorithm}
\usepackage{algorithmicx}
\usepackage{algpseudocode}
\hypersetup{
    colorlinks=true,        % Colorear los enlaces en lugar de usar recuadros
    linkcolor=blue,     % Color para enlaces internos (índice, referencias cruzadas)
    filecolor=blue, % Color para enlaces a archivos locales
    urlcolor=blue,      % Color para URLs
    citecolor=blue,     % Color para citas bibliográficas
}
%----------------------------------------------------------------------------------------

\usepackage{listings}
\usepackage{xcolor} % Para colores en listings
 \definecolor{codegreen}{rgb}{0,0.6,0}
 \definecolor{codegray}{rgb}{0.5,0.5,0.5}
 \definecolor{codepurple}{rgb}{0.58,0,0.82}
 \definecolor{backcolour}{rgb}{0.97,0.97,0.99}

\lstdefinestyle{PythonStyle}{
  language=Python,
  basicstyle=\ttfamily\footnotesize,
  keywordstyle=\color{blue}\bfseries,
  commentstyle=\color{codegreen},
  stringstyle=\color{violet},
  numberstyle=\tiny\color{gray},
  breakatwhitespace=false,
  breaklines=true,
  captionpos=b,
  keepspaces=true,
  numbers=left,
  numbersep=5pt,
  showspaces=false,
  showstringspaces=false,
  showtabs=false,
  tabsize=2,
  frame=lines, % Añade un marco alrededor del código
  framerule=0.4pt, % Grosor del marco
  backgroundcolor=\color{backcolour} % Color de fondo suave
}
\lstset{style=PythonStyle}
%	TITLE PAGE


\title{Subrutinas en Python (1)}
\subtitle{Materia: Algoritmos y Programación}

\author{Prof. D.Sc. BARSEKH-ONJI Aboud}
\institute
{
    Facultad de Ingeniería \\
    Universidad Anáhuac México
}
\date{\today}

%----------------------------------------------------------------------------------------
%	CONTENIDO DE LA PRESENTACIÓN
%----------------------------------------------------------------------------------------

% --- Agenda automática al inicio de cada sección ---
\AtBeginSection[]
{
  \begin{frame}{Agenda}
    \tableofcontents[currentsection]
  \end{frame}
}

\begin{document}

\begin{frame}
    \titlepage
\end{frame}

%------------------------------------------------
%------------------------------------------------
\section{Parte Teórica: Subrutinas}
%------------------------------------------------

\begin{frame}{¿Qué es una Subrutina?}
    
    \begin{block}{Definición}
    Una \textbf{subrutina} (que en Python llamaremos \textbf{función}) es un bloque de código \textbf{reutilizable} que realiza una tarea específica.
    \end{block}
    
    \begin{alertblock}{La Analogía de la Receta}
    Piensa en una subrutina como una \textbf{receta de cocina}:
    \begin{itemize}
        \item Tiene un \textbf{nombre} (ej. 'Receta para pastel de chocolate').
        \item Contiene una serie de \textbf{pasos} (instrucciones).
        \item Cada vez que quieres hacer el pastel, no re-escribes la receta, simplemente \textbf{sigues} la receta que ya existe.
    \end{itemize}
    \end{alertblock}
\end{frame}
\begin{frame}{¿Qué es una Subrutina?}

    \begin{block}{El Principio DRY: 'Don't Repeat Yourself'}
    El objetivo principal de las subrutinas es \textbf{NO REPETIR CÓDIGO}. Si notas que estás escribiendo las mismas 5 líneas de código en tres lugares diferentes, ¡es momento de crear una subrutina!
    \end{block}

\end{frame}

%------------------------------------------------

\begin{frame}[fragile]{Definir vs. Llamar}
    
    Para usar una subrutina, hay dos pasos clave que a menudo se confunden:
    
    \begin{columns}[t]
        \column{.5\textwidth}
            \begin{block}{1. Definir la Subrutina}
            Esto es como \textbf{escribir la receta}. Le decimos a Python qué pasos debe seguir y cómo se llama.
            \begin{itemize}
                \item Se usa la palabra clave \texttt{\textbf{def}}.
                \item El código dentro debe estar \textbf{indentado}.
            \end{itemize}
            \begin{lstlisting}[language=Python]
# Definicion de la subrutina
def saludar():
    print('¡Hola, bienvenido!')
    print('Esta es mi primera subrutina.')
            \end{lstlisting}
            \end{block}

        \column{.5\textwidth}
            \begin{alertblock}{2. Llamar a la Subrutina}
            Esto es como \textbf{preparar la receta}. Le decimos a Python: '¡Ejecuta ese bloque de código ahora!'
            \begin{itemize}
                \item Se hace escribiendo su nombre seguido de paréntesis \texttt{()}.
            \end{itemize}
            \begin{lstlisting}[language=Python]
# Llamada (o invocacion)
saludar()
saludar()
            \end{lstlisting}
            \end{alertblock}
    \end{columns}
\end{frame}
\begin{frame}[fragile]{Definir vs. Llamar}

    \begin{alertblock}{¡Importante!}
    Definir una subrutina no la ejecuta. Es solo el plano. ¡Debes \textbf{llamarla} para que haga algo!
    \end{alertblock}
\end{frame}

%------------------------------------------------
\section{Paso de Argumentos}
%------------------------------------------------

\begin{frame}[fragile]{¿Por qué necesitamos ''Argumentos''?}
    
    \begin{block}{El Problema}
    Nuestra subrutina \texttt{saludar()} es muy aburrida. Siempre hace exactamente lo mismo. ¿Qué pasa si queremos que salude a una persona \textbf{específica}?
    \begin{lstlisting}[language=Python]
# ?Como le decimos a saludar()
# que salude a 'Ana'?
saludar() # Imprime '¡Hola, bienvenido!'

# ?Y si ahora queremos que salude a 'Luis'?
saludar() # Sigue imprimiendo '¡Hola, bienvenido!'
    \end{lstlisting}
    \end{block}
\end{frame}
\begin{frame}[fragile]{¿Por qué necesitamos ''Argumentos''?}

    \begin{alertblock}{La Solución: Parámetros y Argumentos}
    Necesitamos una forma de \textbf{pasar información} (como el nombre 'Ana') \textbf{hacia adentro} de la subrutina.
    \begin{itemize}
        \item En la analogía de la receta, estos son los \textbf{ingredientes}. La receta 'Hacer pastel' necesita 'Harina', 'Huevos', 'Chocolate'.
    \end{itemize}
    \end{alertblock}

\end{frame}

%------------------------------------------------

\begin{frame}[fragile]{Parámetros vs. Argumentos}
    
    Es crucial entender esta terminología:
    
    \begin{block}{Parámetro (El Espacio de Estacionamiento)}
    Un \textbf{parámetro} es la variable que se declara \textbf{dentro de los paréntesis} al \textbf{definir} la subrutina. Es un ''espacio reservado'' que espera recibir un valor.
    \begin{lstlisting}[language=Python]
# 'nombre' es un PARAMETRO
def saludar_a(nombre):
    print(f'¡Hola, {nombre}! Bienvenido.')
    \end{lstlisting}
    \end{block}
\end{frame}
\begin{frame}[fragile]{Parámetros vs. Argumentos}

    \begin{alertblock}{Argumento (El Coche)}
    Un \textbf{argumento} es el \textbf{valor real} que se envía a la subrutina al \textbf{llamarla}. Es el ''coche'' que ocupa el espacio de estacionamiento.
    \begin{lstlisting}[language=Python]
# 'Ana' y 'Luis' son ARGUMENTOS
saludar_a('Ana')
saludar_a('Luis')
    \end{lstlisting}
    \end{alertblock}

\end{frame}

%------------------------------------------------

\begin{frame}[fragile]{Ejemplo 1: Subrutina con Múltiples Argumentos}
    
    \begin{block}{Objetivo}
    Crear una subrutina que presente a una persona, recibiendo su nombre y su edad.
    \end{block}
    
          

            \begin{alertblock}{Salida en Consola}
                \begin{verbatim}
--- Presentacion 1 ---
Les presento a Elena.
Tiene 30 anios.

--- Presentacion 2 ---
Les presento a Miguel.
Tiene 25 anios.
                \end{verbatim}
            \end{alertblock}
            \begin{block}{¡Importante!}
            El orden de los argumentos importa. El primer argumento ('Elena') va al primer parámetro (nombre), y el segundo ('30') va al segundo (edad).
            \end{block}
\end{frame}

\begin{frame}[fragile]{Ejemplo 1: Subrutina con Múltiples Argumentos}

  \begin{block}{Código en Python}
                \begin{lstlisting}[language=Python]
# Definimos con dos parametros: 'nombre' y 'edad'
def presentar_persona(nombre, edad):
    print(f'Les presento a {nombre}.')
    print(f'Tiene {edad} anios.')

# Llamamos con dos argumentos:
print('--- Presentacion 1 ---')
presentar_persona('Elena', 30)

print('\n--- Presentacion 2 ---')
presentar_persona('Miguel', 25)
                \end{lstlisting}
            \end{block}
\end{frame}
%------------------------------------------------

\begin{frame}[fragile]{Ejemplo 2: Reutilizando Código Complejo}
    
    \begin{block}{Objetivo}
    ¡Recordemos nuestro ejemplo de las tablas de multiplicar! En lugar de escribir el código cada vez, encapsulémoslo en una subrutina.
    \end{block}
        

            \begin{alertblock}{Salida (Fragmento)}
                \begin{verbatim}
--- Tabla del 5 ---
5 x 1 = 5
...
5 x 10 = 50
--- Tabla del 7 ---
7 x 1 = 7
...
7 x 10 = 70
--- Tabla del 9 ---
... etc.
                \end{verbatim}
            \end{alertblock}
\end{frame}

\begin{frame}[fragile]{Ejemplo 2: Reutilizando Código Complejo}

   \begin{block}{Código en Python}
                \begin{lstlisting}[language=Python]
# Definimos la subrutina que recibe el
# numero de la tabla que queremos imprimir
def imprimir_tabla(numero_tabla):
    print(f'\n--- Tabla del {numero_tabla} ---')
    
    # Usamos el parametro en nuestro ciclo
    for i in range(1, 11):
        resultado = numero_tabla * i
        print(f'{numero_tabla} x {i} = {resultado}')

# AHORA ES MUY FACIL USARLO:
# Llamamos a la subrutina varias veces
imprimir_tabla(5)
imprimir_tabla(7)
imprimir_tabla(9)
                \end{lstlisting}
            \end{block}
\end{frame}
%------------------------------------------------

\begin{frame}{Resumen}
    
    \begin{block}{Lo que aprendimos hoy}
    \begin{itemize}
        \item \textbf{Subrutina (Función en Python):} Es un bloque de código con nombre, definido con \texttt{\textbf{def}}, que realiza una tarea.
        \item \textbf{DRY:} El objetivo es 'No Repetirse' y organizar mejor el código.
        \item \textbf{Definir vs. Llamar:} Escribir la 'receta' (\texttt{def}) no es lo mismo que 'prepararla' (llamarla con \texttt{()}).
        \item \textbf{Argumentos:} Son los 'ingredientes' que pasamos a la subrutina para que pueda trabajar con ellos.
    \end{itemize}
    \end{block}
    
    \begin{alertblock}{Para la Próxima Clase...}
    Todas nuestras subrutinas de hoy \textit{imprimen} cosas, pero no nos \textit{devuelven} un resultado.
    
    ¿Qué pasa si quiero una subrutina que \textbf{calcule} un promedio y me entregue el número para que yo pueda guardarlo en una variable?
    
    Eso es una \textbf{Función} y usaremos la palabra clave \texttt{\textbf{return}}.
    \end{alertblock}

\end{frame}
\end{document}