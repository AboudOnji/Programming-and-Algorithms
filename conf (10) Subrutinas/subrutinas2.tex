\documentclass[aspectratio=169,xcolor=dvipsnames]{beamer}
\usetheme{SimpleDarkBlue}

\usepackage[spanish]{babel}
\usepackage{graphicx} % Allows including images
\usepackage{booktabs} % Allows the use of \toprule, \midrule and \bottomrule in tables
\usepackage{amsmath}
\usepackage{lettrine}
\setbeamertemplate{caption}[numbered]
\usepackage[dvipsnames,svgnames,x11names]{xcolor}% Para definir y usar colores (ej. en hipervínculos)
\usepackage{xurl}
\usepackage{hyperref}       % Para crear hipervínculos internos y externos
\usepackage{algorithm}
\usepackage{algorithmicx}
\usepackage{algpseudocode}
\hypersetup{
    colorlinks=true,        % Colorear los enlaces en lugar de usar recuadros
    linkcolor=blue,     % Color para enlaces internos (índice, referencias cruzadas)
    filecolor=blue, % Color para enlaces a archivos locales
    urlcolor=blue,      % Color para URLs
    citecolor=blue,     % Color para citas bibliográficas
}
%----------------------------------------------------------------------------------------

\usepackage{listings}
\usepackage{listings}
 \definecolor{codegreen}{rgb}{0,0.6,0}
 \definecolor{codegray}{rgb}{0.5,0.5,0.5}
 \definecolor{codepurple}{rgb}{0.58,0,0.82}
 \definecolor{backcolour}{rgb}{0.97,0.97,0.99}
\lstdefinestyle{PythonStyle}{
  language=Python,
  basicstyle=\ttfamily\footnotesize,
  keywordstyle=\color{blue}\bfseries,
  commentstyle=\color{codegreen},
  stringstyle=\color{violet},
  numberstyle=\tiny\color{gray},
  breakatwhitespace=false,
  breaklines=true,
  captionpos=b,
  keepspaces=true,
  numbers=left,
  numbersep=5pt,
  showspaces=false,
  showstringspaces=false,
  showtabs=false,
  tabsize=2,
  frame=lines, % Añade un marco alrededor del código
  framerule=0.4pt, % Grosor del marco
  backgroundcolor=\color{backcolour} % Color de fondo suave
}
\lstset{style=PythonStyle}
%	TITLE PAGE


\title{Rutinas y Subrutinas en Python - Parte 2}
\subtitle{Materia: Algoritmos y Programación}

\author{Prof. D.Sc. BARSEKH-ONJI Aboud}
\institute
{
    Facultad de Ingeniería \\
    Universidad Anáhuac México
}
\date{\today}

%----------------------------------------------------------------------------------------
%	CONTENIDO DE LA PRESENTACIÓN
%----------------------------------------------------------------------------------------

% --- Agenda automática al inicio de cada sección ---
\AtBeginSection[]
{
  \begin{frame}{Agenda}
    \tableofcontents[currentsection]
  \end{frame}
}

\begin{document}

\begin{frame}
    \titlepage
\end{frame}

%------------------------------------------------
%------------------------------------------------
\section{Funciones y Valor de Retorno}
%------------------------------------------------

\begin{frame}[fragile]{Recordatorio: ¿Qué es una Subrutina?}
    
    \begin{block}{Lo que vimos la clase pasada}
    Una \textbf{subrutina} (o función) es un bloque de código que \textbf{hace una tarea}. Por lo general, su resultado es una acción visible, como imprimir en pantalla.
    \begin{lstlisting}[language=Python]
def imprimir_tabla(numero_tabla):
    print(f"\n--- Tabla del {numero_tabla} ---")
    for i in range(1, 11):
        # ... lineas de codigo ...
        print(f"{numero_tabla} x {i} = ...")

# La llamamos y "hace algo" (imprime)
imprimir_tabla(7)
    \end{lstlisting}
    \end{block}
\end{frame}
\begin{frame}[fragile]{Recordatorio: ¿Qué es una Subrutina?}

    \begin{alertblock}{La Limitación}
    ¿Pero qué pasa si no quiero imprimir la tabla, sino que quiero \textbf{guardar} el resultado de \texttt{7 x 8} en una variable para usarlo después? La subrutina \texttt{imprimir\_tabla} no me ''entrega'' ningún dato.
    \end{alertblock}

\end{frame}

%------------------------------------------------

\begin{frame}{¿Qué es una Función?}
    
    \begin{block}{Definición: Función}
    Una \textbf{función} es un bloque de código que, además de recibir argumentos, está diseñada para \textbf{calcular un valor} y \textbf{devolverlo} al programa principal.
    \end{block}
    
    \begin{alertblock}{La Analogía del Trabajador}
    \begin{itemize}
        \item Una \textbf{Subrutina} (clase pasada) es como un trabajador al que le das una orden ("¡imprime la tabla del 5!") y él la cumple.
        \pause
        \item Una \textbf{Función} (clase de hoy) es como un trabajador al que le das unos datos (''Calcula 5 + 3'') y él te \textbf{entrega un papel} con el resultado ("8"), para que tú decidas qué hacer con él.
    \end{itemize}
    \end{alertblock}
    
    \begin{block}{La Palabra Clave: `return`}
    Para ''entregar ese papel'' y devolver el valor, las funciones usan la palabra clave \texttt{\textbf{return}}.
    \end{block}

\end{frame}

%------------------------------------------------

\begin{frame}[fragile]{Sintaxis: `return`}
    
    \begin{block}{¿Cómo funciona `return`?}
    Cuando Python llega a una línea con \texttt{return} dentro de una función, hace dos cosas:
    \begin{enumerate}
        \item \textbf{Termina la función inmediatamente} (ignora cualquier código que venga después).
        \item \textbf{Envía de vuelta} el valor que está a la derecha de la palabra \texttt{return}.
    \end{enumerate}
    \end{block}
    
    \begin{lstlisting}[language=Python]
# Definicion de una FUNCION
def sumar(a, b):
    resultado = a + b
    return resultado  # Devuelve el valor de 'resultado'

# Definicion de una SUBRUTINA (para comparar)
def sumar_e_imprimir(a, b):
    resultado = a + b
    print(f"El resultado es: {resultado}") # No devuelve nada
    \end{lstlisting}
    
\end{frame}

%------------------------------------------------

\begin{frame}[fragile]{¡La Clave! Capturando el Valor de Retorno}
    
    \begin{block}{¿Cómo usamos el valor devuelto?}
    El valor que \texttt{return} envía de vuelta puede ser ''atrapado'' en una variable usando el signo de asignación (\texttt{=}).
    \end{block}
    
            \begin{block}{Código en Python}
                \begin{lstlisting}[language=Python]
# Definimos la funcion
def sumar(a, b):
    print("...calculando...")
    return a + b
    print("Esto NUNCA se ejecuta")

# --- Programa Principal ---
mi_suma = sumar(10, 5)
print(f"El valor guardado es: {mi_suma}")
print(f"El doble es: {mi_suma * 2}")
                \end{lstlisting}
            \end{block}
\end{frame}
\begin{frame}[fragile]{¡La Clave! Capturando el Valor de Retorno}

            \begin{alertblock}{Salida en Consola}
                \begin{verbatim}
...calculando...
El valor guardado es: 15
El doble es: 30
                \end{verbatim}
            \end{alertblock}
            \begin{block}{Análisis}
            \begin{itemize}
                \item La línea \texttt{mi\_suma = sumar(10, 5)} ejecuta la función.
                \item La función \texttt{return} 15.
                \item La línea original se convierte en \texttt{mi\_suma = 15}.
                \item El \texttt{print} después del \texttt{return} fue ignorado.
            \end{itemize}
            \end{block}
\end{frame}

%------------------------------------------------

\begin{frame}[fragile]{Ejemplo 1: Función de Cálculo Simple}
    
    \begin{block}{Objetivo}
    Escribe un programa para crear una función que reciba la base y la altura de un triángulo y \textbf{devuelva} su área.
    \end{block}
                \begin{alertblock}{Salida en Consola}
                \begin{verbatim}
El area del primer triangulo es: 25.0
El area del segundo triangulo es: 10.5
La suma de las areas es: 35.5
                \end{verbatim}
            \end{alertblock}
            \begin{block}{Ventaja}
            ¡Podemos llamar a la función tantas veces como queramos con diferentes argumentos y reutilizar los resultados!
            \end{block}
\end{frame}

\begin{frame}[fragile]{Ejemplo 1: Función de Cálculo Simple}

     \begin{block}{Código en Python}
                \begin{lstlisting}[language=Python]
# 1. Definimos la funcion
def calcular_area_triangulo(base, altura):
    area = (base * altura) / 2
    return area

# --- Programa Principal ---

# 2. Llamamos a la funcion y guardamos resultados
area1 = calcular_area_triangulo(10, 5)
area2 = calcular_area_triangulo(7, 3)

# 3. Usamos los resultados
print(f"El area del primer triangulo es: {area1}")
print(f"El area del segundo triangulo es: {area2}")
print(f"La suma de las areas es: {area1 + area2}")
                \end{lstlisting}
            \end{block}
\end{frame}
            %------------------------------------------------

\begin{frame}[fragile]{Ejemplo 2: Refactorizando Código (Avanzado)}
    
    \begin{block}{Objetivo}
    Vamos a mejorar nuestro código de promedios de la clase pasada. Crearemos una función que \textbf{reciba una lista} y \textbf{devuelva su promedio} usando una función que llamaremos \texttt{calcular\_promedio}.
    \end{block}
\end{frame}
\begin{frame}[fragile]{Ejemplo 2: Refactorizando Código (Avanzado)}

    \begin{lstlisting}[language=Python]
# --- Definicion de la Funciona ---
def calcular_promedio(lista_de_notas):
    suma = 0
    for nota in lista_de_notas:
        suma += nota
    
    promedio = suma / len(lista_de_notas)
    return promedio # Devuelve el numero calculado

# --- Programa Principal (MAS LIMPIO) ---
calif_estudiante1 = [10, 9, 10]
calif_estudiante2 = [8, 7, 8]

# Usamos nuestra funcion
prom1 = calcular_promedio(calif_estudiante1)
prom2 = calcular_promedio(calif_estudiante2)

print(f"El promedio del Estudiante 1 es: {prom1:.2f}")
print(f"El promedio del Estudiante 2 es: {prom2:.2f}")
    \end{lstlisting}
\end{frame}

%------------------------------------------------

\begin{frame}{Resumen y Próxima Clase}
    
    \begin{block}{Resumen: Funciones y `return`}
    \begin{itemize}
        \item \textbf{Función:} Es un bloque de código (definido con \texttt{def}) que \textbf{calcula} y \textbf{devuelve} un valor.
        \item \textbf{`return`:} Es la palabra clave que \textbf{finaliza} la función y \textbf{envía un valor de vuelta}.
        \item \textbf{Capturar el Valor:} Usamos una variable y el signo \texttt{=} para "atrapar" el valor que la función devuelve (ej. \texttt{mi\_variable = mi\_funcion()}).
        \item \textbf{Ventaja:} Esto nos permite usar el resultado en cálculos futuros, guardarlo, o pasarlo como argumento a \textit{otra} función.
    \end{itemize}
    \end{block}
    

\end{frame}
\end{document}