\documentclass[aspectratio=169,xcolor=dvipsnames]{beamer}
\usetheme{SimpleDarkBlue}

\usepackage[spanish]{babel}
\usepackage{hyperref}
\usepackage{graphicx} % Allows including images
\usepackage{booktabs} % Allows the use of \toprule, \midrule and \bottomrule in tables
\usepackage{amsmath}
\usepackage{lettrine}
\setbeamertemplate{caption}[numbered]
\usepackage[dvipsnames,svgnames,x11names]{xcolor}% Para definir y usar colores (ej. en hipervínculos)
\usepackage{xurl}
\usepackage{hyperref}       % Para crear hipervínculos internos y externos
\usepackage{algorithm}
\usepackage{algorithmicx}
\usepackage{algpseudocode}
\hypersetup{
    colorlinks=true,        % Colorear los enlaces en lugar de usar recuadros
    linkcolor=blue,     % Color para enlaces internos (índice, referencias cruzadas)
    filecolor=blue, % Color para enlaces a archivos locales
    urlcolor=blue,      % Color para URLs
    citecolor=blue,     % Color para citas bibliográficas
}
%----------------------------------------------------------------------------------------

\usepackage{listings}
\usepackage{xcolor} % Para colores en listings
 \definecolor{codegreen}{rgb}{0,0.6,0}
 \definecolor{codegray}{rgb}{0.5,0.5,0.5}
 \definecolor{codepurple}{rgb}{0.58,0,0.82}
 \definecolor{backcolour}{rgb}{0.97,0.97,0.99}

\lstdefinestyle{PythonStyle}{
  language=Python,
  basicstyle=\ttfamily\footnotesize,
  keywordstyle=\color{blue}\bfseries,
  commentstyle=\color{codegreen},
  stringstyle=\color{violet},
  numberstyle=\tiny\color{gray},
  breakatwhitespace=false,
  breaklines=true,
  captionpos=b,
  keepspaces=true,
  numbers=left,
  numbersep=5pt,
  showspaces=false,
  showstringspaces=false,
  showtabs=false,
  tabsize=2,
  frame=lines, % Añade un marco alrededor del código
  framerule=0.4pt, % Grosor del marco
  backgroundcolor=\color{backcolour} % Color de fondo suave
}
\lstset{style=PythonStyle}
%	TITLE PAGE
\title{Dibujando Funciones con Python}
\subtitle{Materia: Algoritmos y Programación}

\author{Prof. D.Sc. BARSEKH-ONJI Aboud}
\institute
{
    Facultad de Ingeniería \\
    Universidad Anáhuac México
}
\date{\today}

%----------------------------------------------------------------------------------------
%   CONTENIDO DE LA PRESENTACIÓN
%----------------------------------------------------------------------------------------

% --- Agenda automática al inicio de cada sección ---
\AtBeginSection[]
{
  \begin{frame}{Agenda}
    \tableofcontents[currentsection]
  \end{frame}
}

\begin{document}

\begin{frame}
    \titlepage
\end{frame}

%------------------------------------------------
\section{Introducción y Herramientas}
%------------------------------------------------

\begin{frame}[fragile]
    \frametitle{¿Por qué Dibujar Funciones?}
    
    \begin{block}{Más allá de \texttt{print()}}
    Hasta ahora, la única forma de ver un resultado es con la función \texttt{print()}. Esto funciona para texto y números, pero es muy poco intuitivo para entender una ecuación.
    \end{block}
    
    \begin{alertblock}{Ver para Entender}
    Para comprender realmente una función, como $y = x^2$, necesitamos \textbf{ver su gráfico}. En programación, a esto se le llama \textbf{visualización} o \textbf{"plotteo"}.
    \end{alertblock}
    
    \begin{exampleblock}{Aplicaciones}
        \begin{itemize}
            \item Visualizar datos de un experimento.
            \item Analizar el rendimiento de un algoritmo.
            \item Simular un fenómeno físico.
            \item ¡Entender las matemáticas que estamos programando!
        \end{itemize}
    \end{exampleblock}

\end{frame}

%------------------------------------------------

\begin{frame}[fragile]
    \frametitle{Las Herramientas Esenciales}
    
    \begin{block}{Python no puede hacerlo solo}
    Python estándar no incluye herramientas de graficación. Para esto, usamos \textbf{bibliotecas externas}. Dos de las más importantes en el mundo de la ciencia y la ingeniería son:
    \end{block}
    
    \begin{columns}[t]
        \column{.48\textwidth}
            \begin{alertblock}{NumPy (Numerical Python)}
                Es la biblioteca para cálculos numéricos avanzados.
                \begin{itemize}
                    \item Su objeto principal es el \textbf{array}.
                    \item La usaremos para \textbf{crear los datos} (nuestros puntos en el eje X) de forma rápida y eficiente.
                \end{itemize}
            \end{alertblock}

        \column{.48\textwidth}
            \begin{exampleblock}{Matplotlib (Plotting Library)}
                Es la biblioteca más popular para dibujar y crear gráficos en Python.
                \begin{itemize}
                    \item Usaremos su módulo \texttt{pyplot}.
                    \item La usaremos para tomar nuestros datos y \textbf{crear el dibujo} (el gráfico).
                \end{itemize}
            \end{exampleblock}
    \end{columns}

\end{frame}

%------------------------------------------------

\begin{frame}[fragile]
    \frametitle{Instalación (Un solo paso)}
    
    \begin{block}{Instalando las Bibliotecas}
    Como \texttt{numpy} y \texttt{matplotlib} son bibliotecas externas, debemos instalarlas en nuestra computadora \textbf{una sola vez}.
    \end{block}
    
    \begin{alertblock}{Usando \texttt{pip}}
    Abrimos una terminal o símbolo del sistema (no es el script de Python) y escribimos el siguiente comando:
    \begin{lstlisting}[language=bash]
# Esto descarga e instala ambas bibliotecas
pip install numpy matplotlib
    \end{lstlisting}
    \end{alertblock}
    
    \begin{examples}{Nota}
    Esto solo se hace una vez. Una vez instaladas, ya puedes usarlas en todos tus programas de Python simplemente con \texttt{import}.
    \end{examples}

\end{frame}

%------------------------------------------------
\section{Nuestro Primer Gráfico}
%------------------------------------------------

\begin{frame}[fragile]
    \frametitle{El "Hola, Mundo" del Ploteo}
    
    \begin{block}{Objetivo: Dibujar $y = x^2$}
    Vamos a crear un gráfico simple para la función $y = x^2$ en el rango de $x = -10$ a $x = 10$.
    \end{block}
\end{frame}

\begin{frame}[fragile]
    \frametitle{El "Hola, Mundo" del Ploteo}
    \begin{alertblock}{Código Completo}
    Este es el código mínimo para crear un gráfico:
    \begin{lstlisting}[language=Python]
import numpy as np
import matplotlib.pyplot as plt

# 1. Preparar los datos
x = np.linspace(-10, 10, 100) # 100 puntos entre -10 y 10
y = x**2                      # Calcula el cuadrado de CADA punto

# 2. Dibujar el grafico
plt.plot(x, y) # Crea el grafico en memoria

# 3. Mostrar el grafico
plt.show()     # Abre la ventana con el resultado
    \end{lstlisting}
    \end{alertblock}

\end{frame}

%------------------------------------------------

\begin{frame}[fragile]
    \frametitle{Análisis (Paso 1): Los Datos con NumPy}
    
    \begin{block}{\texttt{import numpy as np}}
    Importamos la biblioteca y le damos un "alias" o apodo (`np`) para escribir menos. Es una convención estándar.
    \end{block}
    
    \begin{alertblock}{\texttt{x = np.linspace(-10, 10, 100)}}
    Este es el comando clave de NumPy para nosotros.
    \begin{itemize}
        \item \textbf{\texttt{linspace}} significa "espacio lineal".
        \item Genera un array (similar a una lista) con \textbf{100} números, espaciados uniformemente entre \textbf{-10} y \textbf{10}.
        \item Estos 100 números serán nuestros "puntos X".
    \end{itemize}
    \end{alertblock}
    
    \begin{examples}{\texttt{y = x**2}}
    ¡Aquí está la magia de NumPy! No necesitamos un ciclo \texttt{for}. Al elevar el \textit{array} \texttt{x} al cuadrado, NumPy aplica la operación a \textbf{cada uno de los 100 elementos} automáticamente y nos devuelve un nuevo array \texttt{y} con los resultados.
    \end{examples}

\end{frame}

%------------------------------------------------

\begin{frame}[fragile]
    \frametitle{Análisis (Paso 2): El Dibujo con Matplotlib}
    
    \begin{block}{\texttt{import matplotlib.pyplot as plt}}
    De forma similar, importamos el módulo \texttt{pyplot} de \texttt{matplotlib} y le damos el alias estándar \texttt{plt}.
    \end{block}
    
    \begin{alertblock}{\texttt{plt.plot(x, y)}}
    Este es el comando de dibujo principal. Le decimos a Matplotlib: "Toma el array \texttt{x} como eje horizontal y el array \texttt{y} como eje vertical, y dibuja una línea que los conecte".
    \end{alertblock}
    
    \begin{examples}{\texttt{plt.show()}}
    Este comando le dice a Python: "He terminado de configurar mi gráfico. Ahora, por favor, abre la ventana y muéstraselo al usuario". Sin \texttt{plt.show()}, el gráfico se crea en memoria pero nunca se muestra.
    \end{examples}

\end{frame}

%------------------------------------------------
\section{Tarea 1}
%------------------------------------------------

\begin{frame}[fragile]
    \frametitle{Tarea 1: Dibuja una Función Trigonométrica}

    \begin{block}{Objetivo}
    Usando el código que acabamos de ver como plantilla, crea un programa que dibuje la función \textbf{Seno} ($y = \sin(x)$).
    \end{block}
\end{frame}
\begin{frame}[fragile]
    \frametitle{Tarea 1: Dibuja una Función Trigonométrica}

    \begin{alertblock}{Instrucciones}
    Debes graficar la función en el rango de $x = 0$ a $x = 2\pi$.
    
    \begin{lstlisting}[language=Python]
import numpy as np
import matplotlib.pyplot as plt

# Pista: NumPy tiene su propia constante 'pi'
# 1. Crear datos para x:
#    (Usa np.linspace para ir de 0 a 2*np.pi)
x = ...

# 2. Crear datos para y:
#    (Pista: NumPy tiene su propia funcion 'sin')
y = np.sin(x) 
    \end{lstlisting}
    \end{alertblock}
    
\end{frame}

\end{document}