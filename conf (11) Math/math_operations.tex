\documentclass[aspectratio=169,xcolor=dvipsnames]{beamer}
\usetheme{SimpleDarkBlue}

\usepackage[spanish]{babel}
\usepackage{hyperref}
\usepackage{graphicx} % Allows including images
\usepackage{booktabs} % Allows the use of \toprule, \midrule and \bottomrule in tables
\usepackage{amsmath}
\usepackage{lettrine}
\setbeamertemplate{caption}[numbered]
\usepackage[dvipsnames,svgnames,x11names]{xcolor}% Para definir y usar colores (ej. en hipervínculos)
\usepackage{xurl}
\usepackage{hyperref}       % Para crear hipervínculos internos y externos
\usepackage{algorithm}
\usepackage{algorithmicx}
\usepackage{algpseudocode}
\hypersetup{
    colorlinks=true,        % Colorear los enlaces en lugar de usar recuadros
    linkcolor=blue,     % Color para enlaces internos (índice, referencias cruzadas)
    filecolor=blue, % Color para enlaces a archivos locales
    urlcolor=blue,      % Color para URLs
    citecolor=blue,     % Color para citas bibliográficas
}
%----------------------------------------------------------------------------------------

\usepackage{listings}
\usepackage{xcolor} % Para colores en listings
 \definecolor{codegreen}{rgb}{0,0.6,0}
 \definecolor{codegray}{rgb}{0.5,0.5,0.5}
 \definecolor{codepurple}{rgb}{0.58,0,0.82}
 \definecolor{backcolour}{rgb}{0.97,0.97,0.99}

\lstdefinestyle{PythonStyle}{
  language=Python,
  basicstyle=\ttfamily\footnotesize,
  keywordstyle=\color{blue}\bfseries,
  commentstyle=\color{codegreen},
  stringstyle=\color{violet},
  numberstyle=\tiny\color{gray},
  breakatwhitespace=false,
  breaklines=true,
  captionpos=b,
  keepspaces=true,
  numbers=left,
  numbersep=5pt,
  showspaces=false,
  showstringspaces=false,
  showtabs=false,
  tabsize=2,
  frame=lines, % Añade un marco alrededor del código
  framerule=0.4pt, % Grosor del marco
  backgroundcolor=\color{backcolour} % Color de fondo suave
}
\lstset{style=PythonStyle}
%	TITLE PAGE


\title{Funciones Matemáticas: La Biblioteca \texttt{math}}
\subtitle{Materia: Algoritmos y Programación}

\author{Prof. D.Sc. BARSEKH-ONJI Aboud}
\institute
{
    Facultad de Ingeniería \\
    Universidad Anáhuac México
}
\date{\today}

%----------------------------------------------------------------------------------------

% --- Agenda automática al inicio de cada sección ---
\AtBeginSection[]
{
  \begin{frame}{Agenda}
    \tableofcontents[currentsection]
  \end{frame}
}

\begin{document}

\begin{frame}
    \titlepage
\end{frame}

%------------------------------------------------
\section{Introducción a la Biblioteca \texttt{math}}
%------------------------------------------------

\begin{frame}[fragile]
    \frametitle{¿Qué es la Biblioteca \texttt{math}?}
    
    \begin{block}{¿Por qué la necesitamos?}
    Los operadores básicos de Python (`+`, `-`, `*`, `/`, `**`) son excelentes para aritmética simple. Pero, ¿qué pasa si necesitamos calcular la raíz cuadrada, el seno de un ángulo o un logaritmo?
    \end{block}
    
    \begin{alertblock}{Una Caja de Herramientas Matemáticas}
    La biblioteca \texttt{math} es un módulo incorporado en Python que nos da acceso a un conjunto de funciones y constantes matemáticas avanzadas, estandarizadas y optimizadas.
    \end{alertblock}
\end{frame}
\begin{frame}[fragile]
    \frametitle{¿Qué es la Biblioteca \texttt{math}?}

    \begin{examples}{¿Cómo se usa? El comando \texttt{import}}
    Para poder usar las funciones de esta biblioteca, \textbf{primero debemos importarla} al inicio de nuestro programa.
    \begin{lstlisting}[language=Python]
import math
    \end{lstlisting}
    \end{examples}

\end{frame}

%------------------------------------------------

\begin{frame}[fragile]
    \frametitle{Sintaxis: Cómo Llamar a una Función}
    
    \begin{block}{Usando la Notación de Punto ('.')}
    Una vez que hemos importado la biblioteca, usamos la sintaxis \texttt{math.nombre\_funcion()} para llamar a la función que necesitamos.
    \end{block}
    
    \begin{columns}[t]
        \column{.48\textwidth}
            \begin{exampleblock}{Ejemplo Correcto}
                \begin{lstlisting}[language=Python]
import math

raiz = math.sqrt(25)
print(raiz) # Imprime 5.0
                \end{lstlisting}
            \end{exampleblock}

        \column{.48\textwidth}
            \begin{alertblock}{Error Común}
                \begin{lstlisting}[language=Python]
# Incorrecto:
# Python no conoce una funcion
# llamada 'sqrt' por si sola.
raiz = sqrt(25)

# NameError: name 'sqrt' is not defined
                \end{lstlisting}
            \end{alertblock}
    \end{columns}

\end{frame}

%------------------------------------------------
\section{Funciones Esenciales}
%------------------------------------------------

\begin{frame}[fragile]
    \frametitle{Constantes Fundamentales}
    
    \begin{block}{Valores Predefinidos}
    La biblioteca \texttt{math} nos da constantes matemáticas universales para que no tengamos que escribirlas a mano (¡y con mayor precisión!).
    \end{block}
    
    \begin{columns}[t]
        \column{.48\textwidth}
            \begin{alertblock}{\texttt{math.pi}}
                El número Pi ($\pi$), la relación entre la circunferencia de un círculo y su diámetro.
                \begin{lstlisting}[language=Python]
import math
print(math.pi)
# Imprime: 3.141592653589793
                \end{lstlisting}
            \end{alertblock}

        \column{.48\textwidth}
            \begin{exampleblock}{\texttt{math.e}}
                El número de Euler ($e$), la base de los logaritmos naturales.
                \begin{lstlisting}[language=Python]
import math
print(math.e)
# Imprime: 2.718281828459045
                \end{lstlisting}
            \end{exampleblock}
    \end{columns}
\end{frame}
\begin{frame}[fragile]
    \frametitle{Constantes Fundamentales}
    \begin{block}{Ejemplo de Uso: Área de un Círculo}
    \begin{lstlisting}[language=Python]
import math
radio = 5
area = math.pi * (radio ** 2)
print(f"El area es: {area}")
# Imprime: El area es: 78.5398...
    \end{lstlisting}
    \end{block}
\end{frame}

%------------------------------------------------

\begin{frame}[fragile]
    \frametitle{Funciones de Potencia y Logaritmos}
    
    \begin{columns}[t]
        \column{.48\textwidth}
            \begin{block}{Raíz Cuadrada: \texttt{math.sqrt()}}
                Calcula la raíz cuadrada de un número. Es más rápido y preciso que usar \texttt{x ** 0.5}.
                \begin{lstlisting}[language=Python]
import math
# Raiz cuadrada de 81
raiz = math.sqrt(81)
print(raiz) # Imprime: 9.0
                \end{lstlisting}
            \end{block}
    

        \column{.48\textwidth}
            \begin{alertblock}{Potencia: \texttt{math.pow()}}
                Eleva un número \texttt{x} a la potencia \texttt{y}. Similar a \texttt{x ** y}, pero \texttt{pow()} siempre devuelve un flotante.
                \begin{lstlisting}[language=Python]
import math
# 3 elevado a la 4
potencia = math.pow(3, 4)
print(potencia) # Imprime: 81.0
                \end{lstlisting}
            \end{alertblock}
            
    \end{columns}
\end{frame}

\begin{frame}[fragile]
    \frametitle{Funciones de Potencia y Logaritmos}
    
    \begin{columns}[t]
        \column{.48\textwidth}
            
            
            \begin{block}{Logaritmos: \texttt{math.log ()}}
                \begin{itemize}
                    \item \texttt{math.log (x)}: Logaritmo natural (base $e$).
                    \item \texttt{math.log10 (x)}: Logaritmo base 10.
                    \item \texttt{math.log (x, base)}: Logaritmo en cualquier base.
                \end{itemize}
            \end{block}
            

        \column{.48\textwidth}


            \begin{exampleblock}{Exponencial: \texttt{math.exp ()}}
                Calcula $e^x$.
                 \begin{lstlisting}[language=Python]
import math
# e elevado a la 1
resultado = math.exp(1)
print(resultado) # Imprime: 2.71828...
                \end{lstlisting}
            \end{exampleblock}
    \end{columns}
\end{frame}

%------------------------------------------------

\begin{frame}[fragile]
    \frametitle{Redondeo: \texttt{floor()}, \texttt{ceil()} y \texttt{trunc()}}
    
    \begin{block}{Controlando los Decimales}
    Estas funciones nos permiten "limpiar" números flotantes, pero lo hacen de formas muy diferentes.
    \end{block}
\end{frame}

\begin{frame}[fragile]
    \frametitle{Redondeo: \texttt{floor()}, \texttt{ceil()} y \texttt{trunc()}}
    
    \begin{columns}[t]
        \column{.48\textwidth}
            \begin{alertblock}{\texttt{math.floor(x)} (Piso)}
                Redondea un número \textbf{hacia abajo} al entero más cercano.
                \begin{lstlisting}[language=Python]
import math
x = 9.8
print(math.floor(x)) # Imprime: 9
                \end{lstlisting}
            \end{alertblock}
            
            

        \column{.48\textwidth}
            \begin{exampleblock}{\texttt{math.ceil(x)} (Techo)}
                Redondea un número \textbf{hacia arriba} al entero más cercano.
                 \begin{lstlisting}[language=Python]
import math
x = 9.2
print(math.ceil(x)) # Imprime: 10
                \end{lstlisting}
            \end{exampleblock}
            
    
    \end{columns}
\end{frame}

\begin{frame}[fragile]
    \frametitle{Redondeo: \texttt{floor()}, \texttt{ceil()} y \texttt{trunc()}}
    
    \begin{columns}[t]
        \column{.48\textwidth}
            
            
            \begin{alertblock}{\texttt{math.trunc(x)} (Truncar)}
                Simplemente \textbf{corta los decimales}, sin redondear. Para números positivos, es igual a \texttt{floor()}.
                \begin{lstlisting}[language=Python]
import math
print(math.trunc(9.8)) # Imprime: 9
print(math.trunc(-3.7)) # Imprime: -3
                \end{lstlisting}
            \end{alertblock}

        \column{.48\textwidth}
            
             \begin{exampleblock}{\texttt{round(x)} (Función Nativa)}
                Redondea al entero más cercano (el .5 lo redondea al par más cercano). \textbf{No requiere} \texttt{import math}.
                 \begin{lstlisting}[language=Python]
# No se necesita importar math
print(round(9.8)) # Imprime: 10
print(round(9.2)) # Imprime: 9
                \end{lstlisting}
            \end{exampleblock}
    \end{columns}
\end{frame}

%------------------------------------------------

\begin{frame}[fragile]
    \frametitle{Trigonometría: \texttt{sin()}, \texttt{cos()}, \texttt{tan()}}
    
    \begin{block}{Funciones Trigonométricas}
    La biblioteca \texttt{math} incluye todas las funciones trigonométricas estándar.
    \begin{itemize}
        \item \texttt{math.sin(x)}
        \item \texttt{math.cos(x)}
        \item \texttt{math.tan(x)}
        \item Y también las inversas: \texttt{math.asin(x)}, \texttt{math.acos(x)}, \texttt{math.atan(x)}
    \end{itemize}
    \end{block}

\end{frame}
\begin{frame}[fragile]
    \frametitle{Trigonometría: \texttt{sin()}, \texttt{cos()}, \texttt{tan()}}
    
    
    \begin{alertblock}{¡ADVERTENCIA! Python usa Radianes}
    Todas las funciones trigonométricas en Python (y la mayoría de los lenguajes de programación) esperan que los ángulos estén medidos en \textbf{RADIANES}, no en Grados.
    
    Este es el error más común al usar estas funciones.
    \begin{lstlisting}[language=Python]
import math

angulo_rad = math.pi / 2
resultado = math.sin(angulo_rad)
print(resultado)
    \end{lstlisting}
    \end{alertblock}
\end{frame}

%------------------------------------------------

\begin{frame}[fragile]
    \frametitle{Conversión de Ángulos: \texttt{radians()} y \texttt{degrees()}}
    
    \begin{block}{La Solución Fácil}
    Para evitar tener que hacer las conversiones a mano (como \texttt{grados * math.pi / 180}), la biblioteca \texttt{math} nos da dos funciones de ayuda.
    \end{block}
    
    \begin{columns}[t]
        \column{.48\textwidth}
            \begin{alertblock}{\texttt{math.radians(grados)}}
                Convierte un ángulo de grados a radianes.
                \begin{lstlisting}[language=Python]
import math
angulo_grados = 90
angulo_radianes = math.radians(angulo_grados)

print(angulo_radianes)
# Imprime: 1.5707... (que es pi/2)
                \end{lstlisting}
            \end{alertblock}

        \column{.48\textwidth}
            \begin{exampleblock}{\texttt{math.degrees(radianes)}}
                Convierte un ángulo de radianes a grados.
                 \begin{lstlisting}[language=Python]
import math
angulo_radianes = math.pi
angulo_grados = math.degrees(angulo_radianes)

print(angulo_grados)
# Imprime: 180.0
                \end{lstlisting}
            \end{exampleblock}
    \end{columns}
    
    \begin{block}{Ejemplo de Uso Correcto}
    \begin{lstlisting}[language=Python]
import math
# Calcular el coseno de 60 grados
angulo_grados = 60
angulo_radianes = math.radians(angulo_grados)
resultado = math.cos(angulo_radianes)
print(resultado) # Imprime: 0.5
    \end{lstlisting}
    \end{block}
\end{frame}

%------------------------------------------------
\section{Tarea Práctica}
%------------------------------------------------

\begin{frame}[fragile]
    \frametitle{Tarea: Calculadora de Triángulo Rectángulo 📐}

    \begin{block}{Objetivo}
    Escribir un programa que utilice la biblioteca \texttt{math} para resolver un triángulo rectángulo. El usuario debe proveer el valor de los dos catetos.
    \end{block}
\end{frame}

\begin{frame}[fragile]
    \frametitle{Tarea: Calculadora de Triángulo Rectángulo 📐}

    \begin{alertblock}{Instrucciones}
    Tu programa debe hacer lo siguiente:
    \begin{enumerate}
        \item \textbf{Importar} la biblioteca \texttt{math}.
        \item \textbf{Pedir} al usuario la longitud del \textbf{Cateto A} (un número).
        \item \textbf{Pedir} al usuario la longitud del \textbf{Cateto B} (un número).
        \item \textbf{Calcular y mostrar} la longitud de la \textbf{Hipotenusa}.
        \begin{itemize}
            \item Pista: Teorema de Pitágoras ($c = \sqrt{a^2 + b^2}$). Usa \texttt{math.sqrt()}.
        \end{itemize}
        \item \textbf{Calcular y mostrar} los dos ángulos agudos (Ángulo A y Ángulo B) en \textbf{GRADOS}.
        \begin{itemize}
            \item Pista: Usa una función trigonométrica inversa como \texttt{math.atan()} (arco tangente).
            \item Pista: $\tan(A) = \text{Opuesto} / \text{Adyacente} = a / b$.
            \item Pista: ¡Recuerda convertir el resultado de radianes a grados usando \texttt{math.degrees()}!
        \end{itemize}
    \end{enumerate}
    \end{alertblock}

\end{frame}
\end{document}